%% \begin{deluxetable}{} command tell LaTeX how many columns
%% there are and how to align them.
\startlongtable
\begin{deluxetable*}{llll}
    
%% Keep a portrait orientation

%% Over-ride the default font size
%% Use Default (12pt)
\tabletypesize{\scriptsize}

%% Use \tablewidth{?pt} to over-ride the default table width.
%% If you are unhappy with the default look at the end of the
%% *.log file to see what the default was set at before adjusting
%% this value.

%% This is the title of the table.
\tablecaption{\sn\ radial velocities.}
\label{tab:rv}

%% This command over-rides LaTeX's natural table count
%% and replaces it with this number.  LaTeX will increment 
%% all other tables after this table based on this number
%\tablenum{4}

%% The \tablehead gives provides the column headers.  It
%% is currently set up so that the column labels are on the
%% top line and the units surrounded by ()s are in the 
%% bottom line.  You may add more header information by writing
%% another line between these lines. For each column that requries
%% extra information be sure to include a \colhead{text} command
%% and remember to end any extra lines with \\ and include the 
%% correct number of &s.
\tablehead{
  \colhead{Time [BJD$_\mathrm{TDB}$]} &
  \colhead{RV [m$\,$s$^{-1}$]} &
  \colhead{$\sigma_{\rm RV}$ [m$\,$s$^{-1}$]} & 
  \colhead{S-value}
}

%% All data must appear between the \startdata and \enddata commands
\startdata
 2459433.801306 & 152.97 & 12.29 & 0.7355 \\
 2459433.812255 & 100.5 & 13.23 & 0.7434 \\
\enddata

%% Include any \tablenotetext{key}{text}, \tablerefs{ref list},
%% or \tablecomments{text} between the \enddata and 
%% \end{deluxetable} commands

%% General table comment marker
\tablecomments{
Table~\ref{tab:rv} is published in its entirety in a
machine-readable format. 
Example entries are shown for guidance regarding form and content.
}
\end{deluxetable*}
