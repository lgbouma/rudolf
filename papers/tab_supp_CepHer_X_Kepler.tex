%% \begin{deluxetable}{} command tell LaTeX how many columns
%% there are and how to align them.
%\startlongtable
\begin{deluxetable*}{lll}
    
%% Keep a portrait orientation

%% Over-ride the default font size
%% Use Default (12pt)
%\tabletypesize{\scriptsize}
\tabletypesize{\small}

%% Use \tablewidth{?pt} to over-ride the default table width.
%% If you are unhappy with the default look at the end of the
%% *.log file to see what the default was set at before adjusting
%% this value.

%% This is the title of the table.
\tablecaption{Candidate Cep-Her members observed by Kepler}
\label{tab:cepherkepler}
%% This command over-rides LaTeX's natural table count
%% and replaces it with this number.  LaTeX will increment 
%% all other tables after this table based on this number
%\tablenum{3}

%% The \tablehead gives provides the column headers.  It
%% is currently set up so that the column labels are on the
%% top line and the units surrounded by ()s are in the 
%% bottom line.  You may add more header information by writing
%% another line between these lines. For each column that requries
%% extra information be sure to include a \colhead{text} command
%% and remember to end any extra lines with \\ and include the 
%% correct number of &s.
\tablehead{
  \colhead{Parameter} &
  \colhead{Example Value} &
  \colhead{Description}
}

%% All data must appear between the \startdata and \enddata commands
%
% paste from
% /Users/luke/Dropbox/proj/earhart/results/tables/NGC_2516_Prot_cleaned_header.tex
% via drivers/write_NGC2516_main_table.py
\startdata
\texttt{dr2\_source\_id}      & 2073765172933035008 & Gaia DR2 source identifier. \\
\texttt{dr3\_source\_id}      & 2073765172933035008 & Gaia (E)DR3 source identifier. \\
\texttt{kepid} & 5641711      & KIC identifier. \\
\texttt{ra} &    297.4099     & Gaia EDR3 right ascension [deg]. \\
\texttt{dec} &   40.8972      & Gaia EDR3 declination [deg]. \\
  \texttt{weight} & 0.0409    & Strength of the connectivity to other candidate cluster members. \\
  \texttt{v\_l} & -0.5061     & Longitudinal galactic velocity [km\,s$^{-1}$]. \\
  \texttt{v\_b} & -8.2328     & Latitudinal galactic velocity [km\,s$^{-1}$]. \\
  \texttt{x\_pc} & -8035.42&  & Galactocentric $X$ position coordinate [pc]. \\
  \texttt{y\_pc} & 331.44     & Galactocentric $Y$ position coordinate [pc]. \\
  \texttt{z\_pc} & 65.32      & Galactocentric $Z$ position coordinate [pc]. \\
  \texttt{kic\_dr2\_ang\_dist} & 0.298 & Separation between KIC and Gaia DR2 positions [arcsec]. \\
  \texttt{edr3\_dr2\_mag\_diff} & 0.002 & $G$-band difference between EDR3 and DR2 source match [mag]. \\
\enddata

%% Include any \tablenotetext{key}{text}, \tablerefs{ref list},
%% or \tablecomments{text} between the \enddata and 
%% \end{deluxetable} commands

%% General table comment marker
\tablecomments{
Table~\ref{tab:cepherkepler} is published in its entirety in a machine-readable
format.  One entry is shown for guidance regarding form and content.
This involved a conversion based 

is a concatenation of all candidate
NGC\,2516 members reported by \citetalias{cantatgaudin_gaia_2018},
\citetalias{kounkel_untangling_2019}, and \citetalias{meingast_2021}
based on the Gaia DR2 data.  Different levels of purity and
completeness can be achieved using different cuts on photometric
periods, periodogram powers, and lithium eqiuvalent widths.  Sets
$\mathcal{A}$ and $\mathcal{B}$ provide two possible levels of
cleaning (see Section~\ref{subsubsec:cluster}).  When the
target star is the only star present in the TESS aperture,
\texttt{nequal}$=0$, \texttt{nclose}$=1$, and \texttt{nfaint}$=1$.
The light curves are available at
\url{https://archive.stsci.edu/hlsp/cdips}.
Supplementary plots enabling analyses of individual stars
are available at \url{https://lgbouma.com/notes}.
}
\vspace{-0.5cm}
\end{deluxetable*}
