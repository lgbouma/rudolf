\begin{deluxetable*}{lllrrrrrrr}
	%
	\tablecaption{ Priors and posteriors for Kepler-1643 transit model with local
  polynomials removed.}
	\label{tab:kepler1643}
	%
	\tabletypesize{\scriptsize}
	%\tabletypesize{\small}
	%
	%\tablenum{2}
	%
	\tablehead{
		\colhead{Param.} & 
		\colhead{Unit} &
		\colhead{Prior} & 
		\colhead{Median} & 
		\colhead{Mean} & 
		\colhead{Std{.} Dev.} &
		\colhead{3\% HDI} &
		\colhead{97\% HDI} &
		\colhead{ESS} &
		\colhead{$\hat{R}-1$}
	}
  %
	\startdata
$P$ & d & $\mathcal{N}(5.34264; 0.01000)$ & 5.3426257 & 5.3426258 & 0.0000101 & 5.3426071 & 5.3426454 & 7884 & 1.1e-03 \\
$t_0^{(1)}$ & d & $\mathcal{N}(134.38100; 0.02000)$ & 134.3820412 & 134.3820408 & 0.0011172 & 134.3798834 & 134.3840798 & 7390 & 3.7e-04 \\
$\log R_{\rm p}/R_\star$ & -- & $\mathcal{U}(-6.215; 0.000)$ & -3.68806 & -3.68934 & 0.02072 & -3.72788 & -3.65275 & 4449 & -7.8e-05 \\
$b$ & -- & $\mathcal{U}(0; 1+R_{\mathrm{p}}/R_\star)$ & 0.5825 & 0.5781 & 0.0514 & 0.4848 & 0.6734 & 4705 & 1.9e-04 \\
$u_1$ & -- & Kipping 2013 & 0.257 & 0.294 & 0.214 & 0.000 & 0.678 & 5324 & 7.9e-04 \\
$u_2$ & -- & Kipping 2013 & 0.324 & 0.314 & 0.324 & -0.257 & 0.880 & 4908 & 8.4e-04 \\
$R_\star$ & $R_\odot$ & $\mathcal{N}(0.855; 0.044)$ & 0.851 & 0.851 & 0.045 & 0.766 & 0.933 & 7473 & 7.2e-04 \\
$\log g$ & cgs & $\mathcal{N}(4.502; 0.035)$ & 4.507 & 4.507 & 0.035 & 4.442 & 4.576 & 6530 & -1.4e-04 \\
$\log \sigma_f$ & -- & $\mathcal{N}(\log\langle \sigma_f \rangle; 2.000)$ & -8.520 & -8.520 & 0.019 & -8.556 & -8.486 & 7966 & 2.1e-04 \\
$\langle f \rangle$ & -- & $\mathcal{N}(1.000; 0.100)$ & 1.0000 & 1.0000 & 0.0000 & 1.0000 & 1.0000 & 7488 & 3.2e-04 \\
$R_{\rm p}/R_\star$ & -- & -- & 0.025 & 0.025 & 0.001 & 0.024 & 0.026 & 4449 & -7.8e-05 \\
$\rho_\star$ & g$\ $cm$^{-3}$ & -- & 1.943 & 1.953 & 0.191 & 1.603 & 2.313 & 6081 & 9.4e-05 \\
$R_{\rm p}$ & $R_{\mathrm{Jup}}$ & -- & 0.207 & 0.207 & 0.012 & 0.184 & 0.231 & 6326 & 2.5e-04 \\
$R_{\rm p}$ & $R_{\mathrm{Earth}}$ & -- & 2.32 & 2.32 & 0.135 & 2.062 & 2.589 & 6326 & 2.5e-04 \\
$a/R_\star$ & -- & -- & 14.312 & 14.322 & 0.465 & 13.487 & 15.228 & 6081 & 8.2e-05 \\
$\cos i$ & -- & -- & 0.041 & 0.040 & 0.005 & 0.032 & 0.049 & 4929 & 2.4e-04 \\
$T_{14}$ & hr & -- & 2.408 & 2.409 & 0.061 & 2.304 & 2.527 & 4774 & 5.3e-04 \\
$T_{13}$ & hr & -- & 2.232 & 2.233 & 0.070 & 2.109 & 2.362 & 4561 & 6.2e-04 \\
	\enddata
	%
	\tablecomments{
		ESS refers to the number of effective samples.
		$\hat{R}$ is the Gelman-Rubin convergence diagnostic.
		Logarithms in this table are base-$e$.
		$\mathcal{U}$ denotes a uniform distribution,
		and $\mathcal{N}$ a normal distribution.
    Many of the $T_{13}$ statistics may be \texttt{nan} in the event of a
    grazing transit.
     (1) The ephemeris is in units of BJKD (BJDTDB-2454833).
		%  (2) Although $\mathcal{U}(0,1+R_{\rm p}/R_\star)$ is formally
		%  correct, for this model we assumed a non-grazing transit to enable
		%  sampling in $\log \delta$.
		%  (3) The eccentricity vectors are sampled in the $(e\cos\omega,
		%  e\sin\omega)$ plane.
		%  (4) The true planet size is a factor of $((F_1+F_2)/F_1)^{1/2}$
		%  larger than that from the fit because of dilution from Kepler
		%  1627B, where $F_1$ is the flux from the primary, and $F_2$ is that
		%  from the secondary; the mean and standard deviation of $R_{\rm
		%  p}=3.817\pm0.158\,R_{\oplus}$ quoted in the text includes this correction,
		%  assuming $(F_1+F_2)/F_1\approx 1.015$.
	}
	\vspace{-0.3cm}
\end{deluxetable*}
