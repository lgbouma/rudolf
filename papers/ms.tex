%\documentclass[12pt,modern,twocolumn,tighten]{aastex63}
\documentclass[12pt,twocolumn,linenumbers]{aastex63}
%\documentclass[12pt,modern,twocolumn,tighten,linenumbers,trackchanges]{aastex63}
%\documentclass[12pt,twocolumn,tighten,linenumbers]{aastex63}
%\documentclass[12pt,twocolumn,tighten,trackchanges]{aastex63}
\usepackage{amsmath,amstext,amssymb}
\usepackage[T1]{fontenc}
\usepackage{apjfonts}
\usepackage[figure,figure*]{hypcap}
\usepackage{graphics,graphicx}
\usepackage{hyperref}
\usepackage{natbib}
\usepackage[caption=false]{subfig} % for subfloat
\usepackage{enumitem} % for specific spacing of enumerate
\usepackage{epigraph}

\renewcommand*{\sectionautorefname}{Section} %for \autoref
\renewcommand*{\subsectionautorefname}{Section} %for \autoref

\newcommand{\cn}{Cep-Her complex} % cluster name
\newcommand{\sysone}{Kepler-1627} % star system name (binary)
\newcommand{\stone}{Kepler-1627 A} % star system name (binary)
\newcommand{\plone}{Kepler-1627 Ab} % planet name
\newcommand{\systwo}{Kepler-1643} % star system name (binary)
\newcommand{\sttwo}{Kepler-1643} % star system name (binary)
\newcommand{\pltwo}{Kepler-1643 b} % planet name
\newcommand{\systhree}{KOI-7368} % star system name (binary)
\newcommand{\stthree}{KOI-7368} % star system name (binary)
\newcommand{\plthree}{KOI-7368 b} % planet name
\newcommand{\sysfour}{KOI-7913 } % star system name (binary)
\newcommand{\stfour}{KOI-7913 A} % star system name (binary)
\newcommand{\plfour}{KOI-7913 Ab} % planet name

\newcommand{\clusterage}{$38^{+6}_{-5}$\,Myr} % 

\newcommand{\npms}{1097} % 20220311_Kerr_SPYGLASS205_Members_All.csv

%
% Symbols
%
\newcommand{\kms}{\,km\,s$^{-1}$}
\newcommand{\mkms}{\,km\,s^{-1}}  % math mode
\newcommand{\ms}{\,m\,s$^{-1}$}
\newcommand{\bpmrpo}{(G_{\rm BP}-G_{\rm RP})_0}
\newcommand{\bpmrp}{G_{\rm BP}-G_{\rm RP}}

%% Reintroduced the \received and \accepted commands from AASTeX v5.2.
%% Add "Submitted to " argument.
\received{\today}
\revised{---}
\accepted{---}
%\submitjournal{AAS Journals}
\shorttitle{Kepler Mini-Neptunes in Cep-Her}

\begin{document}

\title{
  Three $\approx$40 Million Year Old Mini-Neptunes from Kepler, TESS, and Gaia
}

%\suppressAffiliations
%\NewPageAfterKeywords
\correspondingauthor{L.\,G.\,Bouma}
\email{luke@astro.caltech.edu}

\author[0000-0002-0514-5538]{L. G. Bouma}
\altaffiliation{51 Pegasi b Fellow}
\affiliation{Cahill Center for Astrophysics, California Institute of Technology, Pasadena, CA 91125, USA}

% Key authors:
% ... Kinematics
\author[0000-0002-6549-9792]{R.~Kerr} % Y
\affiliation{Department of Astronomy, The University of Texas at Austin, Austin, TX 78712, USA}% % ... Kepler correlations
%
% ... stellar rotation & the initial crossmatch
\author[0000-0002-2792-134X]{J. L. Curtis} % Y
\affiliation{Department of Astronomy, Columbia University, 550 West 120th Street, New York, NY 10027, USA}
%
% HIRES
\author[0000-0002-0531-1073]{H. Isaacson} % Y
\affiliation{Astronomy Department, University of California, Berkeley, CA 94720, USA}
%
% planet-fitting, kinematics, cluster.
\author{L. A. Hillenbrand} % R
\affiliation{Cahill Center for Astrophysics, California Institute of Technology, Pasadena, CA 91125, USA}
%
% HIRES Collaborators
\author[0000-0001-8638-0320]{A. W. Howard} % Y
\affiliation{Cahill Center for Astrophysics, California Institute of Technology, Pasadena, CA 91125, USA}
%
% AO IMAGING
\author[0000-0001-9811-568X]{A.~L.~Kraus} % Y
\affiliation{Department of Astronomy, The University of Texas at Austin, Austin, TX 78712, USA}
%
% TRES
\author[0000-0001-6637-5401]{A. Bieryla} % Y
\affiliation{Center for Astrophysics \textbar \ Harvard \& Smithsonian, 60 Garden St, Cambridge, MA 02138, USA}
%
% TRES
\author[0000-0001-9911-7388]{D. W.~Latham} % Y
\affiliation{Center for Astrophysics \textbar \ Harvard \& Smithsonian, 60 Garden St, Cambridge, MA 02138, USA}
%
% AO / NIRC2
\author[0000-0003-0967-2893]{E. A.~Petigura} % Y
\affiliation{Department of Physics \& Astronomy, University of California Los Angeles, Los Angeles, CA 90095, USA}
%
% AO / NIRC2
\author[0000-0001-8832-4488]{D. Huber} % R
\affiliation{Institute for Astronomy, University of Hawaii, 2680 Woodlawn Drive, Honolulu, HI 96822, USA}


% 208 words (250 max)
\begin{abstract}
  Stellar positions and velocities from Gaia are yielding a new and
  refined view of how stellar clusters evolve.
  Here we present an analysis of a group of $\approx$40 million
  year old stars spanning Cepheus ($l=100^\circ$) to Hercules
  ($l=40^\circ$), hereafter the Cep-Her complex.
  The group includes four known Kepler Objects of Interest:
  Kepler-1627 Ab ($R_{\rm p} = 3.85 \pm 0.11\,R_\oplus$, $P = 7.2\ {\rm days}$),
  Kepler-1643 b ($R_{\rm p} = 2.32 \pm 0.14\,R_\oplus$, $P = 5.3\ {\rm days}$),
  KOI-7368 b ($R_{\rm p} = 2.22 \pm 0.12\,R_\oplus$, $P = 6.8\ {\rm days}$), and
  KOI-7913 Ab ($R_{\rm p} = 2.34 \pm 0.18\,R_\oplus$, $P = 24.2\ {\rm days}$).
  Kepler-1627 is a Neptune-sized planet in a sub-component of the
  complex called the $\delta$\ Lyr\ cluster
  \citep{bouma_kep1627_2022}.
  Here we focus on the latter three systems, which are in other
  sub-components of the complex (RSG-5 and CH-2).
  Based on kinematic evidence from Gaia, stellar rotation periods from
  TESS, and spectroscopy, these three systems are also $\approx$40
  million years old.
  More specifically, we find that Kepler-1643 (in RSG-5) is
  $46^{+9}_{-7}$\,Myr old, while KOI-7368 and KOI-7913 (in CH-2) are
  $36^{+10}_{-8}$\,Myr old.
  Based on the transit shapes and high resolution imaging, they are
  all most likely planets, with false positive probabilities of
  $6\times10^{-9}$, $5\times10^{-3}$, and $1\times10^{-4}$ for
  Kepler-1643, KOI-7368, and KOI-7913 respectively.
  These planets are therefore the first empirical
  demonstration that mini-Neptunes with sizes of $\approx$2 Earth
  radii exist at ages of $\approx$40 million years.
\end{abstract}

\keywords{
  exoplanet evolution (491),
  open star clusters (1160),
	stellar ages (1581)
}

%%%%%%%%%%%%%%%%%%%%%%%%%%%%%%%%%%%%%%%%%%%%%%%%%%%%%%%%%%%%%%%%%%%%%%%%%%%%%%%


% * Main text <3500 words (not including acknowledgements, appendices, or other
%   supplementary)

\section{Introduction}

The discovery and characterization of transiting planets younger than
a billion years is a major frontier in current exoplanet research.
The reason is that the properties of young planets provide benchmarks
for studies of planetary evolution.  For instance, there are the
questions of when hot Jupiters arrive on their close-in orbits
\citep{dawson_johnson_2018}, how the sizes of planets with massive
gaseous envelopes evolve \citep{rizzuto_tess_2020}, when and if
close-in multiplanet systems fall out of resonance
\citep{arevalo_stability_2022,goldberg_architectures_2022}, and
whether and how mass-loss explains the radius valley
\citep{lopez_how_2012,Owen_Wu_2013,Fulton_et_al_2017,ginzburg_corepowered_2018,lee_primordial_2021}.

The discovery of a young planet requires two claims to be fulfilled:
the planet must exist, and its age must be secured.  Spaced-based
photometry from K2 and TESS has yielded a number of young planets for
which the planetary evidence comes from transits, and the age evidence
is based on either cluster membership \citep{Mann_et_al_2017,david_four_2019,newton_tess_2019,bouma_cluster_2020,nardiello_pathosII_2020}
or else on correlates of youth such as stellar rotation, photospheric
lithium abundances, x-ray activity, or emission line strength
\citep{zhou_2021_tois,hedges_toi-2076_2021}.

In this work, we leverage recent analyses of the Gaia data, which have
greatly expanded our knowledge of stellar group memberships
\citep[{\it e.g.},][]{CantatGaudin2018a,KounkelCovey2019,Kerr2021}.
Broadly speaking these analyses cluster on the stellar positions and
on-sky velocities measured by Gaia, with varying degrees of filtering
and supervision.  One important result is the identification of
diffuse streams and tidal tails comparable in stellar mass to the
previously known cores of nearby open clusters
\citep{meingast_psceri_2019,Meingast2021,gagne_number_2021}.  These
spatially diffuse groups typically have velocity dispersions of
$\approx$1\kms, though they can be much higher due to both projection
effects and internal dynamics.  As an extreme example, in the Hyades
the velocities of stars in the tidal tails are expected to
span up to $\pm 40$\kms relative to the cluster center
\citep{jerabkova_800_2021}.  The stars in such diffuse regions can be
verified to be the same age as the core members ({\it i.e.}, coeval)
through analyses of color-absolute magnitude diagrams, stellar
rotation periods \citep{curtis_tess_2019,bouma_2021_ngc2516}, and
chemical abundances \citep{hawkins_2020}.  While there are many
implications for our understanding of star formation and cluster
evolution \citep{dinnbier_tidal_2020}, a more immediate
consequence is that we now know the ages of many more stars, including
previously known planet hosts.

The prime Kepler mission \citep{borucki_kepler_2010} found most of the
currently known transiting exoplanets, and it was conducted before
Gaia.  It therefore seems sensible to revisit the Kepler field, given
our new knowlege of the stellar ages.

In this work, we expand on our previous study of a $38^{+7}_{-6}$
million year old Neptune-sized planet in the Kepler field (Kepler-1627
Ab; \citealt{bouma_kep1627_2022}).  The age of this planet was derived
based on its host star's membership in the $\delta$\ Lyr\ cluster.
Our analysis of the cluster focused on the immediate spatial and
kinematic vicinity of Kepler-1627~A in order to confirm the age of the
planet.  However it became clear that the $\delta$\ Lyr\ cluster seems
to also be part of a much larger group of similarly aged stars.  This
group, which is at a distance of $\sim$330 parsecs from the Sun,
appears to span Cepheus to Hercules (galactic longitudes, $l$, between
40$^\circ$ and 100$^\circ$), at galactic latitudes roughly between
0$^\circ$ and 20$^\circ$.  We therefore refer to it as the Cep-Her
complex.  It exhibits significant sub-structure over its $\approx$250
parsec length, and a detailed analysis of its memberships, kinematics,
and possible origin is currently being prepared by R.~Kerr and
collaborators.

Here, our focus is on the intersection of the Cep-Her complex with the
Kepler field.  Cross-matching the full set of candidate Cep-Her
members against known Kepler Objects of Interest (KOIs)
\citep{thompson_planetary_2018} yielded four candidate cluster
members: Kepler-1627, Kepler-1643, KOI-7368, and KOI-7913.  Given our
previous analysis of Kepler-1627, we will mostly focus on the latter
three.  After analyzing the relevant properties of Cep-Her
(Section~\ref{sec:cluster}), we discuss the stellar properties
(Section~\ref{sec:stars}) and validate the planetary nature of each
system using a combination of the Kepler photometry and
high-resolution imaging (Section~\ref{sec:planets}).  We conclude with
a discussion of implications for the size-evolution of close-in
mini-Neptunes (Section~\ref{sec:disc_conc}).

\section{The Cep-Her Complex}
\label{sec:cluster}

\begin{figure*}[t]
	\begin{center}
		\leavevmode
		\includegraphics[width=0.99\textwidth]{f1.pdf}
	\end{center}
	\vspace{-0.7cm}
	\caption{
  {\bf Positions and velocities of candidate members of the Cep-Her
  complex.}
  {\it Top row}: On-sky positions in galactic coordinates.  Black
  points are stars for which group membership is more secure than for
  gray points (see Section~\ref{subsec:members}).
  Kepler-1627 is in the outskirts of the $\delta$ Lyr cluster
  \citep{bouma_kep1627_2022}, which is centered at $\{ l, b\} \approx
  \{ 66^\circ, 12^\circ\}$.
  {\it Middle row}: Galactic positions.  The Sun is at $\{X, Y, Z\} =
  \{0, 0, 20.8\}$\,pc; lines of constant heliocentric distance are
  shown between 250 and 400\,pc, spaced by 50\,pc.
  {\it Bottom row}: Galactic tangential velocities (left) and
  longitudinal tangential velocity versus galactic latitude (right).
  The gray band in the lower-right shows the $\pm$1-$\sigma$
  projection of the Solar velocity with respect to the local standard
  of rest.  There is a strong spatial and kinematic overlap between
  Kepler-1643 and RSG-5 ($\{ l, b\} \approx \{ 82^\circ, 6^\circ\}$;
  $\{ X, Y, Z\} \approx \{60, 330, 50\} $ pc).  The local population
  of candidate young stars around KOI-7368 and KOI-7913 is more
  diffuse -- we call this region ``CH-2''.
  The {\bf interactive figure} enables a few different cuts to be
  shown.
	\label{fig:XYZvtang}
	}
\end{figure*}

\subsection{Previous Related Work}

Our focus is on a region of the Galaxy approximately 200 to 500
parsecs from the Sun, above the galactic plane, and spanning galactic
longitudes of roughly $40^\circ$ to $100^\circ$ degrees.  Two rich
clusters in this region are the $\delta$~Lyra cluster
\citep{stephenson_possible_1959} and RSG-5 \citep{roser_nine_2016}.
Each of these clusters was known before Gaia.  They have reported ages
between $\approx$30 and $\approx$60 million years.  Early empirical
evidence that these two clusters could be part of a large and more
diffuse population was apparent in the Gaia-based photometric analysis
of pre-main-sequence stars by \citet[][see their Figures~11
and~13]{Zari2018}.  Further kinematic connections and complexity were
highlighted by \citet{KounkelCovey2019}, who included these previously
known groups in the larger structures dubbed ``Theia~73'' and
``Theia~96''\footnote{See their visualization online at
\url{http://mkounkel.com/mw3d/mw2d.html} (accessed 15 March 2022)}.
The connection made by \citet{KounkelCovey2019} between the previously
known open clusters and the other groups in the region was made as
part of an unsupervised clustering analysis of the Gaia DR2 positions
and tangential velocities with a subsequent manual ``stitching'' step,
and generally supports the idea that there is an overdensity of
$\approx$30 and $\approx$60 million year old stars in this region of
the Galaxy.  \citet{Kerr2021}, in a volume-limited analysis of the
Gaia DR2 point-source catalog out to one third of a kiloparsec,
identified three of the nearest sub-populations, dubbed
``Cepheus-Cygnus'', ``Lyra'', and ``Cerberus''.  \citet{Kerr2021}
reported ages for each of these subgroups between 30 and 35 million
years.


\subsection{Member Selection}
\label{subsec:members}

% {\bf TODO RONAN: PLEASE VET ENTIRE SECTION, AND UPDATE WHERE
% APPROPRIATE!}.

The possibility that the $\delta$~Lyr cluster, RSG-5, and the
sub-populations identified by \citet{Kerr2021} share a common origin
has yet to be fully substantiated, and is the subject of the upcoming
study by R.~Kerr and collaborators.  Our primary interest in the
region stems from the fact that a portion of it was observed by Kepler
(Figure~\ref{fig:XYZvtang}, top panel).  To further explore the
population of stars that were observed, we select candidate Cep-Her
members through four steps, the first three being identical to those
described in Section~3 of \citet{Kerr2021}.  We briefly summarize them
here.

The first step is to select stars that are photometrically distinct
from the field star population based on Gaia EDR3 magnitudes $\{G,
G_{\rm RP}, G_{\rm BP}\}$, parallaxes and auxiliary reddening
estimates \citep{lallement_gaia-2mass_2019}.  This step yielded \npms\
stars with high-quality photometric and astrometry, which are either
pre-main-sequence K and M dwarfs due to their long contraction
timescales, or massive stars near the zero-age main sequence due to
their rapid evolutionary timescales.

The second step is to perform in an unsupervised HDBScan
clustering on the photometrically selected population
\citep{campello_hierarchical_2015,mcinnes_hdbscan_2017}.  The
parameters we use in this clustering analysis are $\{ X, Y, Z, c v_b,
c v_{l^*} \} $, where $c$ is the size-velocity corrective factor,
which is taken as $c=6\,{\rm pc / km\,s}^{-1}$ to ensure that the
spatial and velocity scales have identical standard deviations.
Positions are computed assuming the \texttt{astropy v4.0} coordinate
standard \citep{astropy_2018}, which places the Sun $8122$ pc from the
galactic center, and assumes the solar velocity with respect to the
local standard of rest from \citet{schonrich_local_2010}.  As input
parameters to HDBScan, we set the minimum $\epsilon$ threshold past
which clusters cannot be fragmented as $25$ parsecs in physical space,
and $c$ \kms\ in velocity.  The minimum cluster size $N$ is set to 10,
as is $k$, the parameter used to define the ``core distance'' density
metric. 

This unsupervised clustering in our case yielded 8 distinct groups.
These groups are then used as the ``seed'' populations for the third
step, which is to search for objects at least as close to the 10$^{\rm
th}$ nearest HDBSCAN-identified member in space-velocity coordinates.
This third step yields stars that are spatially and kinematically
close to the photometrically-young stars, but which cannot be
identified as young based on their positions in color versus absolute
magnitude.

The outcome of the analysis up to the point of the third step is shown
in Figure~\ref{fig:XYZvtang}.  To enable a selection cut that filters
out field-star contaminants, we also compute a weight metric, defined
such that the group member with the smallest core distance has a
weight of 1, the group member with the greatest core distance has a
weight of 0, and weights for the other group members are log-normally
distributed between these two extremes.  In Figure~\ref{fig:XYZvtang},
we show 26{,}960 objects with weight exceeding 0.02 as gray points,
and overplot 7{,}560 objects with weights exceeding 0.10 as black
points.  {\bf TODO: FIXME these object numbers impose pretty weak
quality cuts and include junk at G>18...  use
get\_clean\_gaia\_photometric\_sources too, since they come in for the
CAMDs}.

The previously known $\delta$~Lyr\ cluster ($l,b=68^\circ,15^\circ$;
$v_{l'}, v_b=-4.5{\rm kms},-4{\rm kms}$) is visible, as is RSG-5
($l,b=83^\circ,6^\circ$, $v_{l'}, v_b=+5.5{\rm kms},-3.5{\rm kms}$).
Most of the other subclusters, including in Cep-Cyg
($l,b=90^\circ,7^\circ$) and Cerberus ($l,b=48^\circ,18^\circ$) are
too small or dispersed to have previously been analyzed in great
detail.

%
% KOI match numbers from calc_CepHer_group_numbers.py
%
The fourth and final step was to cross-match our candidate Cep-Her
member list against all known Kepler Objects of Interest.  We used the
Cumulative KOI table from the NASA Exoplanet Archive from 27 March
2022, and also compared against the \texttt{q1\_q17\_dr25} table
\citep{thompson_planetary_2018}.  This yielded 32 matches, of which 18
were known false positives, 7 were designated
``confirmed'', and 8 were designated ``candidates''.  A cursory
inspection of the Kepler data validation summaries and Robovetter
classifications for these objects quickly showed whether they were
potentially consistent with being {\it i)} planets, and {\it ii)}
$\lesssim 10^8$ years old, based on the presence of rotational
modulation at the expected period and amplitude \citep[{\it
e.g.},][Figure~9]{rebull_rotation_2020}.  Four objects remained after
this inspection: Kepler-1627, Kepler-1643, KOI-7368, and KOI-7913.

Figure~\ref{fig:XYZvtang} shows the positions of the KOIs along
various projections.  Kepler-1643 is near the core RSG-5 population
both spatially and kinematically.  KOI-7368 and KOI-7913 are in a 
diffuse region $\approx$40 parsecs above RSG-5 in $Z$ and $\approx$100
parsecs closer to the Sun in $Y$.  In tangential galactic velocity
space, there may be some kinematic overlap between the region the
latter two KOIs are in, and the main RSG-5 group.

%
% Numbers are from RSG-5_auto_XYZ_vl_vb_cut.csv,
% CH-2_auto_XYZ_vl_vb_cut.csv
%
We define two sets of stars in the local vicinity of our objects of
interest.  For candidate RSG-5 members, we require:
\begin{align}
  X/{\rm pc} &\in [45, 75] \nonumber \\
  Y/{\rm pc} &\in [320, 350] \nonumber \\
  Z/{\rm pc} &\in [40, 70] \nonumber \\
  v_b/{\rm km\,s^{-1}} &\in [-4, -3] \nonumber \\
  v_{l^*}/{\rm km\,s^{-1}} &\in [4, 6] \nonumber
\end{align}
For the diffuse stars near KOI-7368 and KOI-7913, we require
\begin{align}
  X/{\rm pc} &\in [20, 70] \nonumber \\
  Y/{\rm pc} &\in [230, 270] \nonumber \\
  Z/{\rm pc} &\in [75, 105] \nonumber \\
  v_b/{\rm km\,s^{-1}} &\in [-3.5, -1.5] \nonumber \\
  v_{l^*}/{\rm km\,s^{-1}} &\in [2, 6] \nonumber
\end{align}
and we call this latter set of stars ``CH-2''.  These cuts yield 141
candidate RSG-5 members, and 37 candidate CH-2 members.  An important
consideration, especially for CH-2, is the
contamination rate by field stars.  We assess this in the following
section.


\subsection{The Cluster's Age}
\label{sec:clusterage}

\begin{figure*}[tp]
	\begin{center}
		\leavevmode
		\subfloat{
			\includegraphics[width=0.49\textwidth]{f2a.pdf}
			\includegraphics[width=0.49\textwidth]{f2b.pdf}
		}
		
		\vspace{-0.6cm}
		\subfloat{
			\includegraphics[width=0.49\textwidth]{f2c.pdf}
			\includegraphics[width=0.49\textwidth]{f2d.pdf}
		}
	\end{center}
	\vspace{-0.7cm}
	\caption{
		{\bf The stellar groups near KOI-7368, KOI-7913, and Kepler-1643
    are $\approx$40 million years old.} 
    {\it Top row}: 
    Color-absolute magnitude diagram of candidate Cep-Her members, in
    addition to candidate members of the $\delta$~Lyr~cluster
    ($\approx38$\,Myr; \citealt{bouma_kep1627_2022}) and and the Gaia
    EDR3 Catalog of Nearby Stars (gray background).  The left and
    right columns shows stars in RSG-5 and CH-2, respectively.  The
    range of colors is truncated to emphasize the pre-main-sequence.
    {\it Bottom row}:
    TESS and ZTF-derived stellar rotation periods, with the Pleiades
    ($\approx 112$\,Myr) and Praesepe ($\approx 650$\,Myr) shown for reference
    \citep{rebull_rotation_2016a,douglas_poking_2017}.
    The detection efficiency for reliable rotation periods falls off
    beyond $\bpmrpo \gtrsim 2.6$.
	\label{fig:age}
	}
\end{figure*}

\subsubsection{Color-Absolute Magnitude Diagram}
\label{sec:camd}

Color-absolute magnitude diagrams (CAMDs) of the candidate RSG-5 and
CH-2 members are shown in the upper row of Figure~\ref{fig:age}.  The
stars from the $\delta$~Lyr cluster are from
\citet{bouma_kep1627_2022}, and the field stars are from the Gaia EDR3
Catalog of Nearby Stars \citep{gaia_gcns_2021}.  To make these
diagrams, we imposed the data filtering criteria from
\citet[][Appendix B]{GaiaCollaboration2018}, which are designed to
include binaries while omitting instrumental artifacts from for
instance low photometric signal to noise, or a small number of
visibility periods.  We then corrected for extinction using the
\citet{lallement_threedimensional_2018}\footnote{\url{https://stilism.obspm.fr/}}
dust maps and the extinction coefficients $k_X\equiv A_X/A_0$from
\citet{GaiaCollaboration2018}, assuming that $A_0 = 3.1 E(B-V)$.  This
yielded a mean and standard deviation for the reddening of
$E(B-V)=0.036\pm0.002$ for RSG-5, and $E(B-V)=0.017\pm0.001$
for CH-2.  By way of comparison, in \citet{bouma_kep1627_2022} the
same query for the $\delta$~Lyr cluster yielded
$E(B-V)=0.032\pm0.006$.  Finally, for the plots we set the $\bpmrpo$
color range to best visualize the region of maximal age information
content: the pre-main-sequence.

The CAMDs show that for RSG-5, our membership selection gives stars
that photometrically seem to almost all be on a tight
pre-main-sequence locus.  This implies a false positive rate of 
a few percent, at most.  By comparison, our control sample (the $\delta$~Lyr
candidates) has a false positive rate of $\approx$12\%, based on the
number of stars that photometrically appear to be more consistent with
the field population than the bulk cluster population.  For CH-2, our
membership selection gives 27 objects in the color range displayed,
and 23 of them appear to be consistent with being on the
pre-main-sequence.  This implies a false positive rate of
$\approx$15\%.

In addition, Figure~\ref{fig:age} shows that the candidate RSG-5 and
CH-2 members overlap with the $\delta$~Lyr cluster, and are therefore
roughly the same age.  To quantify this, we use the empirical method
introduced by \citet[][see their Section~6.3]{gagne_mutau_2020}.  This
idea of the method is to fit the pre-main-sequence loci of a set of
reference clusters, and to then model the locus of the target cluster
as a linear combination of these reference cluster loci.  For our
reference clusters, we used UCL, IC\,2602, and the Pleiades, from the
memberships reported by \citet{Damiani2019} and
\cite{CantatGaudin2018a} respectively.  We adopted ages of 16\,Myr for
UCL \citep{pecaut_star_2016}, 38\,Myr for IC\,2602\footnote{ Ages for
IC\,2602 vary from 40 to 46\,Myr based on lithium-depletion-boundary
(LDB) measurements \citep{dobbie_ic_2010,randich_gaiaeso_2018}, and
from 30 to 46\,Myr based on isochronal analyses
\citep{stauffer_rotational_1997,david_ages_2015,bossini_age_2019}.},
and 112\,Myr for the Pleiades \citep{dahm_2015}.  These assumptions
and the consequent processing steps taken to exclude field stars as
well as photometric and astrometric binaries were identical to those
described in \citet{bouma_kep1627_2022}.  The mean and standard
deviation of the resulting age posterior are $46^{+9}_{-7}$\,Myr for
RSG-5, and $36^{+10}_{-8}$\,Myr for CH-2.  For comparison, the this
procedure yields an age for the $\delta$~Lyr cluster of
$38^{+6}_{-5}$\,Myr.  The slightly older isochronal age of RSG-5 is
expected given that its locus is slightly bluer and less luminous in
the upper left panel of Figure~\ref{fig:age} relative to the
$\delta$~Lyr cluster.


\subsubsection{Stellar Rotation Periods}
\label{sec:rotation}

An independent way to assess the age of the candidate cluster members
is to measure their stellar rotation periods.  This approach can be
achieved using surveys such as TESS \citep{ricker_transiting_2015} and
ZTF \citep{bellm_zwicky_2019}, and it leverages a storied tradition of
rotation period measurement for benchmark open clusters \citep[see
{\it e.g.},][]{skumanich_time_1972,curtis_rup147_2020}.  The TESS data
in our case are especially useful, since they provide 3 to 5 lunar
months of photometry for all of our candidate CH-2 and RSG-5 members.

We selected stars suitable for gyrochronology by requiring $\bpmrpo
\geq 0.5$ and $G<16$.  The latter cut corresponds to $\bpmrpo \lesssim
2.6$, at the relevant distances.  These cuts gave 19 stars in CH-2 and
42 stars in RSG-5.  We extracted light curves from the TESS images for
these stars using the \texttt{unpopular} package
\citep{hattorio_2021_cpm}, and regressed them against systematics with
its causal pixel model.  {\bf For the ZTF light curves, we used the
default photometry?  Ran aperture photometry on the image cutouts?}.
We then measured candidate rotation periods using a Lomb-Scargle
periodogram, and visually inspected them following the methods
discussed in \citet{curtis_rup147_2020}.

The lower panels of Figure~\ref{fig:age} show the results.  In RSG-5,
36/42 stars have rotation periods faster than the Pleiades (86\%).
This numerator omits the two stars with periods $>12$\,days visible in
the lower-left panel of Figure~\ref{fig:age}.  The age interpretation
for these latter stars, particularly the $\approx$M2.5 dwarf, is not
obvious.  \citet{rebull_usco_2018} for instance have found numerous
M-dwarfs with 10-12 day rotation periods at ages of USco ($\sim
8$\,Myr), and some may still exist at ages of LCC ($\sim 16$\,Myr;
L.~Rebull in preparation).  Regardless, given that nearly no field
star outliers seem to be present on the RSG-5 CAMD, the fact that we
do not detect rotation periods for $\approx$14\% of stars should
perhaps be taken as an indication for the fraction of stars for
rotation periods might not be detectable, due to {\it e.g.}, pole-on
stars having lower amplitude starspot modulation.

%
% CH-2
% 5 stars with Prot>10 days (from the Prot versus color plot):
%                      DESIGNATION  TESS_Data   bp_rp_x     period
%25  Gaia EDR3 2129930258400157440          4  1.909693  15.309000
%24  Gaia EDR3 2129417404945830784          4  2.040988  10.981582
%4   Gaia EDR3 2107169165113864064          4  2.059904  10.580494
%9   Gaia EDR3 2134851775526125696          4  2.326998  12.227900
%1   Gaia EDR3 2127726081184588160          4  2.347244  11.543281
%
% CH-2
% Photometric field stars (from Glue -- one of these is omitted due to phot S/N cut, 
% presumably 2126008609657702784 )
%
%In [9]: df[['DESIGNATION', 'bp_rp', 'M_G']]
%Out[9]:
%                     DESIGNATION     bp_rp        M_G
%0  Gaia EDR3 2126529468937923456  2.518077   9.844733 --> NaN period, G=16.9
%1  Gaia EDR3 2126008609657702784  2.746735  11.409737 --> NaN, G=18.4
%2  Gaia EDR3 2133330051428541952  2.499458   9.459546 --> 15.7 d (ZTF), G=16.5
%3  Gaia EDR3 2133510921093964928  2.720177  10.923209 --> NaN, G=17.9
%4  Gaia EDR3 2139369183468814464  2.929506  11.170221 --> NaN, G=18.2
%5  Gaia EDR3 2129930258400157440  1.909693   8.200652 --> P=15.3 (ZTF), G=15.2

For CH-2, 13/19 stars have rotation periods that are obviously faster
than their counterparts in the Pleiades (68\%).  4 stars, not included
in the preceding numerator, are M-dwarfs with rotation periods between
10 and 12.5 days.  The age interpretation for these M-dwarfs is, as
just discussed, not obvious.  Regardless, the $\approx$15\% false
positive rate determined from the CAMD seems consistent with our
fraction of detected rotation periods, given that RSG-5 was also
missing rotation period detections for $\approx$15\% of its candidate
members, which all seemed photometrically consistent with being part
of a single pre-main-sequence locus.

It is challenging to convert these stellar rotation periods
to a precise age estimate, since on the pre-main-sequence
the stars are spinning up due to thermal contraction
rather than down due to magnetized braking.  Regardless, the rotation
period distributions of both CH-2 and RSG-5 seem consistent with other
30\,Myr to 50\,Myr clusters ({\it e.g.}, IC\,2602 and IC\,2391;
\citealt{douglas_stephanie_t_2021_5131306}).
They also seem consistent with the false positive rate estimates
determined from the color-absolute magnitude diagrams.


\section{The Stars}
\label{sec:stars}

\begin{deluxetable}{lccc}
\tabletypesize{\scriptsize}
\tablecaption{Selected system parameters of Kepler-1643, KOI-7368, and KOI-7913. \label{tab:sysparams}}
\tablenum{1}

\tablehead{
\colhead{Parameter} & \colhead{Value} & \colhead{68\% Confidence Interval} & \colhead{Comment}
}

\startdata
\hline
\multicolumn{4}{c}{\emph{Kepler-1643}} \\
\hline
{\it Stellar parameters:} & & & \\
  Gaia $G$~[mag]                             & $13.836$           & $\pm 0.003$                & A \\
  $T_{\rm eff}$~[K]                          & $4916$             & $\pm 110$                  & B \\
  %$\log g_\star$~[cgs]                      & $4.54$             & $\pm 0.10$                 & D \\
  $\log g_\star$~[cgs]                       & $4.502$            & $\pm 0.035$                & C \\
  $R_\star$~[R$_{\odot}$]                    & $0.855$            & $\pm 0.044$                & C \\
  $M_\star$~[M$_{\odot}$]                    & $0.845$            & $\pm 0.025$                & C \\
  $\rho_\star$~[g~cm$^{-3}$]                 & $1.910$            & $\pm 0.271$                & C \\
  $P_{\rm rot}$~[days]                       & $5.106$            & $\pm 0.044$                & D \\
  Li EW~[m\AA]                               & $126$              & $+8$, $-4$                 & E \\
{\it Transit parameters:} & & & \\
%  $t_0$~[BJD$_{\rm TDB}$]                   & $X$                & $X$                        & D \\
  $P$~[days]                                 & $X$                & $X$                        & D \\
  $R_{\rm p}/R_\star$                        & $0.X$              & $+0.X$, $-0.X$             & D \\
  $b$                                        & $X$                & $X$                        & D \\
%  $a/R_\star$                               & $X$                & $+0.X$, $-0.X$             & D \\
  $R_{\rm p}$~[R$_{\oplus}$]                 & $X$                & $\pm 0.X$                  & D \\
  $t_{14}$~[hours]                           & $X$                & $X$                        & D \\
\hline
\multicolumn{4}{c}{\emph{KOI-7368}} \\
\hline
{\it Stellar parameters:} & & & \\
  Gaia $G$~[mag]                             & $12.831$           & $\pm 0.004$                & A \\
  $T_{\rm eff}$~[K]                          & $5241$             & $\pm 50$                   & F \\
  $\log g_\star$~[cgs]                       & $4.499$            & $\pm 0.030$                & C \\
  $R_\star$~[R$_{\odot}$]                    & $0.876$            & $\pm 0.035$                & C \\
  $M_\star$~[M$_{\odot}$]                    & $0.879$            & $\pm 0.018$                & C \\
  $\rho_\star$~[g~cm$^{-3}$]                 & $1.840$            & $0.225$                    & C \\
  $P_{\rm rot}$~[days]                       & $2.606$            & $0.038$                    & D \\
  Li EW~[m\AA]                               & $X$                & $X$                        & B \\
{\it Transit parameters:} & & & \\
 % $t_0$~[BJD$_{\rm TDB}$]                    & $X$                & $X$                        & D \\
  $P$~[days]                                 & $X$                & $X$                        & D \\
  $R_{\rm p}/R_\star$                        & $0.X$              & $+0.X$, $-0.X$             & D \\
  $b$                                        & $X$                & $X$                        & D \\
%  $a/R_\star$                                & $X$                & $+0.X$, $-0.X$             & D \\
  $R_{\rm p}$~[R$_{\oplus}$]                 & $X$                & $\pm 0.X$                  & D \\
  $t_{14}$~[hours]                           & $X$                & $X$                        & D \\
\hline
\multicolumn{4}{c}{\emph{KOI-7913 A}} \\
\hline
{\it Stellar parameters:} & & & \\
  Gaia $G$~[mag]                             & $14.200$           & $\pm 0.003$                & A \\
  $T_{\rm eff}$~[K]                          & $4324$             & $\pm 70$                   & B \\
  $\log g_\star$~[cgs]                       & $4.523$            & $\pm 0.043$                & C \\
  $R_\star$~[R$_{\odot}$]                    & $0.790$            & $\pm 0.049$                & C \\
  $M_\star$~[M$_{\odot}$]                    & $0.760$            & $\pm 0.025$                & C \\
  $\rho_\star$~[g~cm$^{-3}$]                 & $2.172$            & $\pm 0.379$                & C \\
  $P_{\rm rot}$~[days]                       & $3.387$            & $0.016$                    & D \\
  Li EW~[m\AA]                               & $X$                & $X$                        & B \\
  $\Delta G_{\rm AB}$~[mag]                  & $0.51$             & $0.01$                     & F \\
  Apparent sep.~[au]                   		   & $959.4$            & $1.9$                      & F \\
{\it Transit parameters:} & & & \\
  %$t_0$~[BJD$_{\rm TDB}$]                    & $X$                & $X$                        & D \\
  $P$~[days]                                 & $X$                & $X$                        & D \\
  $R_{\rm p}/R_\star$                        & $0.X$              & $+0.X$, $-0.X$             & D \\
  $b$                                        & $X$                & $X$                        & D \\
%  $a/R_\star$                                & $X$                & $+0.X$, $-0.X$             & D \\
  $R_{\rm p}$~[R$_{\oplus}$]              & $X$                & $\pm 0.X$                  & D \\
  $t_{14}$~[hours]                           & $X$                & $X$                        & D \\
\enddata
\tablecomments{
  (A) \citet{gaia_collaboration_2021_edr3}.
  (B) HIRES SpecMatch-Emp \citep{yee_SM_2017}.
  (C) Cluster isochrone \citep{choi_mesa_2016, bressan_parsec_2012}.
  (D) Kepler light curve.  The full set of transit parameters is given in CITE APPENDIX TABLE.
  (E) HIRES; this work.
  (F) TRES SPC \citep{buchhave_hatp16b_class_2010,2021tsc2.confE.124B}.
  (G) Magnitude difference and physical distance between primary and secondary; from Gaia EDR3.
  (H) HIRES SpecMatch-Synth \citep{petigura_cksi_2017}.
}
%\vspace{-1cm}
\end{deluxetable}


Some of the most salient stellar parameters of the KOIs in Cep-Her can
be gleaned by inspecting Figure~\ref{fig:age}.  They span spectral
types of G8V (Kepler-1627) to K8V (KOI-7913 A).  The secondary in the
KOI-7913 system is marginally cooler than the primary.  Since a
Solar-mass star with solar metallicity arrives at the zero-age main
sequence at $t\approx40$ million years \citep{choi_mesa_2016}, these
stars are all in the late stages of their pre-main-sequence
contraction.  Their heliocentric distances are $\approx$260\,pc
(KOI-7913 and KOI-7368, in CH-2) and $\approx$340\,pc (Kepler-1643, in
RSG-5).

To derive spectroscopic parameters $(T_{\rm eff}, \log g_\star, [{\rm
Fe/H}])$ and to analyze spectroscopic youth proxies such as the
Li\,\textsc{I} 6708\,\AA\ doublet equivalent width and H$\alpha$
emission, we acquired spectra.  The system-by-system details follow.
The results are summarized in Table~\ref{tab:sysparams}.

%\citep[{\it e.g.},][]{soderblom_ages_2014}, we acquired an iodine-free
%spectrum from Keck/HIRES on the night of 2021 March 26 using the
%standard setup and reduction techniques of the California Planet
%Survey \citep{howard_cps_2010}.  Following the equivalent width
%measurement procedure described by \citet{bouma_2021_ngc2516}, we find
%${\rm EW}_{\rm Li} = 233^{+5}_{-7}$\,m\AA.   This value does not
%correct for the Fe\,\textsc{I} blend at 6707.44\AA.  Nonetheless,
%given the stellar effective temperature (Table~\ref{tab:starparams}),
%this measurement is in agreement with expectations for a
%$\approx40$\,Myr star ({\it e.g.}, as measured in IC~2602 by
%\citealt{randich_gaiaeso_2018}).  It is also larger than any lithium
%equivalent widths measured by \citet{berger_identifying_2018} in their
%analysis of 1{,}301 Kepler-star spectra.
%
%
%% rotation period from plot_rotation_period_windowslider.py
%Based on the spatial and kinematic association of \sn\ with the \cn,
%and the assumption that the planet formed shortly after the star, it
%seems likely that \sn\ is the same age as the cluster. There are two
%consistency checks on whether this is true: rotation and lithium.
%Based on the Kepler light curve, the rotation period is
%$2.642\pm0.042$\,days, where the quoted uncertainty is based on the
%scatter in rotation periods measured from each individual Kepler
%quarter.  This is consistent with comparable cluster members
%(Figure~\ref{fig:age}).


%\subsection{Kepler\,1627A}

\subsection{Kepler\,1643}

For Kepler-1643,
we acquired an iodine-free spectrum from Keck/HIRES
...
%FIMXE TODO TODO TODO WORDS LET'S GO!

spectra  (YYYY/MM/DD) and KOI-7913 (YYYY/MM/DD and
YYYY/MM/DD), where for the latter the two different epochs
corresponded to observations of the secondary and primary
respectively.  The acquisition and analysis followed the standard
reduction techniques of the California Planet Survey
\citep{howard_cps_2010}.  For KOI-7368, we acquired TRES spectra on
YYYY/MM/DD and YYYY/MM/DD.  CITE METHOD PAPER.  


\subsection{KOI-7368}
\subsection{KOI-7913}
Is a binary.
The latter system is 3$\farcs$501 
%FIXME FIXME FIXME TODO TODO TODO



\section{The Planets}
\label{sec:planets}

\begin{figure*}[tp]
	\begin{center}
		\leavevmode
		\subfloat{
			\includegraphics[width=0.85\textwidth]{f3e.pdf}
		}

		\vspace{-0.5cm}
		\subfloat{
			\includegraphics[width=0.41\textwidth]{f3a.pdf}
			\includegraphics[width=0.41\textwidth]{f3b.pdf}
		}

		\vspace{-1.45cm}	
		\subfloat{
			\includegraphics[width=0.41\textwidth]{f3c.pdf}
			\includegraphics[width=0.41\textwidth]{f3d.pdf}
		}
	\end{center}
	\vspace{-0.7cm}
	\caption{
		{\bf Raw and processed light curves for the objects of
    interest.} Top: raw.  Bottom: processed.
    The increased scatter during transit is likely due to starspot
    crossing events.  KOI-7913 is janky, but P=24 days.
		\label{fig:planets}
	}
\end{figure*}


\section{Discussion \& Conclusions}
\label{sec:disc_conc}

\subsection{Is CH-2 really a star cluster?}

RSG-5, and Kepler-1643's membership inside it, clearly meet the
typical expectations of a star claimed to be in an open cluster.
RSG-5 shows an obvious overdensity relative to the local field
population ({\it e.g.}, Figure~\ref{fig:XYZvtang}), and our membership
selection easily produces a clean pre-main-sequence locus in 
color-absolute magnitude space (Figure~\ref{fig:age}).
CH-2, and KOI-7913 and KOI-7368's membership inside it, do not meet
those expectations in as obvious a manner.  This is because this
association of stars is diffuse.

%really 80:9
To quantify the density discrepancy, we can compare the spatial and
velocity volumes searched to select candidate members of each cluster.
For RSG-5, we drew 141 candidate members from a $30\,{\rm
pc}\times30\,{\rm pc}\times30\,{\rm pc}$ spatial cube, given a $1 {\rm \mkms} \times 2 {\rm \mkms }$ rectangle in apparent galactic velocity.
For CH-2, our 37 candidate members came from a spatial cube of
dimension $50\,{\rm pc}\times40\,{\rm pc}\times30\,{\rm pc}$, and the
velocity rectangle of $2 {\rm \mkms} \times 4 {\rm \mkms}$.  If we
define the ``searched volume'' in units of ${\rm pc}^3 {\rm
(\mkms)}^2$, then the volume ratio of CH-2:RSG-5 is $\approx$9:1.  The
density (number of stars per unit searched volume) within RSG-5 relative to
CH-2 similarly comes out to 34 to 1.

So what makes a star cluster?  Historic answers to this question have
recently by reviewed by \citet{krumholz_star_2019}: some definitions
that have been offered include criteria such as being gravitationally
bound, and having a mass density that significantly exceeds the mean
in a cluster's galactic neighborhood.  We prefer a modified version of
the definition adopted by \citet{krumholz_star_2019}: for our purposes
a star cluster is a group of at least 12 stars that was physically
associated at its time of formation.  The somewhat arbitrary ``12'' is
set to distinguish clusters from high-order multiple star systems.  We
therefore explicitly include dissolved clusters and their tidal tails
in our concept of clusters.  We also explicitly exclude the idea that
a particular number of stars per unit spatial volume is required to
define a cluster.  The latter point acknowledges the fact that an
important factor in cluster identification is now also the number per
unit velocity volume, whether in 2-dimensional tangential velocity, or
when including the third radial component.  Perhaps
once stellar rotation periods and chemical abundances reach the same
level of ubiquity as stellar proper motions, they might enable further
refinement of our ability in cluster discovery.

From a data-driven perspective, how do we demonstrate that a star is
in a cluster, {\it i.e.}, that it is part of a group of stars that was
physically associated at its time of formation?
Back-integrating the orbits is one convincing approach, but it does
not always work {\bf (CITE)}.
The relatively minimal approach suggested by \citet{tofflemire_2021}
is intriguing: search for coeval, phase-space neighbors, measure their
ages, and determine if they share a common age.
This approach is more accurately be described as a method for
determining whether a star is currently associated with a set of coeval
stars, which is much easier to determine than what the association
looked like in the past.
From this standard, our analysis thus far of CH-2 has already
demonstrated the existence of such an assocation.

A crucial logical step in this method however is to ensure that the
(automated) search process for coeval phase-space neighbors in fact
produces neighbors at a rate different from what it would for field
stars.

% TODO
And so we ask: what is the density of field stars in the CH-2 region?
If we had applied Ronan's pre-main-sequence selection, + HDBScan, +
local neighbors... what would we have gotten?
Similarly... if you just applied Adam Kraus' Comove 
What is the likelihood we are fooling ourselves?



\begin{figure*}[!t]
	\begin{center}
		\leavevmode
		\includegraphics[width=0.9\textwidth]{f4.pdf}
	\end{center}
	\vspace{-0.7cm}
	\caption{
    %    namelist = ['Kepler-1627 A', 'KOI-7368', 'KOI-7913 A', 'KOI-7913 B', 'Kepler-1643']
    %    markers = ['P', 'v', 'X', 'X', 's']
		{\bf Radii, orbital periods, and ages of transiting exoplanets}.
		Planets younger than a gigayear with ${\rm \tau}/\sigma_{\tau} >
		3$ are emphasized, where $\tau$ is the age and $\sigma_{\tau}$ is
    its uncertainty. Kepler-1627 (+), KOI-7368 (down-triangle),
    KOI-7913 (X), Kepler-1643 (diamond).  The large sizes of
		the youngest transiting planets could be explained by their
		primordial atmospheres not yet having evaporated; direct
		measurements of the atmospheric outflows or planetary masses would
		help to confirm this expectation.  Selection effects may also be
		important.  Parameters are from the NASA Exoplanet Archive (2022
		Feb 27).
		\label{fig:rp_period_age}
	}
\end{figure*}




%%%%%%%%%%%%%%%%%%%%%%%%%%%%%%%%%%%%%%%%%%%%%%%%%%%%%%%%%%%%%%%%%%%%%%%%%%%%%%%


%\clearpage
\acknowledgements
\raggedbottom
%
L.G.B{.} acknowledges support from the TESS GI Program (NASA grants
80NSSC19K0386 and 80NSSC19K1728) and the Heising-Simons Foundation (51 Pegasi~b
Fellowship).
% programs G011103 and G022117, through 
%
%
%FIXME
Keck/NIRC2 imaging was acquired by program 2015A/N301N2L
(PI: A.~Kraus). % and 2019A/N069 (PI: E.~Petigura).
%
% ACKNOWLEDGE PFS / CAMPANAS.
%
This paper also includes data collected by the TESS mission, which are
publicly available from the Mikulski Archive for Space Telescopes
(MAST).
%
Funding for the TESS mission is provided by NASA's Science Mission
directorate.
%
We thank the TESS Architects (G.~Ricker, R.~Vanderspek, D.~Latham,
S.~Seager, J.~Jenkins) and the many TESS team members for their
efforts to make the mission a continued success.
%
%
%This study was based in part on observations at Cerro Tololo
%Inter-American Observatory at NSF's NOIRLab (NOIRLab Prop{.} ID
%2020A-0146; 2020B-0029 PI: Bouma), which is managed by the
%Association of Universities for Research in Astronomy (AURA) under a
%cooperative agreement with the National Science Foundation.
%
%
%Finally, this research has made use of the Keck Observatory Archive (KOA),
%which is operated by the W. M. Keck Observatory and the NASA Exoplanet
%Science Institute (NExScI), under contract with the National
%Aeronautics and Space Administration.
Finally, we also thank the Keck Observatory staff for their support of
HIRES and remote observing.  We recognize the importance that the
summit of Maunakea has within the indigenous Hawaiian community, and
are deeply grateful to have the opportunity to conduct observations
from this mountain.
%
% The Digitized Sky Survey was produced at the Space Telescope Science
% Institute under U.S. Government grant NAG W-2166.
% Figure~\ref{fig:scene} is based on photographic data obtained using
% the Oschin Schmidt Telescope on Palomar Mountain.
%

% %
% This research made use of the NASA Exoplanet Archive, which is
% operated by the California Institute of Technology, under contract
% with the National Aeronautics and Space Administration under the
% Exoplanet Exploration Program.
% %

% Resources supporting this work were provided by the NASA High-End
% Computing (HEC) Program through the NASA Advanced Supercomputing (NAS)
% Division at Ames Research Center for the production of the SPOC data
% products.
%

% A.J.\ and R.B.\ acknowledge support from project IC120009 ``Millennium
% Institute of Astrophysics (MAS)'' of the Millenium Science Initiative,
% Chilean Ministry of Economy. A.J.\ acknowledges additional support
% from FONDECYT project 1171208.  J.I.V\ acknowledges support from
% CONICYT-PFCHA/Doctorado Nacional-21191829.  R.B.\ acknowledges support
% from FONDECYT Post-doctoral Fellowship Project 3180246.
% %
% C.T.\ and C.B\ acknowledge support from Australian Research Council
% grants LE150100087, LE160100014, LE180100165, DP170103491 and
% DP190103688.
% %
% C.Z.\ is supported by a Dunlap Fellowship at the Dunlap Institute for
% Astronomy \& Astrophysics, funded through an endowment established by
% the Dunlap family and the University of Toronto.
% %
% D.D.\ acknowledges support through the TESS Guest Investigator Program
% Grant 80NSSC19K1727.
%
%
%
% %
% Based on observations obtained at the Gemini Observatory, which is
% operated by the Association of Universities for Research in Astronomy,
% Inc., under a cooperative agreement with the NSF on behalf of the
% Gemini partnership: the National Science Foundation (United States),
% National Research Council (Canada), CONICYT (Chile), Ministerio de
% Ciencia, Tecnolog\'{i}a e Innovaci\'{o}n Productiva (Argentina),
% Minist\'{e}rio da Ci\^{e}ncia, Tecnologia e Inova\c{c}\~{a}o (Brazil),
% and Korea Astronomy and Space Science Institute (Republic of Korea).
% %
% Observations in the paper made use of the High-Resolution Imaging
% instrument Zorro at Gemini-South. Zorro was funded by the NASA
% Exoplanet Exploration Program and built at the NASA Ames Research
% Center by Steve B. Howell, Nic Scott, Elliott P. Horch, and Emmett
% Quigley.
% %
% This research has made use of the VizieR catalogue access tool, CDS,
% Strasbourg, France. The original description of the VizieR service was
% published in A\&AS 143, 23.
% %
% This work has made use of data from the European Space Agency (ESA)
% mission {\it Gaia} (\url{https://www.cosmos.esa.int/gaia}), processed
% by the {\it Gaia} Data Processing and Analysis Consortium (DPAC,
% \url{https://www.cosmos.esa.int/web/gaia/dpac/consortium}). Funding
% for the DPAC has been provided by national institutions, in particular
% the institutions participating in the {\it Gaia} Multilateral
% Agreement.
%
% (Some of) The data presented herein were obtained at the W. M. Keck
% Observatory, which is operated as a scientific partnership among the
% California Institute of Technology, the University of California and
% the National Aeronautics and Space Administration. The Observatory was
% made possible by the generous financial support of the W. M. Keck
% Foundation.
% The authors wish to recognize and acknowledge the very significant
% cultural role and reverence that the summit of Maunakea has always had
% within the indigenous Hawaiian community.  We are most fortunate to
% have the opportunity to conduct observations from this mountain.
%
% \newline
%

\software{
  %\texttt{arviz} \citep{arviz_2019},
  %\texttt{altaipony} \citep{ilin_flares_2021},
  \texttt{astrobase} \citep{bhatti_astrobase_2018},
  %\texttt{astroplan} \citep{astroplan2018},
	%\texttt{AstroImageJ} \citep{collins_astroimagej_2017},
  \texttt{astropy} \citep{astropy_2018},
  \texttt{astroquery} \citep{astroquery_2018},
  %\texttt{BATMAN} \citep{kreidberg_batman_2015},
  %\texttt{ceres} \citep{brahm_2017_ceres},
  %\texttt{cdips-pipeline} \citep{bhatti_cdips-pipeline_2019},
  \texttt{corner} \citep{corner_2016},
  %\texttt{emcee} \citep{foreman-mackey_emcee_2013},
  \texttt{exoplanet} \citep{exoplanet:exoplanet}, and its
  dependencies \citep{exoplanet:agol20, exoplanet:kipping13, exoplanet:luger18,
   	exoplanet:theano},
	%\texttt{gala} \citep{gala,PriceWhelan_2017_gala_zenodo},
	%\texttt{IDL Astronomy User's Library} \citep{landsman_1995},
  %\texttt{IPython} \citep{perez_2007},
	%\texttt{isochrones} \citep{morton_2015_isochrones},
	%\texttt{lightkurve} \citep{lightkurve_2018},
  %\texttt{matplotlib} \citep{hunter_matplotlib_2007}, 
  %\texttt{MESA} \citep{paxton_modules_2011,paxton_modules_2013,paxton_modules_2015}
  %\texttt{numpy} \citep{walt_numpy_2011}, 
  %\texttt{pandas} \citep{mckinney-proc-scipy-2010},
  %\texttt{pyGAM} \citep{serven_pygam_2018_1476122},
  \texttt{PyMC3} \citep{salvatier_2016_PyMC3},
  %\texttt{radvel} \citep{fulton_radvel_2018},
  %\texttt{scikit-learn} \citep{scikit-learn},
  \texttt{scipy} \citep{jones_scipy_2001},
  %\texttt{TESS-point}  \citep{burke_2020},
  %\texttt{tesscut} \citep{brasseur_astrocut_2019},
	%\texttt{VESPA} \citep{morton_efficient_2012,vespa_2015},
  %\texttt{webplotdigitzer} \citep{rohatgi_2019},
  %\texttt{wotan} \citep{hippke_wotan_2019}.
}
\ 

\facilities{
 	{\it Astrometry}:
 	Gaia \citep{gaia_collaboration_gaia_2018,gaia_collaboration_2021_edr3}.
 	{\it Imaging}:
    Second Generation Digitized Sky Survey. %,
    %SOAR~(HRCam; \citealt{tokovinin_ten_2018}).
 	Keck:II~(NIRC2; \url{www2.keck.hawaii.edu/inst/nirc2}).
 	%Gemini:South~(Zorro; \citealt{scott_nessi_2018}.
 	%Gemini:North~(`Alopeke; \citealt{scott_nessi_2018,scott_twin_2021}.
 	{\it Spectroscopy}:
	%CTIO1.5$\,$m~(CHIRON; \citealt{tokovinin_chironfiber_2013}),
  %PFS ({\bf CITE}),
	Tillinghast:1.5m~(TRES; \citealt{furesz_tres_2008}).
  %MPG2.2$\,$m~(FEROS; \citealt{kaufer_commissioning_1999}),
	%AAT~(Veloce; \citealt{gilbert_veloce_2018}).
	%AAT~(HERMES; \citealt{lewis_2002_hermers_2df,sheinis_2015_hermes}),
 	Keck:I~(HIRES; \citealt{vogt_hires_1994}).
 	%VLT:Kueyen~(FLAMES; \citealt{pasquini_2002}).
% 	Euler1.2m~(CORALIE),
% 	ESO:3.6m~(HARPS; \citealt{mayor_setting_2003}).
 	{\it Photometry}:
%	  ASTEP:0.40$\,$m (ASTEP400),
% 	CTIO:1.0m (Y4KCam),
% 	Danish 1.54m Telescope,
%	  El Sauce:0.356$\,$m,
% 	Elizabeth 1.0m at SAAO,
% 	Euler1.2m (EulerCam),
	  Kepler \citep{borucki_kepler_2010},
% 	Magellan:Baade (MagIC),
% 	Max Planck:2.2m	(GROND; \citealt{greiner_grond7-channel_2008})
%   MuSCAT3 \citep{Narita_2020},
% 	NTT,
% 	SOAR (SOI),
 	  TESS \citep{ricker_transiting_2015}.
% 	TRAPPIST \citep{jehin_trappist_2011},
% 	VLT:Antu (FORS2).
}

% \begin{table*}
\scriptsize
\setlength{\tabcolsep}{2pt}
\centering
\caption{Literature and Measured Properties for Kepler$\,$1627 A}
\label{tab:starparams}
%\tablenum{2}
\begin{tabular}{llcc}
  \hline
  \hline
Other identifiers\dotfill & \\
\multicolumn{3}{c}{TIC 120105470} \\
\multicolumn{3}{c}{GAIADR2 2103737241426734336} \\
\multicolumn{3}{c}{GAIAEDR3 2103737241426734336} \\
\hline
\hline
Parameter & Description & Value & Source\\
\hline 
$\alpha_{J2015.5}$\dotfill	&Right Ascension (hh:mm:ss)\dotfill & 18:56:13.6 & 1	\\
$\delta_{J2015.5}$\dotfill	&Declination (dd:mm:ss)\dotfill & +41:34:36.22 & 1	\\
%$l_{J2015.5}$\dotfill	&Galactic Longitude (deg)\dotfill & 288.2644 & 1	\\
%$b_{J2015.5}$\dotfill	&Galactic Latitude (deg)\dotfill & -5.7950 & 1	\\
%\\
%$NUV$\dotfill           & GALEX $NUV$ mag.\dotfill & 13.804 $\pm$ 0.004 & 2 \\
%$FUV$\dotfill           & GALEX $FUV$ mag.\dotfill & 18.466 $\pm$ 0.056 & 2 \\
\\
%B\dotfill			&Johnson B mag.\dotfill & 11.119 $\pm$ 0.107		& 2	\\
V\dotfill			&Johnson V mag.\dotfill & 13.11 $\pm$ 0.08		& 2	\\
%$B$\tablenote{The uncertainties of the photometry have a systematic error floor applied. Even still, the global fit requires a significant scaling of the uncertainties quoted here to be consistent with our model, suggesting they are still significantly underestimated for one or more of the broad band magnitudes}\dotfill		& APASS Johnson $B$ mag.\dotfill	& 13.001 $\pm$	0.02& 2	\\
%$V$\dotfill		& APASS Johnson $V$ mag.\dotfill	& 11.808 $\pm$	0.02& 2	\\
%\\
${\rm G}$\dotfill     & Gaia $G$ mag.\dotfill     & 13.049$\pm$0.02 & 1\\
%${\rm Bp}$\dotfill     & Gaia $Bp$ mag.\dotfill     & 10.695 $\pm$0.020 & 1\\
%${\rm Rp}$\dotfill     & Gaia $Rp$ mag.\dotfill     & 9.887$\pm$0.020 & 1\\
${\rm T}$\dotfill     & TESS $T$ mag.\dotfill     & 12.53$\pm$0.02 & 2\\
%$u'$\dotfill        & Sloan $u'$ mag.\dotfill & 14.706 $\pm$ 0.006& 3\\
%$g'$\dotfill		& APASS Sloan $g'$ mag.\dotfill	& 12.407 $\pm$ 0.02	&  2	\\
%$r'$\dotfill		& APASS Sloan $r'$ mag.\dotfill	& 11.311 $\pm$ 0.02	&  2	\\
%$i'$\dotfill		& APASS Sloan $i'$ mag.\dotfill	& 10.927 $\pm$ 0.04 &  2	\\
%\\
J\dotfill			& 2MASS J mag.\dotfill & 11.69  $\pm$ 0.02	& 3	\\
H\dotfill			& 2MASS H mag.\dotfill & 11.30 $\pm$ 0.02	    &  3	\\
K$_{\rm S}$\dotfill			& 2MASS ${\rm K_S}$ mag.\dotfill & 11.19 $\pm$ 0.02 &  3	\\
%\\
%W1\dotfill		& WISE1 mag.\dotfill & 8.901 $\pm$ 0.023 & 4	\\
%W2\dotfill		& WISE2 mag.\dotfill & 8.875 $\pm$ 0.021 &  4 \\
%W3\dotfill		& WISE3 mag.\dotfill &  8.875 $\pm$ 0.020& 4	\\
%W4\dotfill		& WISE4 mag.\dotfill & 8.936 $\pm$ N/A &  4	\\
\\
$\pi$\dotfill & Gaia EDR3 parallax (mas) \dotfill & 3.009 $\pm$ 0.032 &  1 \\
$d$\dotfill & Distance (pc)\dotfill & $329.5 \pm 3.5$ & 1, 4 \\
$\mu_{\alpha}$\dotfill		& Gaia EDR3 proper motion\dotfill & 1.716 $\pm$ 0.034 & 1 \\
                    & \hspace{3pt} in RA (mas yr$^{-1}$)	&  \\
$\mu_{\delta}$\dotfill		& Gaia EDR3 proper motion\dotfill 	&  -1.315 $\pm$ 0.034 &  1 \\
                    & \hspace{3pt} in DEC (mas yr$^{-1}$) &  \\
RUWE\dotfill		& Gaia EDR3 renormalized\dotfill 	&  2.899 &  1 \\
                    & \hspace{3pt} unit weight error &  \\
%
\\
RV\dotfill & Systemic radial \hspace{9pt}\dotfill  & $-14.3 \pm 1.0$ & 5 \\
                    & \hspace{3pt} velocity (\kms)  & \\
Spec. Type\dotfill & Spectral Type\dotfill & 	G8V & 5 \\
$v\sin{i_\star}$\dotfill &  Rotational velocity (\kms) \hspace{9pt}\dotfill &  18.9 $\pm$ 1.0 & 5 \\
Li EW\dotfill & 6708\AA\ Equiv{.} Width (m\AA) \dotfill & $233^{+5}_{-7}$  & 5 \\
%$v_{\rm mac}$\dotfill &  Macroturbulence velocity (\kms) \hspace{9pt}\dotfill &  8.4 $\pm$ 2.9 & 5 \\
%${\rm [Fe/H]}$\dotfill &   Metallicity$^\dagger$ \hspace{9pt}\dotfill & -0.02 $\pm$ 0.09 & 5 \\
%$T_{\rm eff}$\dotfill &  Effective Temperature (K) \hspace{9pt}\dotfill & 5777 $\pm$ 110 &  5  \\
%$\log{g_{\star}}$\dotfill &  Surface Gravity (cgs)\hspace{9pt}\dotfill &  4.6 $\pm$ 0.1  &  5 \\
$T_{\rm eff}$\dotfill &  Effective Temperature (K) \hspace{9pt}\dotfill & 5505 $\pm$ 39 &  6  \\
$\log{g_{\star}}$\dotfill &  Surface Gravity (cgs)\hspace{9pt}\dotfill &  4.53 $\pm$ 0.02  &  6 \\
%
% $E(B-V)$\dotfill & Reddening (mag)\dotfill & $0.06 \pm 0.02$ & 9 \\
%
%
$R_\star$\dotfill & Stellar radius ($R_\odot$)\dotfill & 0.881$\pm$0.018 & 6 \\
$M_\star$\dotfill & Stellar mass ($R_\odot$)\dotfill & 0.953$\pm$0.019 & 6 \\
%$F_{\rm bol}$\dotfill & Stellar bolometric flux (cgs)\dotfill & (1.967$\pm$0.046)$\times10^{-9}$ & 9 \\
%
%
$A_{\rm V}$\dotfill & Interstellar reddening (mag)\dotfill & 0.2$\pm$0.1 & 6 \\
${\rm [Fe/H]}$\dotfill &   Metallicity \hspace{9pt}\dotfill & 0.1 $\pm$ 0.1 & 6 \\
%
$P_{\rm rot}$\dotfill & Rotation period (d)\dotfill & $2.642\pm 0.042$  & 7 \\
Age & Adopted stellar age (Myr)\dotfill & $38^{+6}_{-5}$  &  8 \\
% $U^{*}$\dotfill & Space Velocity (\kms)\dotfill & $26.24 \pm 0.46$  & \S\ref{sec:uvw} \\
% $V$\dotfill       & Space Velocity (\kms)\dotfill & $-71.52 \pm 1.68$ & \S\ref{sec:uvw} \\
% $W$\dotfill       & Space Velocity (\kms)\dotfill & $ -1.31 \pm 0.27$ & \S\ref{sec:uvw} \\
\hline
\end{tabular}
\begin{flushleft}
 \footnotesize{ \textsc{NOTE}---
Provenances are:
$^1$\citet{gaia_collaboration_gaia_2018},
$^2$\citet{stassun_TIC8_2019},
$^3$\citet{skrutskie_tmass_2006},
$^4$\citet{Lindegren_2021_offset},
$^5$HIRES spectra and \citet{yee_SM_2017},
$^6$Cluster isochrone (MIST+PARSEC),
%$^7$FEROS spectra,
$^7$\citet{capitanio_threedimensional_2017} and \citet{lallement_threedimensional_2018},
$^8$Kepler light curve,
$^9$Pre-main-sequence CMD, with LDB age for IC~2602 being most
important (Section~\ref{sec:cmd}).
%$*$ $U$ is in the direction of the Galactic center. \\
%$^{10}$Method~1 (photometric SED fit, Section~\ref{subsec:starparams}).}
}
\end{flushleft}
\vspace{-0.5cm}
\end{table*}

% % Table of best fit parameters
%\startlongtable
\begin{deluxetable*}{lllrrrrrrr}
%
  \tablecaption{ Priors and Posteriors for Model Fitted to the Long
  Cadence Kepler 1627Ab Light Curve.}
\label{tab:posterior}
%
\tabletypesize{\scriptsize}
%\tabletypesize{\small}
%
%\tablenum{2}
%
\tablehead{
  \colhead{Param.} & 
  \colhead{Unit} &
  \colhead{Prior} & 
  \colhead{Median} & 
  \colhead{Mean} & 
  \colhead{Std{.} Dev.} &
  \colhead{3\%} &
  \colhead{97\%} &
  \colhead{ESS} &
  \colhead{$\hat{R}-1$}
}

%/Users/luke/Dropbox/proj/rudolf/results/run_RotGPtransit/Kepler_1627_RotGPtransit_posteriortable.tex
\startdata
{\it Sampled} & & & & & & & & & \\
\hline
$P$ & d & $\mathcal{N}(7.20281; 0.01000)$ & 7.2028038 & 7.2028038 & 0.0000073 & 7.2027895 & 7.2028168 & 7464 & 3.9e-04 \\
$t_0^{(1)}$ & d & $\mathcal{N}(120.79053; 0.02000)$ & 120.7904317 & 120.7904254 & 0.0009570 & 120.7886377 & 120.7921911 & 3880 & 2.0e-03 \\
$\log \delta$ & -- & $\mathcal{N}(-6.3200; 2.0000)$ & -6.3430 & -6.3434 & 0.0354 & -6.4094 & -6.2767 & 6457 & 3.0e-04 \\
$b^{(2)}$ & -- & $\mathcal{U}(0.000; 1.000)$ & 0.4669 & 0.4442 & 0.2025 & 0.0662 & 0.8133 & 1154 & 1.6e-03 \\
$u_1$ & -- & \citet{exoplanet:kipping13} & 0.271 & 0.294 & 0.190 & 0.000 & 0.628 & 3604 & 1.5e-03 \\
$u_2$ & -- & \citet{exoplanet:kipping13} & 0.414 & 0.377 & 0.326 & -0.240 & 0.902 & 3209 & 1.4e-03 \\
$R_\star$ & $R_\odot$ & $\mathcal{N}(0.881; 0.018)$ & 0.881 & 0.881 & 0.018 & 0.847 & 0.915 & 8977 & 3.1e-04 \\
$\log g$ & cgs & $\mathcal{N}(4.530; 0.050)$ & 4.532 & 4.533 & 0.051 & 4.435 & 4.627 & 6844 & 1.6e-03 \\
$\langle f \rangle$ & -- & $\mathcal{N}(0.000; 0.100)$ & -0.0003 & -0.0003 & 0.0001 & -0.0005 & -0.0000 & 8328 & 1.1e-03 \\
$e^{(3)}$ & -- & \citet{vaneylen19} & 0.154 & 0.186 & 0.152 & 0.000 & 0.459 & 1867 & 2.0e-03 \\
$\omega$ & rad & $\mathcal{U}(0.000; 6.283)$ & 0.055 & 0.029 & 1.845 & -3.139 & 2.850 & 3557 & 8.6e-05 \\
$\log \sigma_f$ & -- & $\mathcal{N}(\log\langle \sigma_f \rangle; 2.000)$ & -8.035 & -8.035 & 0.008 & -8.049 & -8.021 & 9590 & 3.9e-04 \\
$\sigma_{\mathrm{rot}}$ & d$^{-1}$ & $\mathrm{InvGamma}(1.000; 5.000)$ & 0.070 & 0.070 & 0.001 & 0.068 & 0.072 & 9419 & 1.4e-03 \\
$\log P_{\mathrm{rot}}$ & $\log (\mathrm{d})$ & $\mathcal{N}(0.958; 0.020)$ & 0.978 & 0.978 & 0.001 & 0.975 & 0.980 & 8320 & 2.2e-04 \\
$\log Q_0$ & -- & $\mathcal{N}(0.000; 2.000)$ & -0.327 & -0.326 & 0.043 & -0.407 & -0.246 & 9659 & 2.7e-04 \\
$\log \mathrm{d}Q$ & -- & $\mathcal{N}(0.000; 2.000)$ & 7.697 & 7.698 & 0.103 & 7.511 & 7.899 & 5824 & 3.7e-04 \\
$f$ & -- & $\mathcal{U}(0.010; 1.000)$ & 0.01006 & 0.01009 & 0.00009 & 0.01000 & 0.01025 & 4645 & 4.0e-04 \\
{\it Derived} & & & & & & & & & \\
\hline
$\delta$ & -- & -- & 0.001759 & 0.001759 & 0.000062 & 0.001641 & 0.001875 & 6457 & 3.0e-04 \\
$R_{\rm p}/R_\star$ & -- & -- & 0.039 & 0.039 & 0.001 & 0.037 & 0.042 & 1811 & 1.1e-03 \\
$\rho_\star$ & g$\ $cm$^{-3}$ & -- & 1.990 & 2.004 & 0.240 & 1.570 & 2.461 & 6905 & 2.1e-03 \\
$R_{\rm p}^{(4)}$ & $R_{\mathrm{Jup}}$ & -- & 0.337 & 0.338 & 0.014 & 0.314 & 0.367 & 2311 & 1.0e-03 \\
$R_{\rm p}^{(4)}$ & $R_{\mathrm{Earth}}$ & -- & 3.777 & 3.789 & 0.157 & 3.52 & 4.114 & 2311 & 1.0e-03 \\
$a/R_\star$ & -- & -- & 17.606 & 17.619 & 0.702 & 16.277 & 18.906 & 6905 & 2.1e-03 \\
$\cos i$ & -- & -- & 0.027 & 0.025 & 0.010 & 0.004 & 0.040 & 1312 & 1.2e-03 \\
$T_{14}$ & hr & -- & 2.841 & 2.843 & 0.060 & 2.734 & 2.958 & 3199 & 3.6e-04 \\
$T_{13}$ & hr & -- & 2.555 & 2.539 & 0.094 & 2.360 & 2.692 & 1960 & 1.4e-03 \\
\enddata
%
\tablecomments{
  ESS refers to the number of effective samples.
  $\hat{R}$ is the Gelman-Rubin convergence diagnostic.
  Logarithms in this table are base-$e$.
  $\mathcal{U}$ denotes a uniform distribution,
  and $\mathcal{N}$ a normal distribution.
  (1) The ephemeris is in units of BJDTDB - 2454833.
  (2) Although $\mathcal{U}(0,1+R_{\rm p}/R_\star)$ is formally
  correct, for this model we assumed a non-grazing transit to enable
  sampling in $\log \delta$.
  (3) The eccentricity vectors are sampled in the $(e\cos\omega,
  e\sin\omega)$ plane.
  (4) The true planet size is a factor of $((F_1+F_2)/F_1)^{1/2}$
  larger than that from the fit because of dilution from Kepler
  1627B, where $F_1$ is the flux from the primary, and $F_2$ is that
  from the secondary; the mean and standard deviation of $R_{\rm
  p}=3.817\pm0.158\,R_{\oplus}$ quoted in the text includes this correction,
  assuming $(F_1+F_2)/F_1\approx 1.015$.
}
\vspace{-0.3cm}
\end{deluxetable*}

% %% \begin{deluxetable}{} command tell LaTeX how many columns
%% there are and how to align them.
%\startlongtable
\begin{deluxetable*}{lll}
    
%% Keep a portrait orientation

%% Over-ride the default font size
%% Use Default (12pt)
\tabletypesize{\scriptsize}
%\tabletypesize{\small}

%% Use \tablewidth{?pt} to over-ride the default table width.
%% If you are unhappy with the default look at the end of the
%% *.log file to see what the default was set at before adjusting
%% this value.

%% This is the title of the table.
\tablecaption{Young, Age-dated, and Age-dateable Stars Within the
  Nearest Few Kiloparsecs.}
\label{tab:v06}

%% This command over-rides LaTeX's natural table count
%% and replaces it with this number.  LaTeX will increment 
%% all other tables after this table based on this number
%\tablenum{3}

%% The \tablehead gives provides the column headers.  It
%% is currently set up so that the column labels are on the
%% top line and the units surrounded by ()s are in the 
%% bottom line.  You may add more header information by writing
%% another line between these lines. For each column that requries
%% extra information be sure to include a \colhead{text} command
%% and remember to end any extra lines with \\ and include the 
%% correct number of &s.
\tablehead{
  \colhead{Parameter} &
  \colhead{Example Value} &
  \colhead{Description}
}

%% All data must appear between the \startdata and \enddata commands
%
% paste from
% /Users/luke/Dropbox/proj/rudolf/results/tables/v06_main_tableheader.tex
% via drivers/write_v06_main_tableheader.py
\startdata
          \texttt{source\_id} &                                          1709456705329541504 &                                              Gaia DR2 source identifier. \\
                  \texttt{ra} &                                                      247.826 &                                          Gaia DR2 right ascension [deg]. \\
                 \texttt{dec} &                                                       79.789 &                                              Gaia DR2 declination [deg]. \\
            \texttt{parallax} &                                                       35.345 &                                                 Gaia DR2 parallax [mas]. \\
     \texttt{parallax\_error} &                                                        0.028 &                                     Gaia DR2 parallax uncertainty [mas]. \\
                \texttt{pmra} &                                                       94.884 &      Gaia DR2 proper motion $\mu_\alpha \cos \delta$ [mas$\,$yr$^{-1}$]. \\
               \texttt{pmdec} &                                                      -86.971 &                  Gaia DR2 proper motion $\mu_\delta$ [mas$\,$yr$^{-1}$]. \\
  \texttt{phot\_g\_mean\_mag} &                                                         6.85 &                                                  Gaia DR2 $G$ magnitude. \\
 \texttt{phot\_bp\_mean\_mag} &                                                        6.409 &                                      Gaia DR2 $G_\mathrm{BP}$ magnitude. \\
 \texttt{phot\_rp\_mean\_mag} &                                                        7.189 &                                      Gaia DR2 $G_\mathrm{RP}$ magnitude. \\
             \texttt{cluster} &                  NASAExoArchive\_ps\_20210506,Uma,IR\_excess &                                   Comma-separated cluster or group name. \\
                 \texttt{age} &                                                 9.48,nan,nan &  Comma-separated logarithm (base-10) of reported$^{\rm a}$ age in years. \\
           \texttt{mean\_age} &                                                         9.48 &                           Mean (ignoring NaNs) of $\texttt{age}$ column. \\
       \texttt{reference\_id} &       NASAExoArchive\_ps\_20210506,Ujjwal2020,CottenSong2016 &                           Comma-separted provenance of group membership. \\
  \texttt{reference\_bibcode} &  2013PASP..125..989A,2020AJ....159..166U,2016ApJS..225...15C &                   ADS bibcode corresponding to $\texttt{reference\_id}$. \\
\enddata

%% Include any \tablenotetext{key}{text}, \tablerefs{ref list},
%% or \tablecomments{text} between the \enddata and 
%% \end{deluxetable} commands

%% General table comment marker
\tablecomments{
Table~\ref{tab:v06} is published in its entirety in a machine-readable
format.   This table is a concatenation of the studies listed in
Table~\ref{tab:metadata}.  One entry is shown for guidance regarding
form and content.  In this particular example, the star has a cold
Jupiter on a 16 year orbit, HD 150706b \citep{2012AA...545A..55B}.  An
infrared excess has been reported \citep{CottenSong2016}, and the star
was identified by \citet{Ujjwal2020} as a candidate UMa moving group
member ($\approx 400\,{\rm Myr}$; \citealt{mann_tess_2020}).  The
star's RV activity and TESS rotation period corroborate its youth.
}
\vspace{-0.5cm}
\end{deluxetable*}

% %% \begin{deluxetable}{} command tell LaTeX how many columns
%% there are and how to align them.
%\startlongtable
\begin{deluxetable*}{lccc}
    
%% Keep a portrait orientation

%% Over-ride the default font size
%% Use Default (12pt)
\tabletypesize{\scriptsize}
%\tabletypesize{\small}
%\tabletypesize{\normal}

%% Use \tablewidth{?pt} to over-ride the default table width.
%% If you are unhappy with the default look at the end of the
%% *.log file to see what the default was set at before adjusting
%% this value.

%% This is the title of the table.
\tablecaption{Provenances of Young and Age-dateable Stars.}
\label{tab:metadata}

%% This command over-rides LaTeX's natural table count
%% and replaces it with this number.  LaTeX will increment 
%% all other tables after this table based on this number
%\tablenum{3}

%% The \tablehead gives provides the column headers.  It
%% is currently set up so that the column labels are on the
%% top line and the units surrounded by ()s are in the 
%% bottom line.  You may add more header information by writing
%% another line between these lines. For each column that requries
%% extra information be sure to include a \colhead{text} command
%% and remember to end any extra lines with \\ and include the 
%% correct number of &s.
\tablehead{
  \colhead{Reference} &
  \colhead{$N_{\rm Gaia}$} &
  \colhead{$N_{\rm Age}$} &
  \colhead{$N_{G_{\rm RP}<16}$}
}

%% All data must appear between the \startdata and \enddata commands
%
% paste from
% /Users/luke/Dropbox/proj/rudolf/results/tables/metadata_table_data.tex
% via drivers/write_metadata_table.py
\startdata
                           \citet{Kounkel2020}  &             987376 &            987376 &                775363 \\
                     \citet{CantatGaudin2020a}  &             433669 &            412671 &                269566 \\
                     \citet{CantatGaudin2018a}  &             399654 &            381837 &                246067 \\
                      \citet{KounkelCovey2019}  &             288370 &            288370 &                229506 \\
                     \citet{CantatGaudin2020b}  &             233369 &            227370 &                183974 \\
                           \citet{Zari2018} UMS &              86102 &                 0 &                 86102 \\
                  \citet{SIMBAD} $\texttt{Y*?}$ &              61432 &                 0 &                 45076 \\
                           \citet{Zari2018} PMS &              43719 &                 0 &                 38435 \\
\citet{GaiaCollaboration2018} $d>250\,{\rm pc}$ &              35506 &             31182 &                 18830 \\
                      \citet{CastroGinard2020}  &              33635 &             24834 &                 31662 \\
                              \citet{Kerr2021}  &              30518 &             25324 &                 27307 \\
                  \citet{SIMBAD} $\texttt{Y*O}$ &              28406 &                 0 &                 16205 \\
                        \citet{VillaVelez2018}  &              14459 &             14459 &                 13866 \\
                     \citet{CantatGaudin2019a}  &              11843 &             11843 &                  9246 \\
                        \citet{Damiani2019} PMS &              10839 &             10839 &                  9901 \\
                                \citet{Oh2017}  &              10379 &                 0 &                 10370 \\
                          \citet{Meingast2021}  &               7925 &              7925 &                  5878 \\
                 \citet{SIMBAD} $\texttt{pMS*}$ &               5901 &                 0 &                  3006 \\
\citet{GaiaCollaboration2018} $d<250\,{\rm pc}$ &               5378 &               817 &                  3968 \\
                           \citet{Kounkel2018}  &               5207 &              3740 &                  5207 \\
                        \citet{Ratzenbock2020}  &               4269 &              4269 &                  2662 \\
                  \citet{SIMBAD} $\texttt{TT*}$ &               4022 &                 0 &                  3344 \\
                        \citet{Damiani2019} UMS &               3598 &              3598 &                  3598 \\
                           \citet{Rizzuto2017}  &               3294 &              3294 &                  2757 \\
            \citet{NASAExoArchive_ps_20210506}  &               3107 &               868 &                  3098 \\
                              \citet{Tian2020}  &               1989 &              1989 &                  1394 \\
                           \citet{Goldman2018}  &               1844 &              1844 &                  1783 \\
                        \citet{CottenSong2016}  &               1695 &                 0 &                  1693 \\
                            \citet{Gagne2018a}  &               1429 &                 0 &                  1389 \\
             \citet{RoserSchilbach2020} Psc-Eri &               1387 &              1387 &                  1107 \\
            \citet{RoserSchilbach2020} Pleiades &               1245 &              1245 &                  1019 \\
                  \citet{SIMBAD} $\texttt{TT?}$ &               1198 &                 0 &                   853 \\
                            \citet{Gagne2018c}  &                914 &                 0 &                   913 \\
                          \citet{Pavlidou2021}  &                913 &               913 &                   504 \\
                            \citet{Gagne2018b}  &                692 &                 0 &                   692 \\
                            \citet{Ujjwal2020}  &                563 &                 0 &                   563 \\
                             \citet{Gagne2020}  &                566 &               566 &                   351 \\
                      \citet{EsplinLuhman2019}  &                377 &               443 &                   296 \\
                     \citet{Roccatagliata2020}  &                283 &               283 &                   232 \\
                          \citet{Meingast2019}  &                238 &               238 &                   238 \\
                 \citet{Furnkranz2019} Coma-Ber &                214 &               214 &                   213 \\
           \citet{Furnkranz2019} Neighbor Group &                177 &               177 &                   167 \\
                             \citet{Kraus2014}  &                145 &               145 &                   145 \\
\enddata

%% Include any \tablenotetext{key}{text}, \tablerefs{ref list},
%% or \tablecomments{text} between the \enddata and 
%% \end{deluxetable} commands

%% General table comment marker
\tablecomments{
Table~\ref{tab:metadata} describes the provenances for the young and
age-dateable stars in Table~\ref{tab:v06}.  $N_{\rm Gaia}$: number of
Gaia stars we parsed from the literature source.  $N_{\rm Age}$:
number of stars in the literature source with ages reported.
$N_{G_{\rm RP}<16}$: number of Gaia stars we parsed from the
literature source with either $G_{\rm RP}<16$, or a parallax S/N
exceeding 5 and a distance closer than 100\,pc.  The latter criterion
included a few hundred white dwarfs that would have otherwise been
neglected.  Some studies are listed multiple times if they contain
multiple tables.  \citet{SIMBAD} refers to the \texttt{SIMBAD}
database.
}
\vspace{-0.5cm}
\end{deluxetable*}


\clearpage
\bibliographystyle{yahapj}                            
\bibliography{bibliography} 

%\appendix
%\section{Young, Age-Dated, and Age-Dateable Star Compilation}
%\label{app:targetlist}


%\listofchanges
%\allauthors
\end{document}
