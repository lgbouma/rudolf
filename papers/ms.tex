%\documentclass[12pt,modern,twocolumn,tighten]{aastex63}
\documentclass[12pt,modern,tighten]{aastex63}
%\documentclass[12pt,modern,twocolumn,tighten,linenumbers,trackchanges]{aastex63}
%\documentclass[12pt,twocolumn,tighten,linenumbers]{aastex63}
%\documentclass[12pt,twocolumn,tighten,trackchanges]{aastex63}
\usepackage{amsmath,amstext,amssymb}
\usepackage[T1]{fontenc}
\usepackage{apjfonts}
\usepackage[figure,figure*]{hypcap}
\usepackage{graphics,graphicx}
\usepackage{hyperref}
\usepackage{natbib}
\usepackage[caption=false]{subfig} % for subfloat
\usepackage{enumitem} % for specific spacing of enumerate
\usepackage{epigraph}

\renewcommand*{\sectionautorefname}{Section} %for \autoref
\renewcommand*{\subsectionautorefname}{Section} %for \autoref

\newcommand{\cn}{Cep-Her complex} % cluster name
\newcommand{\sysone}{Kepler-1627} % star system name (binary)
\newcommand{\stone}{Kepler-1627 A} % star system name (binary)
\newcommand{\plone}{Kepler-1627 Ab} % planet name
\newcommand{\systwo}{Kepler-1643} % star system name (binary)
\newcommand{\sttwo}{Kepler-1643} % star system name (binary)
\newcommand{\pltwo}{Kepler-1643 b} % planet name
\newcommand{\systhree}{KOI-7368} % star system name (binary)
\newcommand{\stthree}{KOI-7368} % star system name (binary)
\newcommand{\plthree}{KOI-7368 b} % planet name
\newcommand{\sysfour}{KOI-7913 } % star system name (binary)
\newcommand{\stfour}{KOI-7913 A} % star system name (binary)
\newcommand{\plfour}{KOI-7913 Ab} % planet name

\newcommand{\clusterage}{$38^{+6}_{-5}$\,Myr} % 

%
% Symbols
%
\newcommand{\kms}{\,km\,s$^{-1}$}
\newcommand{\ms}{\,m\,s$^{-1}$}
\newcommand{\bpmrpo}{(G_{\rm BP}-G_{\rm RP})_0}
\newcommand{\bpmrp}{G_{\rm BP}-G_{\rm RP}}

%% Reintroduced the \received and \accepted commands from AASTeX v5.2.
%% Add "Submitted to " argument.
\received{2021 Oct 12}
\revised{---}
\accepted{---}
%\submitjournal{AAS Journals}
\shorttitle{Kepler Mini-Neptunes in Cep-Her}

\begin{document}

\title{
  Three 38 Million Year Old Mini-Neptunes from Kepler, TESS, and Gaia
}

%\suppressAffiliations
%\NewPageAfterKeywords
\correspondingauthor{L.\,G.\,Bouma}
\email{luke@astro.caltech.edu}

\author[0000-0002-0514-5538]{L. G. Bouma}
\altaffiliation{51 Pegasi b Fellow}
\affiliation{Cahill Center for Astrophysics, California Institute of Technology, Pasadena, CA 91125, USA}

% Key authors:
% ... Kinematics
\author[0000-0002-6549-9792]{R.~Kerr} % Y
\affiliation{Department of Astronomy, The University of Texas at Austin, Austin, TX 78712, USA}% % ... Kepler correlations
%
% ... stellar rotation & the initial crossmatch
\author[0000-0002-2792-134X]{J. L. Curtis} % Y
\affiliation{Department of Astronomy, Columbia University, 550 West 120th Street, New York, NY 10027, USA}
%
% HIRES
\author[0000-0002-0531-1073]{H. Isaacson} % Y
\affiliation{Astronomy Department, University of California, Berkeley, CA 94720, USA}
%
% planet-fitting, kinematics, cluster.
\author{L. A. Hillenbrand} % R
\affiliation{Cahill Center for Astrophysics, California Institute of Technology, Pasadena, CA 91125, USA}
%
% HIRES Collaborators
\author[0000-0001-8638-0320]{A. W. Howard} % Y
\affiliation{Cahill Center for Astrophysics, California Institute of Technology, Pasadena, CA 91125, USA}
%
% AO IMAGING
\author[0000-0001-9811-568X]{A.~L.~Kraus} % Y
\affiliation{Department of Astronomy, The University of Texas at Austin, Austin, TX 78712, USA}
%
% TRES
\author[0000-0001-6637-5401]{A. Bieryla} % Y
\affiliation{Center for Astrophysics \textbar \ Harvard \& Smithsonian, 60 Garden St, Cambridge, MA 02138, USA}
%
% TRES
\author[0000-0001-9911-7388]{D. W.~Latham} % Y
\affiliation{Center for Astrophysics \textbar \ Harvard \& Smithsonian, 60 Garden St, Cambridge, MA 02138, USA}
%
% AO / NIRC2
\author[0000-0003-0967-2893]{E. A.~Petigura} % Y
\affiliation{Department of Physics \& Astronomy, University of California Los Angeles, Los Angeles, CA 90095, USA}
%
% AO / NIRC2
\author[0000-0001-8832-4488]{D. Huber} % R
\affiliation{Institute for Astronomy, University of Hawaii, 2680 Woodlawn Drive, Honolulu, HI 96822, USA}


% 208 words (250 max)
\begin{abstract}
  Stellar positions and velocities from Gaia
  are yielding a refined view of stellar clusters during the
  first hundred million years of their lives.
  Here we present an analysis of a group of $38 \pm 6$ million year old stars
  spanning Cepheus ($l=100^\circ$) to Hercules ($l=40^\circ$), hereafter the Cep-Her complex.
  This group of stars includes four previously known Kepler Objects of
  Interest:
  Kepler-1627 Ab ($R_{\rm p} = 3.85 \pm 0.11\,R_\oplus$, $P = 7.2\ {\rm days}$),
  Kepler-1643 b ($R_{\rm p} = 2.32 \pm 0.14\,R_\oplus$, $P = 5.3\ {\rm days}$),
  KOI-7368 b ($R_{\rm p} = 2.22 \pm 0.12\,R_\oplus$, $P = 6.8\ {\rm days}$), and
  KOI-7913 Ab ($R_{\rm p} = 2.34 \pm 0.18\,R_\oplus$, $P = 24.2\ {\rm days}$).
  Kepler-1627 is a Neptune-sized planet in a component of the Cep-Her
  complex called the $\delta$\ Lyr\ cluster
  \citep{bouma_kep1627_2022}.  Here we focus on the latter three
  systems, which are in other sub-components of the complex (RSG-5 and
  CH-2).  Based on kinematic evidence from Gaia, stellar rotation
  periods from TESS, and spectroscopy, these three systems are also
  $38 \pm 6$ million years old.  Based on the transit shapes and high
  resolution imaging, we statistically validate that they are all most
  likely planets (false positive probabilities of $6\times10^{-9}$,
  $5\times10^{-3}$, and $1\times10^{-4}$ for Kepler-1643, KOI-7368,
  and KOI-7913 respectively).  Supplemented by Gaia and TESS, the main
  Kepler mission is now contributing to the census of young close-in
  planets, and Kepler-1643 and KOI-7913 are the first empirical
  demonstration that mini-Neptunes with sizes of $\approx$2 Earth
  radii exist at ages of roughly 40 million years.
\end{abstract}

\keywords{
  exoplanet evolution (491),
  open star clusters (1160),
	stellar ages (1581)
}

%%%%%%%%%%%%%%%%%%%%%%%%%%%%%%%%%%%%%%%%%%%%%%%%%%%%%%%%%%%%%%%%%%%%%%%%%%%%%%%


% * Main text <3500 words (not including acknowledgements, appendices, or other
%   supplementary)

\section{Introduction}


\section{The Cluster}
\label{sec:cluster}

\begin{figure*}[t]
	\begin{center}
		\leavevmode
		\includegraphics[width=0.99\textwidth]{f1.pdf}
	\end{center}
	\vspace{-0.7cm}
	\caption{
    {\bf Galactic positions  and tangential velocities of stars in the
    complex.} 
    Sub-clusters include the $\delta$ Lyr cluster, RSG-5, and the
    worryingly diffuse ``CH-2''.
		\label{fig:XYZvtang}
	}
\end{figure*}

\subsection{Selecting Cluster Members}
\label{sec:kinematicselection}

\begin{figure*}[tp]
	\begin{center}
		\leavevmode
		\subfloat{
			\includegraphics[width=0.49\textwidth]{f2a.pdf}
			\includegraphics[width=0.469\textwidth]{f2b.pdf}
		}
		
		\vspace{-0.6cm}
		\subfloat{
			\includegraphics[width=0.49\textwidth]{f2c.pdf}
			\includegraphics[width=0.49\textwidth]{TEMP_f2d.pdf}
		}
	\end{center}
	\vspace{-0.7cm}
	\caption{
		{\bf The \cn\ is \clusterage\ old.} 
    The top row shows CAMDs.  Left shows CH-2, right shows RSG-5.
    The bottom left shows gyro.
    The bottom right shows lithium (and is a {\bf place-holder}).  (Somewhere we will have
    H-alpha?).
		\label{fig:age}
	}
\end{figure*}

\subsection{The Cluster's Age}
\label{sec:clusterage}

\subsubsection{Color-Absolute Magnitude Diagram}
\label{sec:camd}

\subsubsection{Stellar Rotation Periods}

\section{The Stars}
\label{sec:stars}

\subsection{Kepler\,1627A}
\subsection{Kepler\,1643}
\subsection{KOI-7368}
\subsection{KOI-7913}
Is a binary.

\section{The Planets}
\label{sec:planet}

\begin{figure*}[tp]
	\begin{center}
		\leavevmode
		\subfloat{
			\includegraphics[width=0.99\textwidth]{f3e.pdf}
		}

		\subfloat{
			\includegraphics[width=0.5\textwidth]{f3a.pdf}
			\includegraphics[width=0.5\textwidth]{f3b.pdf}
		}
	
		\subfloat{
			\includegraphics[width=0.5\textwidth]{f3c.pdf}
			\includegraphics[width=0.5\textwidth]{f3d.pdf}
		}
	\end{center}
	\vspace{-0.7cm}
	\caption{
		{\bf Raw and processed light curves for the objects of
    interest.} Top: raw.  Bottom: processed.
    There increases scatter during transit is likely due to starspot
    crossing events.
		\label{fig:planets}
	}
\end{figure*}


\section{Discussion \& Conclusions}
\label{sec:conc}

\begin{figure*}[!t]
	\begin{center}
		\leavevmode
		\includegraphics[width=0.9\textwidth]{f4.pdf}
	\end{center}
	\vspace{-0.7cm}
	\caption{
    %    namelist = ['Kepler-1627 A', 'KOI-7368', 'KOI-7913 A', 'KOI-7913 B', 'Kepler-1643']
    %    markers = ['P', 'v', 'X', 'X', 's']
		{\bf Radii, orbital periods, and ages of transiting exoplanets}.
		Planets younger than a gigayear with ${\rm \tau}/\sigma_{\tau} >
		3$ are emphasized, where $\tau$ is the age and $\sigma_{\tau}$ is
    its uncertainty. Kepler-1627 (+), KOI-7368 (down-triangle),
    KOI-7913 (X), Kepler-1643 (diamond).  The large sizes of
		the youngest transiting planets could be explained by their
		primordial atmospheres not yet having evaporated; direct
		measurements of the atmospheric outflows or planetary masses would
		help to confirm this expectation.  Selection effects may also be
		important.  Parameters are from the NASA Exoplanet Archive (2022
		Feb 27).
		\label{fig:rp_period_age}
	}
\end{figure*}


%%%%%%%%%%%%%%%%%%%%%%%%%%%%%%%%%%%%%%%%%%%%%%%%%%%%%%%%%%%%%%%%%%%%%%%%%%%%%%%


%\clearpage
\acknowledgements
\raggedbottom
%
L.G.B{.} acknowledges support from the TESS GI Program (NASA grants
80NSSC19K0386 and 80NSSC19K1728) and the Heising-Simons Foundation (51 Pegasi~b
Fellowship).
% programs G011103 and G022117, through 
%
%
%FIXME
Keck/NIRC2 imaging was acquired by program 2015A/N301N2L
(PI: A.~Kraus). % and 2019A/N069 (PI: E.~Petigura).
%
% ACKNOWLEDGE PFS / CAMPANAS.
%
This paper also includes data collected by the TESS mission, which are
publicly available from the Mikulski Archive for Space Telescopes
(MAST).
%
Funding for the TESS mission is provided by NASA's Science Mission
directorate.
%
We thank the TESS Architects (G.~Ricker, R.~Vanderspek, D.~Latham,
S.~Seager, J.~Jenkins) and the many TESS team members for their
efforts to make the mission a continued success.
%
%
%This study was based in part on observations at Cerro Tololo
%Inter-American Observatory at NSF's NOIRLab (NOIRLab Prop{.} ID
%2020A-0146; 2020B-0029 PI: Bouma), which is managed by the
%Association of Universities for Research in Astronomy (AURA) under a
%cooperative agreement with the National Science Foundation.
%
%
%Finally, this research has made use of the Keck Observatory Archive (KOA),
%which is operated by the W. M. Keck Observatory and the NASA Exoplanet
%Science Institute (NExScI), under contract with the National
%Aeronautics and Space Administration.
Finally, we also thank the Keck Observatory staff for their support of
HIRES and remote observing.  We recognize the importance that the
summit of Maunakea has within the indigenous Hawaiian community, and
are deeply grateful to have the opportunity to conduct observations
from this mountain.
%
% The Digitized Sky Survey was produced at the Space Telescope Science
% Institute under U.S. Government grant NAG W-2166.
% Figure~\ref{fig:scene} is based on photographic data obtained using
% the Oschin Schmidt Telescope on Palomar Mountain.
%

% %
% This research made use of the NASA Exoplanet Archive, which is
% operated by the California Institute of Technology, under contract
% with the National Aeronautics and Space Administration under the
% Exoplanet Exploration Program.
% %

% Resources supporting this work were provided by the NASA High-End
% Computing (HEC) Program through the NASA Advanced Supercomputing (NAS)
% Division at Ames Research Center for the production of the SPOC data
% products.
%

% A.J.\ and R.B.\ acknowledge support from project IC120009 ``Millennium
% Institute of Astrophysics (MAS)'' of the Millenium Science Initiative,
% Chilean Ministry of Economy. A.J.\ acknowledges additional support
% from FONDECYT project 1171208.  J.I.V\ acknowledges support from
% CONICYT-PFCHA/Doctorado Nacional-21191829.  R.B.\ acknowledges support
% from FONDECYT Post-doctoral Fellowship Project 3180246.
% %
% C.T.\ and C.B\ acknowledge support from Australian Research Council
% grants LE150100087, LE160100014, LE180100165, DP170103491 and
% DP190103688.
% %
% C.Z.\ is supported by a Dunlap Fellowship at the Dunlap Institute for
% Astronomy \& Astrophysics, funded through an endowment established by
% the Dunlap family and the University of Toronto.
% %
% D.D.\ acknowledges support through the TESS Guest Investigator Program
% Grant 80NSSC19K1727.
%
%
%
% %
% Based on observations obtained at the Gemini Observatory, which is
% operated by the Association of Universities for Research in Astronomy,
% Inc., under a cooperative agreement with the NSF on behalf of the
% Gemini partnership: the National Science Foundation (United States),
% National Research Council (Canada), CONICYT (Chile), Ministerio de
% Ciencia, Tecnolog\'{i}a e Innovaci\'{o}n Productiva (Argentina),
% Minist\'{e}rio da Ci\^{e}ncia, Tecnologia e Inova\c{c}\~{a}o (Brazil),
% and Korea Astronomy and Space Science Institute (Republic of Korea).
% %
% Observations in the paper made use of the High-Resolution Imaging
% instrument Zorro at Gemini-South. Zorro was funded by the NASA
% Exoplanet Exploration Program and built at the NASA Ames Research
% Center by Steve B. Howell, Nic Scott, Elliott P. Horch, and Emmett
% Quigley.
% %
% This research has made use of the VizieR catalogue access tool, CDS,
% Strasbourg, France. The original description of the VizieR service was
% published in A\&AS 143, 23.
% %
% This work has made use of data from the European Space Agency (ESA)
% mission {\it Gaia} (\url{https://www.cosmos.esa.int/gaia}), processed
% by the {\it Gaia} Data Processing and Analysis Consortium (DPAC,
% \url{https://www.cosmos.esa.int/web/gaia/dpac/consortium}). Funding
% for the DPAC has been provided by national institutions, in particular
% the institutions participating in the {\it Gaia} Multilateral
% Agreement.
%
% (Some of) The data presented herein were obtained at the W. M. Keck
% Observatory, which is operated as a scientific partnership among the
% California Institute of Technology, the University of California and
% the National Aeronautics and Space Administration. The Observatory was
% made possible by the generous financial support of the W. M. Keck
% Foundation.
% The authors wish to recognize and acknowledge the very significant
% cultural role and reverence that the summit of Maunakea has always had
% within the indigenous Hawaiian community.  We are most fortunate to
% have the opportunity to conduct observations from this mountain.
%
% \newline
%

\software{
  %\texttt{arviz} \citep{arviz_2019},
  %\texttt{altaipony} \citep{ilin_flares_2021},
  \texttt{astrobase} \citep{bhatti_astrobase_2018},
  %\texttt{astroplan} \citep{astroplan2018},
	%\texttt{AstroImageJ} \citep{collins_astroimagej_2017},
  \texttt{astropy} \citep{astropy_2018},
  \texttt{astroquery} \citep{astroquery_2018},
  %\texttt{BATMAN} \citep{kreidberg_batman_2015},
  %\texttt{ceres} \citep{brahm_2017_ceres},
  %\texttt{cdips-pipeline} \citep{bhatti_cdips-pipeline_2019},
  \texttt{corner} \citep{corner_2016},
  %\texttt{emcee} \citep{foreman-mackey_emcee_2013},
  \texttt{exoplanet} \citep{exoplanet:exoplanet}, and its
  dependencies \citep{exoplanet:agol20, exoplanet:kipping13, exoplanet:luger18,
   	exoplanet:theano},
	%\texttt{gala} \citep{gala,PriceWhelan_2017_gala_zenodo},
	%\texttt{IDL Astronomy User's Library} \citep{landsman_1995},
  %\texttt{IPython} \citep{perez_2007},
	%\texttt{isochrones} \citep{morton_2015_isochrones},
	%\texttt{lightkurve} \citep{lightkurve_2018},
  %\texttt{matplotlib} \citep{hunter_matplotlib_2007}, 
  %\texttt{MESA} \citep{paxton_modules_2011,paxton_modules_2013,paxton_modules_2015}
  %\texttt{numpy} \citep{walt_numpy_2011}, 
  %\texttt{pandas} \citep{mckinney-proc-scipy-2010},
  %\texttt{pyGAM} \citep{serven_pygam_2018_1476122},
  \texttt{PyMC3} \citep{salvatier_2016_PyMC3},
  %\texttt{radvel} \citep{fulton_radvel_2018},
  %\texttt{scikit-learn} \citep{scikit-learn},
  \texttt{scipy} \citep{jones_scipy_2001},
  %\texttt{TESS-point}  \citep{burke_2020},
  %\texttt{tesscut} \citep{brasseur_astrocut_2019},
	%\texttt{VESPA} \citep{morton_efficient_2012,vespa_2015},
  %\texttt{webplotdigitzer} \citep{rohatgi_2019},
  %\texttt{wotan} \citep{hippke_wotan_2019}.
}
\ 

\facilities{
 	{\it Astrometry}:
 	Gaia \citep{gaia_collaboration_gaia_2018,gaia_collaboration_2021_edr3}.
 	{\it Imaging}:
    Second Generation Digitized Sky Survey. %,
    %SOAR~(HRCam; \citealt{tokovinin_ten_2018}).
 	Keck:II~(NIRC2; \url{www2.keck.hawaii.edu/inst/nirc2}).
 	%Gemini:South~(Zorro; \citealt{scott_nessi_2018}.
 	%Gemini:North~(`Alopeke; \citealt{scott_nessi_2018,scott_twin_2021}.
 	{\it Spectroscopy}:
	%CTIO1.5$\,$m~(CHIRON; \citealt{tokovinin_chironfiber_2013}),
  %PFS ({\bf CITE}),
	Tillinghast:1.5m~(TRES; \citealt{furesz_tres_2008}).
  %MPG2.2$\,$m~(FEROS; \citealt{kaufer_commissioning_1999}),
	%AAT~(Veloce; \citealt{gilbert_veloce_2018}).
	%AAT~(HERMES; \citealt{lewis_2002_hermers_2df,sheinis_2015_hermes}),
 	Keck:I~(HIRES; \citealt{vogt_hires_1994}).
 	%VLT:Kueyen~(FLAMES; \citealt{pasquini_2002}).
% 	Euler1.2m~(CORALIE),
% 	ESO:3.6m~(HARPS; \citealt{mayor_setting_2003}).
 	{\it Photometry}:
%	  ASTEP:0.40$\,$m (ASTEP400),
% 	CTIO:1.0m (Y4KCam),
% 	Danish 1.54m Telescope,
%	  El Sauce:0.356$\,$m,
% 	Elizabeth 1.0m at SAAO,
% 	Euler1.2m (EulerCam),
	  Kepler \citep{borucki_kepler_2010},
% 	Magellan:Baade (MagIC),
% 	Max Planck:2.2m	(GROND; \citealt{greiner_grond7-channel_2008})
%   MuSCAT3 \citep{Narita_2020},
% 	NTT,
% 	SOAR (SOI),
 	  TESS \citep{ricker_transiting_2015}.
% 	TRAPPIST \citep{jehin_trappist_2011},
% 	VLT:Antu (FORS2).
}

% \begin{table*}
\scriptsize
\setlength{\tabcolsep}{2pt}
\centering
\caption{Literature and Measured Properties for Kepler$\,$1627 A}
\label{tab:starparams}
%\tablenum{2}
\begin{tabular}{llcc}
  \hline
  \hline
Other identifiers\dotfill & \\
\multicolumn{3}{c}{TIC 120105470} \\
\multicolumn{3}{c}{GAIADR2 2103737241426734336} \\
\multicolumn{3}{c}{GAIAEDR3 2103737241426734336} \\
\hline
\hline
Parameter & Description & Value & Source\\
\hline 
$\alpha_{J2015.5}$\dotfill	&Right Ascension (hh:mm:ss)\dotfill & 18:56:13.6 & 1	\\
$\delta_{J2015.5}$\dotfill	&Declination (dd:mm:ss)\dotfill & +41:34:36.22 & 1	\\
%$l_{J2015.5}$\dotfill	&Galactic Longitude (deg)\dotfill & 288.2644 & 1	\\
%$b_{J2015.5}$\dotfill	&Galactic Latitude (deg)\dotfill & -5.7950 & 1	\\
%\\
%$NUV$\dotfill           & GALEX $NUV$ mag.\dotfill & 13.804 $\pm$ 0.004 & 2 \\
%$FUV$\dotfill           & GALEX $FUV$ mag.\dotfill & 18.466 $\pm$ 0.056 & 2 \\
\\
%B\dotfill			&Johnson B mag.\dotfill & 11.119 $\pm$ 0.107		& 2	\\
V\dotfill			&Johnson V mag.\dotfill & 13.11 $\pm$ 0.08		& 2	\\
%$B$\tablenote{The uncertainties of the photometry have a systematic error floor applied. Even still, the global fit requires a significant scaling of the uncertainties quoted here to be consistent with our model, suggesting they are still significantly underestimated for one or more of the broad band magnitudes}\dotfill		& APASS Johnson $B$ mag.\dotfill	& 13.001 $\pm$	0.02& 2	\\
%$V$\dotfill		& APASS Johnson $V$ mag.\dotfill	& 11.808 $\pm$	0.02& 2	\\
%\\
${\rm G}$\dotfill     & Gaia $G$ mag.\dotfill     & 13.049$\pm$0.02 & 1\\
%${\rm Bp}$\dotfill     & Gaia $Bp$ mag.\dotfill     & 10.695 $\pm$0.020 & 1\\
%${\rm Rp}$\dotfill     & Gaia $Rp$ mag.\dotfill     & 9.887$\pm$0.020 & 1\\
${\rm T}$\dotfill     & TESS $T$ mag.\dotfill     & 12.53$\pm$0.02 & 2\\
%$u'$\dotfill        & Sloan $u'$ mag.\dotfill & 14.706 $\pm$ 0.006& 3\\
%$g'$\dotfill		& APASS Sloan $g'$ mag.\dotfill	& 12.407 $\pm$ 0.02	&  2	\\
%$r'$\dotfill		& APASS Sloan $r'$ mag.\dotfill	& 11.311 $\pm$ 0.02	&  2	\\
%$i'$\dotfill		& APASS Sloan $i'$ mag.\dotfill	& 10.927 $\pm$ 0.04 &  2	\\
%\\
J\dotfill			& 2MASS J mag.\dotfill & 11.69  $\pm$ 0.02	& 3	\\
H\dotfill			& 2MASS H mag.\dotfill & 11.30 $\pm$ 0.02	    &  3	\\
K$_{\rm S}$\dotfill			& 2MASS ${\rm K_S}$ mag.\dotfill & 11.19 $\pm$ 0.02 &  3	\\
%\\
%W1\dotfill		& WISE1 mag.\dotfill & 8.901 $\pm$ 0.023 & 4	\\
%W2\dotfill		& WISE2 mag.\dotfill & 8.875 $\pm$ 0.021 &  4 \\
%W3\dotfill		& WISE3 mag.\dotfill &  8.875 $\pm$ 0.020& 4	\\
%W4\dotfill		& WISE4 mag.\dotfill & 8.936 $\pm$ N/A &  4	\\
\\
$\pi$\dotfill & Gaia EDR3 parallax (mas) \dotfill & 3.009 $\pm$ 0.032 &  1 \\
$d$\dotfill & Distance (pc)\dotfill & $329.5 \pm 3.5$ & 1, 4 \\
$\mu_{\alpha}$\dotfill		& Gaia EDR3 proper motion\dotfill & 1.716 $\pm$ 0.034 & 1 \\
                    & \hspace{3pt} in RA (mas yr$^{-1}$)	&  \\
$\mu_{\delta}$\dotfill		& Gaia EDR3 proper motion\dotfill 	&  -1.315 $\pm$ 0.034 &  1 \\
                    & \hspace{3pt} in DEC (mas yr$^{-1}$) &  \\
RUWE\dotfill		& Gaia EDR3 renormalized\dotfill 	&  2.899 &  1 \\
                    & \hspace{3pt} unit weight error &  \\
%
\\
RV\dotfill & Systemic radial \hspace{9pt}\dotfill  & $-14.3 \pm 1.0$ & 5 \\
                    & \hspace{3pt} velocity (\kms)  & \\
Spec. Type\dotfill & Spectral Type\dotfill & 	G8V & 5 \\
$v\sin{i_\star}$\dotfill &  Rotational velocity (\kms) \hspace{9pt}\dotfill &  18.9 $\pm$ 1.0 & 5 \\
Li EW\dotfill & 6708\AA\ Equiv{.} Width (m\AA) \dotfill & $233^{+5}_{-7}$  & 5 \\
%$v_{\rm mac}$\dotfill &  Macroturbulence velocity (\kms) \hspace{9pt}\dotfill &  8.4 $\pm$ 2.9 & 5 \\
%${\rm [Fe/H]}$\dotfill &   Metallicity$^\dagger$ \hspace{9pt}\dotfill & -0.02 $\pm$ 0.09 & 5 \\
%$T_{\rm eff}$\dotfill &  Effective Temperature (K) \hspace{9pt}\dotfill & 5777 $\pm$ 110 &  5  \\
%$\log{g_{\star}}$\dotfill &  Surface Gravity (cgs)\hspace{9pt}\dotfill &  4.6 $\pm$ 0.1  &  5 \\
$T_{\rm eff}$\dotfill &  Effective Temperature (K) \hspace{9pt}\dotfill & 5505 $\pm$ 39 &  6  \\
$\log{g_{\star}}$\dotfill &  Surface Gravity (cgs)\hspace{9pt}\dotfill &  4.53 $\pm$ 0.02  &  6 \\
%
% $E(B-V)$\dotfill & Reddening (mag)\dotfill & $0.06 \pm 0.02$ & 9 \\
%
%
$R_\star$\dotfill & Stellar radius ($R_\odot$)\dotfill & 0.881$\pm$0.018 & 6 \\
$M_\star$\dotfill & Stellar mass ($R_\odot$)\dotfill & 0.953$\pm$0.019 & 6 \\
%$F_{\rm bol}$\dotfill & Stellar bolometric flux (cgs)\dotfill & (1.967$\pm$0.046)$\times10^{-9}$ & 9 \\
%
%
$A_{\rm V}$\dotfill & Interstellar reddening (mag)\dotfill & 0.2$\pm$0.1 & 6 \\
${\rm [Fe/H]}$\dotfill &   Metallicity \hspace{9pt}\dotfill & 0.1 $\pm$ 0.1 & 6 \\
%
$P_{\rm rot}$\dotfill & Rotation period (d)\dotfill & $2.642\pm 0.042$  & 7 \\
Age & Adopted stellar age (Myr)\dotfill & $38^{+6}_{-5}$  &  8 \\
% $U^{*}$\dotfill & Space Velocity (\kms)\dotfill & $26.24 \pm 0.46$  & \S\ref{sec:uvw} \\
% $V$\dotfill       & Space Velocity (\kms)\dotfill & $-71.52 \pm 1.68$ & \S\ref{sec:uvw} \\
% $W$\dotfill       & Space Velocity (\kms)\dotfill & $ -1.31 \pm 0.27$ & \S\ref{sec:uvw} \\
\hline
\end{tabular}
\begin{flushleft}
 \footnotesize{ \textsc{NOTE}---
Provenances are:
$^1$\citet{gaia_collaboration_gaia_2018},
$^2$\citet{stassun_TIC8_2019},
$^3$\citet{skrutskie_tmass_2006},
$^4$\citet{Lindegren_2021_offset},
$^5$HIRES spectra and \citet{yee_SM_2017},
$^6$Cluster isochrone (MIST+PARSEC),
%$^7$FEROS spectra,
$^7$\citet{capitanio_threedimensional_2017} and \citet{lallement_threedimensional_2018},
$^8$Kepler light curve,
$^9$Pre-main-sequence CMD, with LDB age for IC~2602 being most
important (Section~\ref{sec:cmd}).
%$*$ $U$ is in the direction of the Galactic center. \\
%$^{10}$Method~1 (photometric SED fit, Section~\ref{subsec:starparams}).}
}
\end{flushleft}
\vspace{-0.5cm}
\end{table*}

% % Table of best fit parameters
%\startlongtable
\begin{deluxetable*}{lllrrrrrrr}
%
  \tablecaption{ Priors and Posteriors for Model Fitted to the Long
  Cadence Kepler 1627Ab Light Curve.}
\label{tab:posterior}
%
\tabletypesize{\scriptsize}
%\tabletypesize{\small}
%
%\tablenum{2}
%
\tablehead{
  \colhead{Param.} & 
  \colhead{Unit} &
  \colhead{Prior} & 
  \colhead{Median} & 
  \colhead{Mean} & 
  \colhead{Std{.} Dev.} &
  \colhead{3\%} &
  \colhead{97\%} &
  \colhead{ESS} &
  \colhead{$\hat{R}-1$}
}

%/Users/luke/Dropbox/proj/rudolf/results/run_RotGPtransit/Kepler_1627_RotGPtransit_posteriortable.tex
\startdata
{\it Sampled} & & & & & & & & & \\
\hline
$P$ & d & $\mathcal{N}(7.20281; 0.01000)$ & 7.2028038 & 7.2028038 & 0.0000073 & 7.2027895 & 7.2028168 & 7464 & 3.9e-04 \\
$t_0^{(1)}$ & d & $\mathcal{N}(120.79053; 0.02000)$ & 120.7904317 & 120.7904254 & 0.0009570 & 120.7886377 & 120.7921911 & 3880 & 2.0e-03 \\
$\log \delta$ & -- & $\mathcal{N}(-6.3200; 2.0000)$ & -6.3430 & -6.3434 & 0.0354 & -6.4094 & -6.2767 & 6457 & 3.0e-04 \\
$b^{(2)}$ & -- & $\mathcal{U}(0.000; 1.000)$ & 0.4669 & 0.4442 & 0.2025 & 0.0662 & 0.8133 & 1154 & 1.6e-03 \\
$u_1$ & -- & \citet{exoplanet:kipping13} & 0.271 & 0.294 & 0.190 & 0.000 & 0.628 & 3604 & 1.5e-03 \\
$u_2$ & -- & \citet{exoplanet:kipping13} & 0.414 & 0.377 & 0.326 & -0.240 & 0.902 & 3209 & 1.4e-03 \\
$R_\star$ & $R_\odot$ & $\mathcal{N}(0.881; 0.018)$ & 0.881 & 0.881 & 0.018 & 0.847 & 0.915 & 8977 & 3.1e-04 \\
$\log g$ & cgs & $\mathcal{N}(4.530; 0.050)$ & 4.532 & 4.533 & 0.051 & 4.435 & 4.627 & 6844 & 1.6e-03 \\
$\langle f \rangle$ & -- & $\mathcal{N}(0.000; 0.100)$ & -0.0003 & -0.0003 & 0.0001 & -0.0005 & -0.0000 & 8328 & 1.1e-03 \\
$e^{(3)}$ & -- & \citet{vaneylen19} & 0.154 & 0.186 & 0.152 & 0.000 & 0.459 & 1867 & 2.0e-03 \\
$\omega$ & rad & $\mathcal{U}(0.000; 6.283)$ & 0.055 & 0.029 & 1.845 & -3.139 & 2.850 & 3557 & 8.6e-05 \\
$\log \sigma_f$ & -- & $\mathcal{N}(\log\langle \sigma_f \rangle; 2.000)$ & -8.035 & -8.035 & 0.008 & -8.049 & -8.021 & 9590 & 3.9e-04 \\
$\sigma_{\mathrm{rot}}$ & d$^{-1}$ & $\mathrm{InvGamma}(1.000; 5.000)$ & 0.070 & 0.070 & 0.001 & 0.068 & 0.072 & 9419 & 1.4e-03 \\
$\log P_{\mathrm{rot}}$ & $\log (\mathrm{d})$ & $\mathcal{N}(0.958; 0.020)$ & 0.978 & 0.978 & 0.001 & 0.975 & 0.980 & 8320 & 2.2e-04 \\
$\log Q_0$ & -- & $\mathcal{N}(0.000; 2.000)$ & -0.327 & -0.326 & 0.043 & -0.407 & -0.246 & 9659 & 2.7e-04 \\
$\log \mathrm{d}Q$ & -- & $\mathcal{N}(0.000; 2.000)$ & 7.697 & 7.698 & 0.103 & 7.511 & 7.899 & 5824 & 3.7e-04 \\
$f$ & -- & $\mathcal{U}(0.010; 1.000)$ & 0.01006 & 0.01009 & 0.00009 & 0.01000 & 0.01025 & 4645 & 4.0e-04 \\
{\it Derived} & & & & & & & & & \\
\hline
$\delta$ & -- & -- & 0.001759 & 0.001759 & 0.000062 & 0.001641 & 0.001875 & 6457 & 3.0e-04 \\
$R_{\rm p}/R_\star$ & -- & -- & 0.039 & 0.039 & 0.001 & 0.037 & 0.042 & 1811 & 1.1e-03 \\
$\rho_\star$ & g$\ $cm$^{-3}$ & -- & 1.990 & 2.004 & 0.240 & 1.570 & 2.461 & 6905 & 2.1e-03 \\
$R_{\rm p}^{(4)}$ & $R_{\mathrm{Jup}}$ & -- & 0.337 & 0.338 & 0.014 & 0.314 & 0.367 & 2311 & 1.0e-03 \\
$R_{\rm p}^{(4)}$ & $R_{\mathrm{Earth}}$ & -- & 3.777 & 3.789 & 0.157 & 3.52 & 4.114 & 2311 & 1.0e-03 \\
$a/R_\star$ & -- & -- & 17.606 & 17.619 & 0.702 & 16.277 & 18.906 & 6905 & 2.1e-03 \\
$\cos i$ & -- & -- & 0.027 & 0.025 & 0.010 & 0.004 & 0.040 & 1312 & 1.2e-03 \\
$T_{14}$ & hr & -- & 2.841 & 2.843 & 0.060 & 2.734 & 2.958 & 3199 & 3.6e-04 \\
$T_{13}$ & hr & -- & 2.555 & 2.539 & 0.094 & 2.360 & 2.692 & 1960 & 1.4e-03 \\
\enddata
%
\tablecomments{
  ESS refers to the number of effective samples.
  $\hat{R}$ is the Gelman-Rubin convergence diagnostic.
  Logarithms in this table are base-$e$.
  $\mathcal{U}$ denotes a uniform distribution,
  and $\mathcal{N}$ a normal distribution.
  (1) The ephemeris is in units of BJDTDB - 2454833.
  (2) Although $\mathcal{U}(0,1+R_{\rm p}/R_\star)$ is formally
  correct, for this model we assumed a non-grazing transit to enable
  sampling in $\log \delta$.
  (3) The eccentricity vectors are sampled in the $(e\cos\omega,
  e\sin\omega)$ plane.
  (4) The true planet size is a factor of $((F_1+F_2)/F_1)^{1/2}$
  larger than that from the fit because of dilution from Kepler
  1627B, where $F_1$ is the flux from the primary, and $F_2$ is that
  from the secondary; the mean and standard deviation of $R_{\rm
  p}=3.817\pm0.158\,R_{\oplus}$ quoted in the text includes this correction,
  assuming $(F_1+F_2)/F_1\approx 1.015$.
}
\vspace{-0.3cm}
\end{deluxetable*}

% %% \begin{deluxetable}{} command tell LaTeX how many columns
%% there are and how to align them.
%\startlongtable
\begin{deluxetable*}{lll}
    
%% Keep a portrait orientation

%% Over-ride the default font size
%% Use Default (12pt)
\tabletypesize{\scriptsize}
%\tabletypesize{\small}

%% Use \tablewidth{?pt} to over-ride the default table width.
%% If you are unhappy with the default look at the end of the
%% *.log file to see what the default was set at before adjusting
%% this value.

%% This is the title of the table.
\tablecaption{Young, Age-dated, and Age-dateable Stars Within the
  Nearest Few Kiloparsecs.}
\label{tab:v06}

%% This command over-rides LaTeX's natural table count
%% and replaces it with this number.  LaTeX will increment 
%% all other tables after this table based on this number
%\tablenum{3}

%% The \tablehead gives provides the column headers.  It
%% is currently set up so that the column labels are on the
%% top line and the units surrounded by ()s are in the 
%% bottom line.  You may add more header information by writing
%% another line between these lines. For each column that requries
%% extra information be sure to include a \colhead{text} command
%% and remember to end any extra lines with \\ and include the 
%% correct number of &s.
\tablehead{
  \colhead{Parameter} &
  \colhead{Example Value} &
  \colhead{Description}
}

%% All data must appear between the \startdata and \enddata commands
%
% paste from
% /Users/luke/Dropbox/proj/rudolf/results/tables/v06_main_tableheader.tex
% via drivers/write_v06_main_tableheader.py
\startdata
          \texttt{source\_id} &                                          1709456705329541504 &                                              Gaia DR2 source identifier. \\
                  \texttt{ra} &                                                      247.826 &                                          Gaia DR2 right ascension [deg]. \\
                 \texttt{dec} &                                                       79.789 &                                              Gaia DR2 declination [deg]. \\
            \texttt{parallax} &                                                       35.345 &                                                 Gaia DR2 parallax [mas]. \\
     \texttt{parallax\_error} &                                                        0.028 &                                     Gaia DR2 parallax uncertainty [mas]. \\
                \texttt{pmra} &                                                       94.884 &      Gaia DR2 proper motion $\mu_\alpha \cos \delta$ [mas$\,$yr$^{-1}$]. \\
               \texttt{pmdec} &                                                      -86.971 &                  Gaia DR2 proper motion $\mu_\delta$ [mas$\,$yr$^{-1}$]. \\
  \texttt{phot\_g\_mean\_mag} &                                                         6.85 &                                                  Gaia DR2 $G$ magnitude. \\
 \texttt{phot\_bp\_mean\_mag} &                                                        6.409 &                                      Gaia DR2 $G_\mathrm{BP}$ magnitude. \\
 \texttt{phot\_rp\_mean\_mag} &                                                        7.189 &                                      Gaia DR2 $G_\mathrm{RP}$ magnitude. \\
             \texttt{cluster} &                  NASAExoArchive\_ps\_20210506,Uma,IR\_excess &                                   Comma-separated cluster or group name. \\
                 \texttt{age} &                                                 9.48,nan,nan &  Comma-separated logarithm (base-10) of reported$^{\rm a}$ age in years. \\
           \texttt{mean\_age} &                                                         9.48 &                           Mean (ignoring NaNs) of $\texttt{age}$ column. \\
       \texttt{reference\_id} &       NASAExoArchive\_ps\_20210506,Ujjwal2020,CottenSong2016 &                           Comma-separted provenance of group membership. \\
  \texttt{reference\_bibcode} &  2013PASP..125..989A,2020AJ....159..166U,2016ApJS..225...15C &                   ADS bibcode corresponding to $\texttt{reference\_id}$. \\
\enddata

%% Include any \tablenotetext{key}{text}, \tablerefs{ref list},
%% or \tablecomments{text} between the \enddata and 
%% \end{deluxetable} commands

%% General table comment marker
\tablecomments{
Table~\ref{tab:v06} is published in its entirety in a machine-readable
format.   This table is a concatenation of the studies listed in
Table~\ref{tab:metadata}.  One entry is shown for guidance regarding
form and content.  In this particular example, the star has a cold
Jupiter on a 16 year orbit, HD 150706b \citep{2012AA...545A..55B}.  An
infrared excess has been reported \citep{CottenSong2016}, and the star
was identified by \citet{Ujjwal2020} as a candidate UMa moving group
member ($\approx 400\,{\rm Myr}$; \citealt{mann_tess_2020}).  The
star's RV activity and TESS rotation period corroborate its youth.
}
\vspace{-0.5cm}
\end{deluxetable*}

% %% \begin{deluxetable}{} command tell LaTeX how many columns
%% there are and how to align them.
%\startlongtable
\begin{deluxetable*}{lccc}
    
%% Keep a portrait orientation

%% Over-ride the default font size
%% Use Default (12pt)
\tabletypesize{\scriptsize}
%\tabletypesize{\small}
%\tabletypesize{\normal}

%% Use \tablewidth{?pt} to over-ride the default table width.
%% If you are unhappy with the default look at the end of the
%% *.log file to see what the default was set at before adjusting
%% this value.

%% This is the title of the table.
\tablecaption{Provenances of Young and Age-dateable Stars.}
\label{tab:metadata}

%% This command over-rides LaTeX's natural table count
%% and replaces it with this number.  LaTeX will increment 
%% all other tables after this table based on this number
%\tablenum{3}

%% The \tablehead gives provides the column headers.  It
%% is currently set up so that the column labels are on the
%% top line and the units surrounded by ()s are in the 
%% bottom line.  You may add more header information by writing
%% another line between these lines. For each column that requries
%% extra information be sure to include a \colhead{text} command
%% and remember to end any extra lines with \\ and include the 
%% correct number of &s.
\tablehead{
  \colhead{Reference} &
  \colhead{$N_{\rm Gaia}$} &
  \colhead{$N_{\rm Age}$} &
  \colhead{$N_{G_{\rm RP}<16}$}
}

%% All data must appear between the \startdata and \enddata commands
%
% paste from
% /Users/luke/Dropbox/proj/rudolf/results/tables/metadata_table_data.tex
% via drivers/write_metadata_table.py
\startdata
                           \citet{Kounkel2020}  &             987376 &            987376 &                775363 \\
                     \citet{CantatGaudin2020a}  &             433669 &            412671 &                269566 \\
                     \citet{CantatGaudin2018a}  &             399654 &            381837 &                246067 \\
                      \citet{KounkelCovey2019}  &             288370 &            288370 &                229506 \\
                     \citet{CantatGaudin2020b}  &             233369 &            227370 &                183974 \\
                           \citet{Zari2018} UMS &              86102 &                 0 &                 86102 \\
                  \citet{SIMBAD} $\texttt{Y*?}$ &              61432 &                 0 &                 45076 \\
                           \citet{Zari2018} PMS &              43719 &                 0 &                 38435 \\
\citet{GaiaCollaboration2018} $d>250\,{\rm pc}$ &              35506 &             31182 &                 18830 \\
                      \citet{CastroGinard2020}  &              33635 &             24834 &                 31662 \\
                              \citet{Kerr2021}  &              30518 &             25324 &                 27307 \\
                  \citet{SIMBAD} $\texttt{Y*O}$ &              28406 &                 0 &                 16205 \\
                        \citet{VillaVelez2018}  &              14459 &             14459 &                 13866 \\
                     \citet{CantatGaudin2019a}  &              11843 &             11843 &                  9246 \\
                        \citet{Damiani2019} PMS &              10839 &             10839 &                  9901 \\
                                \citet{Oh2017}  &              10379 &                 0 &                 10370 \\
                          \citet{Meingast2021}  &               7925 &              7925 &                  5878 \\
                 \citet{SIMBAD} $\texttt{pMS*}$ &               5901 &                 0 &                  3006 \\
\citet{GaiaCollaboration2018} $d<250\,{\rm pc}$ &               5378 &               817 &                  3968 \\
                           \citet{Kounkel2018}  &               5207 &              3740 &                  5207 \\
                        \citet{Ratzenbock2020}  &               4269 &              4269 &                  2662 \\
                  \citet{SIMBAD} $\texttt{TT*}$ &               4022 &                 0 &                  3344 \\
                        \citet{Damiani2019} UMS &               3598 &              3598 &                  3598 \\
                           \citet{Rizzuto2017}  &               3294 &              3294 &                  2757 \\
            \citet{NASAExoArchive_ps_20210506}  &               3107 &               868 &                  3098 \\
                              \citet{Tian2020}  &               1989 &              1989 &                  1394 \\
                           \citet{Goldman2018}  &               1844 &              1844 &                  1783 \\
                        \citet{CottenSong2016}  &               1695 &                 0 &                  1693 \\
                            \citet{Gagne2018a}  &               1429 &                 0 &                  1389 \\
             \citet{RoserSchilbach2020} Psc-Eri &               1387 &              1387 &                  1107 \\
            \citet{RoserSchilbach2020} Pleiades &               1245 &              1245 &                  1019 \\
                  \citet{SIMBAD} $\texttt{TT?}$ &               1198 &                 0 &                   853 \\
                            \citet{Gagne2018c}  &                914 &                 0 &                   913 \\
                          \citet{Pavlidou2021}  &                913 &               913 &                   504 \\
                            \citet{Gagne2018b}  &                692 &                 0 &                   692 \\
                            \citet{Ujjwal2020}  &                563 &                 0 &                   563 \\
                             \citet{Gagne2020}  &                566 &               566 &                   351 \\
                      \citet{EsplinLuhman2019}  &                377 &               443 &                   296 \\
                     \citet{Roccatagliata2020}  &                283 &               283 &                   232 \\
                          \citet{Meingast2019}  &                238 &               238 &                   238 \\
                 \citet{Furnkranz2019} Coma-Ber &                214 &               214 &                   213 \\
           \citet{Furnkranz2019} Neighbor Group &                177 &               177 &                   167 \\
                             \citet{Kraus2014}  &                145 &               145 &                   145 \\
\enddata

%% Include any \tablenotetext{key}{text}, \tablerefs{ref list},
%% or \tablecomments{text} between the \enddata and 
%% \end{deluxetable} commands

%% General table comment marker
\tablecomments{
Table~\ref{tab:metadata} describes the provenances for the young and
age-dateable stars in Table~\ref{tab:v06}.  $N_{\rm Gaia}$: number of
Gaia stars we parsed from the literature source.  $N_{\rm Age}$:
number of stars in the literature source with ages reported.
$N_{G_{\rm RP}<16}$: number of Gaia stars we parsed from the
literature source with either $G_{\rm RP}<16$, or a parallax S/N
exceeding 5 and a distance closer than 100\,pc.  The latter criterion
included a few hundred white dwarfs that would have otherwise been
neglected.  Some studies are listed multiple times if they contain
multiple tables.  \citet{SIMBAD} refers to the \texttt{SIMBAD}
database.
}
\vspace{-0.5cm}
\end{deluxetable*}


\clearpage
\bibliographystyle{yahapj}                            
\bibliography{bibliography} 

%\appendix
%\section{Young, Age-Dated, and Age-Dateable Star Compilation}
%\label{app:targetlist}


%\listofchanges
%\allauthors
\end{document}
