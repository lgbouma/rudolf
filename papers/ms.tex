%\documentclass[12pt,modern,twocolumn,tighten]{aastex63}
\documentclass[12pt,twocolumn]{aastex63}
%\documentclass[12pt,modern,twocolumn,tighten,linenumbers,trackchanges]{aastex63}
%\documentclass[12pt,twocolumn,tighten,linenumbers]{aastex63}
%\documentclass[12pt,twocolumn,tighten,trackchanges]{aastex63}
\usepackage{amsmath,amstext,amssymb}
\usepackage[T1]{fontenc}
\usepackage{apjfonts}
\usepackage[figure,figure*]{hypcap}
\usepackage{graphics,graphicx}
\usepackage{hyperref}
\usepackage{natbib}
\usepackage[caption=false]{subfig} % for subfloat
\usepackage{enumitem} % for specific spacing of enumerate
\usepackage{epigraph}

\renewcommand*{\sectionautorefname}{Section} %for \autoref
\renewcommand*{\subsectionautorefname}{Section} %for \autoref

\newcommand{\cn}{Cep-Her complex} % cluster name
\newcommand{\sysone}{Kepler-1627} % star system name (binary)
\newcommand{\stone}{Kepler-1627 A} % star system name (binary)
\newcommand{\plone}{Kepler-1627 Ab} % planet name
\newcommand{\systwo}{Kepler-1643} % star system name (binary)
\newcommand{\sttwo}{Kepler-1643} % star system name (binary)
\newcommand{\pltwo}{Kepler-1643 b} % planet name
\newcommand{\systhree}{KOI-7368} % star system name (binary)
\newcommand{\stthree}{KOI-7368} % star system name (binary)
\newcommand{\plthree}{KOI-7368 b} % planet name
\newcommand{\sysfour}{KOI-7913 } % star system name (binary)
\newcommand{\stfour}{KOI-7913 A} % star system name (binary)
\newcommand{\plfour}{KOI-7913 Ab} % planet name

\newcommand{\clusterage}{$38^{+6}_{-5}$\,Myr} % 

\newcommand{\npms}{1{,}097} % 20220311_Kerr_SPYGLASS205_Members_All.csv
\newcommand{\nchtwo}{37} % CH-2_auto_XYZ_vl_vb_cut.csv
\newcommand{\nrsgfive}{173} % RSG-5_auto_XYZ_vl_vb_cut.csv

%
% Symbols
%
\newcommand{\kms}{\,km\,s$^{-1}$}
\newcommand{\mkms}{{\rm \,km\,s^{-1}}}  % math mode
\newcommand{\ms}{\,m\,s$^{-1}$}
\newcommand{\bpmrpo}{(G_{\rm BP}-G_{\rm RP})_0}
\newcommand{\bpmrp}{G_{\rm BP}-G_{\rm RP}}

%% Reintroduced the \received and \accepted commands from AASTeX v5.2.
%% Add "Submitted to " argument.
\received{---}
\revised{---}
\accepted{---}
%\submitjournal{AAS Journals}
\shorttitle{Trio of Mini-Neptunes in Cep-Her}

\begin{document}

\title{
  Kepler and the Behemoth: Three Mini-Neptunes in a 40 Million Year Old Association
}

%\suppressAffiliations
%\NewPageAfterKeywords
\correspondingauthor{L.\,G.\,Bouma}
\email{luke@astro.caltech.edu}

\author[0000-0002-0514-5538]{L. G. Bouma}
\altaffiliation{51 Pegasi b Fellow}
\affiliation{Cahill Center for Astrophysics, California Institute of Technology, Pasadena, CA 91125, USA}

% Key authors:
% ... Kinematics
\author[0000-0002-6549-9792]{R.~Kerr} % Y
\affiliation{Department of Astronomy, The University of Texas at Austin, Austin, TX 78712, USA}% % ... Kepler correlations
%
% ... stellar rotation & the initial crossmatch
\author[0000-0002-2792-134X]{J. L. Curtis} % Y
\affiliation{Department of Astronomy, Columbia University, 550 West 120th Street, New York, NY 10027, USA}
%
% HIRES
\author[0000-0002-0531-1073]{H. Isaacson} % Y
\affiliation{Astronomy Department, University of California, Berkeley, CA 94720, USA}
%
% planet-fitting, kinematics, cluster.
\author{L. A. Hillenbrand} % R
\affiliation{Cahill Center for Astrophysics, California Institute of Technology, Pasadena, CA 91125, USA}
%
% HIRES Collaborators
\author[0000-0001-8638-0320]{A. W. Howard} % Y
\affiliation{Cahill Center for Astrophysics, California Institute of Technology, Pasadena, CA 91125, USA}
%
% AO IMAGING
\author[0000-0001-9811-568X]{A.~L.~Kraus} % Y
\affiliation{Department of Astronomy, The University of Texas at Austin, Austin, TX 78712, USA}
%
% TRES
\author[0000-0001-6637-5401]{A. Bieryla} % Y
\affiliation{Center for Astrophysics \textbar \ Harvard \& Smithsonian, 60 Garden St, Cambridge, MA 02138, USA}
%
% TRES
\author[0000-0001-9911-7388]{D. W.~Latham} % Y
\affiliation{Center for Astrophysics \textbar \ Harvard \& Smithsonian, 60 Garden St, Cambridge, MA 02138, USA}
%
% AO / NIRC2
\author[0000-0003-0967-2893]{E. A.~Petigura} % Y
\affiliation{Department of Physics \& Astronomy, University of California Los Angeles, Los Angeles, CA 90095, USA}
%
% AO / NIRC2
\author[0000-0001-8832-4488]{D. Huber} % R
\affiliation{Institute for Astronomy, University of Hawaii, 2680 Woodlawn Drive, Honolulu, HI 96822, USA}


% 208 words (250 max)
\begin{abstract}
  Stellar positions and velocities from Gaia are yielding a new window
  on open cluster dispersal.  Here we present an analysis of a group
  of stars spanning Cepheus ($l=100^\circ$) to Hercules ($l=40^\circ$),
  hereafter the Cep-Her complex.  The group includes four Kepler
  Objects of Interest:
  Kepler-1643 b ($R_{\rm p} = 2.32 \pm 0.14\,R_\oplus$, $P = 5.3\ {\rm days}$),
  KOI-7368 b ($R_{\rm p} = 2.22 \pm 0.12\,R_\oplus$, $P = 6.8\ {\rm days}$), 
  KOI-7913 Ab ($R_{\rm p} = 2.34 \pm 0.18\,R_\oplus$, $P = 24.2\ {\rm days}$), and
  Kepler-1627 Ab ($R_{\rm p} = 3.85 \pm 0.11\,R_\oplus$, $P = 7.2\ {\rm days}$).
  The latter Neptune-sized planet is in part of the Cep-Her complex
  called the $\delta$\ Lyr\ cluster \citep{bouma_kep1627_2022}.  Here
  we focus on the former three systems, which are in other regions of
  the association.  Based on kinematic evidence from Gaia, stellar
  rotation periods from TESS, and spectroscopy, these three objects
  are also $\approx$40 million years (Myr) old.  More specifically, we
  find that Kepler-1643 is $46^{+9}_{-7}$\,Myr old, based on its
  membership in a dense sub-cluster of the complex called RSG-5.
  KOI-7368 and KOI-7913 are $36^{+10}_{-8}$\,Myr old, and are in a
  diffuse region that we call CH-2.  Based on the transit shapes and
  high resolution imaging, all three objects are most likely planets,
  with false positive probabilities of $6\times10^{-9}$,
  $4\times10^{-3}$, and $1\times10^{-4}$ for Kepler-1643, KOI-7368,
  and KOI-7913 respectively.  These planets demonstrate
  that mini-Neptunes with sizes of $\sim$2 Earth radii exist at ages
  of 40 million years.
\end{abstract}

\keywords{
  exoplanet evolution (491),
  open star clusters (1160),
	stellar ages (1581)
}

%%%%%%%%%%%%%%%%%%%%%%%%%%%%%%%%%%%%%%%%%%%%%%%%%%%%%%%%%%%%%%%%%%%%%%%%%%%%%%%

\section{Introduction}

The discovery and characterization of planets younger than a billion
years is a major frontier in current exoplanet research.  The reason
is that the properties of young planets provide benchmarks for studies
of planetary evolution.  For instance, young planets can inform our
understanding of when hot Jupiters arrive on their close-in orbits
\citep{dawson_johnson_2018}, how the sizes of planets with massive
gaseous envelopes evolve \citep{rizzuto_tess_2020}, the timescales for
close-in multiplanet systems to fall out of resonance
\citep{izidoro_breaking_2017,arevalo_stability_2022,goldberg_architectures_2022}, and
whether and how mass-loss explains the radius valley
\citep{lopez_how_2012,Owen_Wu_2013,Fulton_et_al_2017,ginzburg_corepowered_2018,lee_primordial_2021}.

The discovery of a young planet requires two claims to be true: the
planet must exist, and its age must be secured.  Spaced-based
photometry from K2 and TESS has yielded a number of exemplars for
which the planetary evidence comes from transits, and the age is based
on either cluster membership
\citep{Mann_et_al_2017,david_four_2019,newton_tess_2019,bouma_cluster_2020,nardiello_pathosII_2020}
or else on correlates of youth such as stellar rotation, photospheric
lithium content, x-ray activity, and emission line strength
\citep{zhou_2021_tois,hedges_toi-2076_2021}.

In this work, we leverage recent analyses of the Gaia data, which have
greatly expanded our knowledge of stellar groups
\citep[{e.g.},][]{CantatGaudin2018a,KounkelCovey2019,Kerr2021}.  To
date these analyses have mostly clustered on stellar positions and 2D
velocities.  One important result has been the discovery of diffuse
streams and tidal tails comparable in stellar mass to the previously
known cores of nearby open clusters
\citep{meingast_psceri_2019,Meingast2021,gagne_number_2021}.  Even
though these streams are spread over tens to hundreds of parsecs,
their velocity dispersions can remain coherent at the $\sim$1\kms\
level.  Internal dynamics and projection effects can also drive them
to be much larger: in the Hyades, stars in the tidal tails are
expected to span up to $\pm 40\mkms$ in velocity relative to the
cluster center \citep{jerabkova_800_2021}.  The stars in such diffuse
regions can be verified to be the same age as the core cluster members
through analyses of color--absolute magnitude diagrams
\citep{KounkelCovey2019}, stellar rotation periods
\citep{curtis_tess_2019,bouma_2021_ngc2516}, and chemical abundances
\citep{hawkins_2020}.  While there are implications for our
understanding of star formation and cluster evolution
\citep{dinnbier_tidal_2020}, a separate consequence is that we now
know the ages of many more stars, including previously known planet
hosts.

The prime Kepler mission \citep{borucki_kepler_2010} found most of the
currently known transiting exoplanets, and it was conducted before
Gaia.  It is therefore sensible to revisit the Kepler field, given our
new knowledge of the stellar ages.

Here, we expand on our earlier study of a $38^{+7}_{-6}$ Myr old
Neptune-sized planet in the Kepler field (Kepler-1627~Ab;
\citealt{bouma_kep1627_2022}).  This planet's age was derived based on
its host star's membership in the $\delta$\ Lyr\ cluster.  While our
analysis of the cluster focused on the immediate vicinity of
Kepler-1627 in order to have a reasonable scope,  it became clear that
the $\delta$\ Lyr\ cluster seems to also be part of a much larger
group of similarly aged stars.  This group, which is at a distance of
$\sim$330\,pc from the Sun, spans Cepheus to Hercules (galactic
longitudes, $l$, between 40$^\circ$ and 100$^\circ$), at galactic
latitudes between 0$^\circ$ and 20$^\circ$.  We therefore refer to it
as the Cep-Her complex.  It exhibits significant sub-structure over
its $\approx$250 parsec length, and a detailed analysis of its
memberships, kinematics, and possible origin is currently being
prepared as part of the 1\,kpc expansion of the SPYGLASS project
(R.~Kerr et al.\ in prep).

Our focus is on the intersection of the Cep-Her complex with the
Kepler field.  Cross-matching our Cep-Her members against known Kepler
Objects of Interest (KOIs; \citealt{thompson_planetary_2018}) yielded
four candidate cluster members: Kepler-1627, Kepler-1643, KOI-7368,
and KOI-7913.  Given our earlier analysis of Kepler-1627, we focus
here on the latter three objects.  After analyzing the relevant
properties of Cep-Her (Section~\ref{sec:cluster}), we derive the
stellar properties (Section~\ref{sec:stars}) and validate the
planetary nature of each system using a combination of the Kepler
photometry and high-resolution imaging (Section~\ref{sec:planets}).
We conclude with a discussion of mini-Neptune size evolution, and
point out possible directions for future work
(Section~\ref{sec:disc_conc}).

\section{The Cep-Her Complex}
\label{sec:cluster}

\begin{figure*}[t]
	\begin{center}
		\leavevmode
		\includegraphics[width=0.99\textwidth]{f1.pdf}
	\end{center}
	\vspace{-0.7cm}
	\caption{
  {\bf Positions and velocities of candidate members of the Cep-Her
  complex.}
  {\it Top row}: On-sky positions in galactic coordinates.  Black
  points are stars for which group membership is more secure than for
  gray points.  Kepler-1627 is in the outskirts of the $\delta$ Lyr
  cluster \citep{bouma_kep1627_2022}, which is centered at $(l,b)
  \approx (66^\circ, 12^\circ)$.
  The Kepler footprint is shown in gray.
  {\it Middle row}: Galactic positions.  The Sun is at $(X, Y, Z) =
  (0, 0, 20.8)$\,pc; lines of constant heliocentric distance are
  shown between 250 and 400\,pc, spaced by 50\,pc.
  {\it Bottom row}: Galactic tangential velocities (left) and
  galactic longitudinal velocity versus galactic longitude (right).
  The gray band in the lower-right shows the $\pm$1-$\sigma$
  projection of the Solar velocity with respect to the local standard
  of rest.  There is a strong spatial and kinematic overlap between
  Kepler-1643 and RSG-5 (magenta).  The local population
  of candidate young stars around KOI-7368 and KOI-7913 is more
  diffuse -- we call this region ``CH-2'' (lime-green).
  The selection method for all of the stars displayed
  is described in Section~\ref{subsec:members}.
	\label{fig:XYZvtang}
	}
\end{figure*}

\subsection{Previous Related Work}

Our focus is on a region of the Galaxy 200 to 500\,pc
from the Sun, above the galactic plane, and spanning galactic
longitudes of $40^\circ$ to $100^\circ$.  Two rich clusters in
this region are the $\delta$~Lyra cluster
\citep{stephenson_possible_1959} and RSG-5 \citep{roser_nine_2016}.
Each of these clusters was known before Gaia.  Their reported ages are
between 30 and 60\,Myr.  Early empirical evidence that these two
clusters could be part of a large and more diffuse population was
apparent in the Gaia-based photometric analysis of pre-main-sequence
stars by \citet[][see their Figures~11 and~13]{Zari2018}.  Further
kinematic connections and complexity were highlighted by
\citet{KounkelCovey2019}, who included these previously known groups
in the larger structures dubbed ``Theia~73'' and
``Theia~96''\footnote{See their visualization online at
\url{http://mkounkel.com/mw3d/mw2d.html} (accessed 15 March 2022)}.
The connection made by \citet{KounkelCovey2019} between the previously
known open clusters and the other groups in the region was made as
part of an unsupervised clustering analysis of the Gaia DR2 positions
and on-sky velocities with a subsequent manual ``stitching'' step.
Their results support the idea that there is an overdensity of 30 to
60\,Myr old stars in this region of the Galaxy.  \citet{Kerr2021}, in
a volume-limited analysis of the Gaia DR2 point-source catalog out to
one third of a kiloparsec, identified three of the nearest
sub-populations, dubbed ``Cepheus-Cygnus'', ``Lyra'', and
``Cerberus''.  \citet{Kerr2021} reported ages for each of these
subgroups between 30 and 35 Myr.


\subsection{Member Selection}
\label{subsec:members}

The possibility that the $\delta$~Lyr cluster, RSG-5, and the
sub-populations identified by \citet{Kerr2021} share a common origin
has yet to be fully substantiated, but preliminary clustering results
from the 1\,kpc SPYGLASS analysis (R.~Kerr et al.\ in prep) suggest
the presence of contiguous stellar populations connecting each of
these groups.  In this work, our primary interest in the region stems
from the fact that a portion of it was observed by Kepler
(Figure~\ref{fig:XYZvtang}, top panel).  To further explore this
sub-population, we select candidate Cep-Her members through four
steps, the first three being identical to those described in Section~3
of \citet{Kerr2021}.  We briefly summarize them here.

The first step is to select stars that are photometrically distinct
from the field star population based on Gaia EDR3 magnitudes $\{G,
G_{\rm RP}, G_{\rm BP}\}$, parallaxes and auxiliary reddening
estimates \citep{lallement_gaia-2mass_2019}.  This step yielded \npms\
stars with high-quality photometry and astrometry.  These stars are
either pre-main-sequence K and M dwarfs due to their long contraction
timescales, or massive stars near the zero-age main sequence due to
their rapid evolutionary timescales.

The second step is to perform an unsupervised HDBScan clustering on
the photometrically selected population
\citep{campello_hierarchical_2015,mcinnes_hdbscan_2017}.  The
parameters we use in the clustering are $\{ X, Y, Z, c v_b, c v_{l^*}
\} $, where $c$ is the size-velocity corrective factor, which is taken
as $c=6\,{\rm pc / km\,s}^{-1}$ to ensure that the spatial and
velocity scales have identical standard deviations.  Positions are
computed assuming the \texttt{astropy v4.0} coordinate standard
\citep{astropy_2018}, which places the Sun $8122$ pc from the galactic
center, and assumes the solar velocity with respect to the local
standard of rest from \citet{schonrich_local_2010}.  As input
parameters to HDBScan, we set the minimum $\epsilon$ threshold past
which clusters cannot be fragmented as $25$\,pc in physical space, and
$25/c$\,\kms\ in velocity.  The minimum cluster size $N$ is set to 10, as
is $k$, the parameter used to define the ``core distance'' density
metric. 

This unsupervised clustering in this case yielded 8 distinct subgroups.
These groups are then used as the ``seed'' populations for the third
step, which is to search for objects at least as close to the 10$^{\rm
th}$ nearest HDBScan-identified member in space-velocity coordinates.
This third step yields stars that are spatially and kinematically
close to the photometrically young stars, but which cannot be
identified as young based on their positions in the color--absolute
magnitude diagram.

The outcome of the analysis up to the point of the third step is shown
in Figure~\ref{fig:XYZvtang}.  To enable a selection cut that filters
out field-star contaminants, we also compute a weight metric,
$D$, defined such that the group member with the smallest core
distance has $D=1$, the group member with the greatest core
distance has $D=0$, and weights for the other group members are
log-normally distributed between these two extremes.  After applying a
set of quality cuts on the astrometry and
photometry,\footnote{$\varpi/\sigma_\varpi>5$; $G/\sigma_{G}>50$;
$G_{\rm RP}/\sigma_{G_{RP}}>20$; $G_{\rm BP}/\sigma_{G_{BP}}>20$.}
this procedure yields $D \sim \log_{10}\mathcal{N}(-1.55,0.61)$.
To visualize the results, in Figure~\ref{fig:XYZvtang} we show
12{,}436 objects with $D>0.02$ as gray points, and 
4{,}763 objects with $D>0.10$ as black points.  The previously
known $\delta$~Lyr\ cluster is visible at $(l,b)=(68^\circ,15^\circ$)
and $(v_{l'}, v_b)=(-4.5 ,-4 )\mkms$.  RSG-5 is visible at
$(l,b)=(83^\circ,6^\circ)$, $(v_{l'}, v_b)=(5.5 ,-3.5 )\mkms$.  Most
of the other subclusters, including in Cep-Cyg
($l,b=90^\circ,7^\circ$) and Cerberus ($l,b=48^\circ,18^\circ$) are
too small or dispersed to have previously been analyzed in great
detail.

%
% KOI match numbers from calc_CepHer_group_numbers.py
%
Our fourth and final step was to cross-match the candidate Cep-Her
member list against all known Kepler Objects of Interest.  We used the
Cumulative KOI table from the NASA Exoplanet Archive from 27 March
2022, and also compared against the \texttt{q1\_q17\_dr25} table
\citep{thompson_planetary_2018}.  From the candidate members with
weights exceeding 0.02, this yielded 11 known false positives, 6
``confirmed'' planets, and 8 ``candidate'' planets.  Inspection of the
Kepler data validation summaries and Robovetter classifications for
these objects showed whether they were potentially consistent with
being {\it i)} planets, and {\it ii)} $\lesssim 10^8$ years old, based
on the presence of rotational modulation at the expected period and
amplitude \citep[{e.g.},][Figure~9]{rebull_rotation_2020}.  Four
objects remained after this inspection: Kepler-1627, Kepler-1643,
KOI-7368, and KOI-7913.

Figure~\ref{fig:XYZvtang} shows the positions of the KOIs along
various projections.  Kepler-1643 is near the core RSG-5 population
both spatially and kinematically.  KOI-7368 and KOI-7913 are in a
diffuse region $\approx$40\,pc above RSG-5 in $Z$ and $\approx$100\,pc
closer to the Sun in $Y$.  In tangential galactic velocity space,
there is some kinematic overlap between the region the latter two KOIs
are in and the main RSG-5 group.

%
% Numbers are from RSG-5_auto_XYZ_vl_vb_cut.csv,
% CH-2_auto_XYZ_vl_vb_cut.csv
%
We define two sets of stars in the local vicinity of our objects of
interest.  For candidate RSG-5 members, we require:
\begin{align}
  X/{\rm pc} &\in [45, 75] \nonumber \\
  Y/{\rm pc} &\in [320, 350] \nonumber \\
  Z/{\rm pc} &\in [40, 80] \nonumber \\
  v_b/{\rm km\,s^{-1}} &\in [-4, -2.5] \nonumber \\
  v_{l^*}/{\rm km\,s^{-1}} &\in [3.5, 6], \nonumber
\end{align}
though the cluster does have a greater spatial extent toward smaller
$X$ (Figure~\ref{fig:XYZvtang}, middle panels).  For the diffuse stars
near KOI-7368 and KOI-7913, we require
\begin{align}
  X/{\rm pc} &\in [20, 70] \nonumber \\
  Y/{\rm pc} &\in [230, 270] \nonumber \\
  Z/{\rm pc} &\in [75, 105] \nonumber \\
  v_b/{\rm km\,s^{-1}} &\in [-3.5, -1.5] \nonumber \\
  v_{l^*}/{\rm km\,s^{-1}} &\in [2, 6] \nonumber
\end{align}
and we call this latter set of stars ``CH-2'', using the preliminary 
Cep-Her (CH) subgroup identifier from R.~Kerr et al.\ (in prep).
These cuts yield
\nrsgfive\ candidate RSG-5 members, and \nchtwo\ candidate CH-2
members.  These stars are listed in Appendix~\ref{app:members}, as are
the Cep-Her candidates that were observed by Kepler, and all of our
candidate matches against the Kepler Objects of Interest.  An important
consideration, especially for CH-2, is the contamination rate by field
stars.  We assess this in the following section.


\subsection{The Cluster's Age}
\label{sec:clusterage}

\begin{figure*}[tp]
	\begin{center}
		\leavevmode
		\subfloat{
			\includegraphics[width=0.49\textwidth]{f2a.pdf}
			\includegraphics[width=0.49\textwidth]{f2b.pdf}
		}
		
		\vspace{-0.6cm}
		\subfloat{
			\includegraphics[width=0.49\textwidth]{f2c.pdf}
			\includegraphics[width=0.49\textwidth]{f2d.pdf}
		}
	\end{center}
	\vspace{-0.7cm}
	\caption{
		{\bf The stellar groups near KOI-7368, KOI-7913, and Kepler-1643
    are 40 to 50 million years old.} 
    {\it Top row}: Color--absolute magnitude diagram of candidate
    Cep-Her members, plotted over candidate members of the
    $\delta$~Lyr~cluster ($\approx38$\,Myr;
    \citealt{bouma_kep1627_2022}) and the Gaia EDR3 Catalog of Nearby
    Stars (gray background).  The left and right columns shows stars
    in RSG-5 and CH-2, respectively.  The range of colors is truncated
    to emphasize the pre-main-sequence.  Stars that fall far below the
    cluster sequences are field interlopers.  {\it Bottom row}: TESS
    and ZTF-derived stellar rotation periods, with the Pleiades
    ($\approx 112$\,Myr) and Praesepe ($\approx 650$\,Myr) shown for
    reference \citep{rebull_rotation_2016a,douglas_poking_2017}.  The
    detection efficiency for reliable rotation periods falls off
    beyond $\bpmrpo \gtrsim 2.6$.
	\label{fig:age}
	}
\end{figure*}

\subsubsection{Color--Absolute Magnitude Diagram}
\label{sec:camd}

Color--absolute magnitude diagrams (CAMDs) of the candidate RSG-5 and
CH-2 members are shown in the upper row of Figure~\ref{fig:age}.  The
stars from the $\delta$~Lyr cluster are from
\citet{bouma_kep1627_2022}, and the field stars are from the Gaia EDR3
Catalog of Nearby Stars \citep{gaia_gcns_2021}.  To make these
diagrams, we imposed the data filtering criteria from
\citet[][Appendix B]{GaiaCollaboration2018}, which include binaries
while omitting artifacts from for instance low
photometric signal to noise, or a small number of visibility periods.
We then corrected for extinction using the
\citet{lallement_threedimensional_2018} dust
maps and the extinction
coefficients $k_X\equiv A_X/A_0$ from \citet{GaiaCollaboration2018},
assuming that $A_0 = 3.1 E(B-V)$.  This yielded a mean and standard
deviation for the reddening of $E(B-V)=0.036\pm0.002$ for RSG-5, and
$E(B-V)=0.017\pm0.001$ for CH-2.  By way of comparison, in
\citet{bouma_kep1627_2022} the same query for the $\delta$~Lyr cluster
yielded $E(B-V)=0.032\pm0.006$.  Finally, for the plots we set the
color range to best visualize the region of maximal age information
content: the pre-main-sequence.

The CAMDs show that for RSG-5, all but one of the candidate
members are on a tight pre-main-sequence locus.  This implies a false
positive rate of a few percent, at most.  By comparison, our control
sample (the $\delta$~Lyr candidates) has a false positive rate of
$\approx$12\%, based on the number of stars that photometrically
overlap with the field population, while being outliers from the bulk
cluster population.  For CH-2, our membership selection gives 27
objects in the color range displayed, and 23 of them appear to be
consistent with being on the pre-main-sequence.  This implies a false
positive rate in CH-2 of $\approx$15\%.

Figure~\ref{fig:age} also shows that most RSG-5 and CH-2 members
overlap with the $\delta$~Lyr cluster, and that the groups are
therefore roughly the same age.  To quantify this, we use the method
introduced by \citet[][their Section~6.3]{gagne_mutau_2020}.  The idea
is to fit the pre-main-sequence loci of a set of reference clusters,
and to then model the locus of the target cluster as a linear
combination of these reference cluster loci.  For our reference
clusters, we used UCL, IC\,2602, and the Pleiades, with the
memberships reported by \citet{Damiani2019} and
\cite{CantatGaudin2018a} respectively.  We adopted ages of 16\,Myr for
UCL \citep{pecaut_star_2016}, 38\,Myr for IC\,2602
\citep{david_ages_2015,randich_gaiaeso_2018} and 112\,Myr for the
Pleiades \citep{dahm_2015}.  These assumptions and the subsequent
processing steps taken to exclude field stars and binaries were
identical to those described in \citet{bouma_kep1627_2022}.  The mean
and uncertainty of the resulting age posterior are $46^{+9}_{-7}$\,Myr
for RSG-5, and $36^{+10}_{-8}$\,Myr for CH-2.  For comparison, this
procedure yields an age for the $\delta$~Lyr cluster of
$38^{+6}_{-5}$\,Myr.  The older isochronal age of RSG-5 is consistent
with its location relative to the $\delta$~Lyr cluster in the upper
left panel of Figure~\ref{fig:age}.


\subsubsection{Stellar Rotation Periods}
\label{sec:rotation}

An independent way to assess the age of the candidate cluster members
is to measure their stellar rotation periods.  This approach can be
achieved using surveys such as TESS \citep{ricker_transiting_2015} and
the Zwicky Transient Facility (ZTF, \citealt{bellm_zwicky_2019}); it
leverages a storied tradition of measuring rotation periods of stars
in benchmark open clusters \citep[see
{e.g.},][]{skumanich_time_1972,curtis_rup147_2020}.  The TESS data in
our case are especially useful, since they provide 3 to 5 lunar months
of photometry for all of our candidate CH-2 and RSG-5 members.

We selected stars suitable for gyrochronology by requiring $\bpmrpo
\geq 0.5$ to focus on FGKM stars that experience magnetic braking.
For TESS, we also restricted our sample to $G<16$, to ensure the stars
are bright enough to extract usable light curves from the full-frame
images.  The magnitude cut corresponds to $\bpmrpo < 2.6$
($\sim$M3V) at the relevant distances.  These cuts gave 19 stars in
CH-2 and 47 stars in RSG-5.  We extracted light curves from the TESS
images using the \texttt{unpopular} package \citep{hattorio_2021_cpm},
and regressed them against systematics with its causal pixel model.
We measured rotation periods using Lomb-Scargle periodograms and
visually vetted the results using an interactive program that allows
us to switch between TESS Cycles, select particular sectors, flag
stars with multiple periods, and correct half-period harmonics. For
ZTF, we used the same color cut to focus on FGKM stars, but restricted
the sample to $13 < G < 18$ to avoid the saturation limit on the
bright end and ensure sufficient photometric precision at the faint
end. We followed the procedure outlined in \citet{curtis_rup147_2020}:
we downloaded $8'\times8'$ image cutouts, ran aperture photometry for
the target and neighboring stars identified with Gaia, and used them
to define a systematics correction to refine the target light curves. 

The lower panels of Figure~\ref{fig:age} show the results.  In RSG-5,
41/47 stars have rotation periods faster than the Pleiades (87\%).
This numerator omits the two stars with periods $>$$12$\,days visible
in the lower-left panel of Figure~\ref{fig:age}.  The age
interpretation for these latter stars, particularly the $\approx$M2.5
dwarf, is not obvious.  \citet{rebull_usco_2018} for instance have
found numerous M-dwarfs with 10-12 day rotation periods at ages of
USco ($\sim$$8$\,Myr), and some may still exist at ages of LCC
($\sim$$16$\,Myr; L.~Rebull submitted).  Regardless, since
nearly no field star outliers seem to be present on the RSG-5 CAMD,
the fact that we do not detect rotation periods for $\approx$13\% of
stars should perhaps be taken as an indication for the fraction of
stars for which rotation periods might not be detectable, due to
{e.g.}, pole-on stars having lower amplitude starspot modulation.

For CH-2, 13/19 stars have rotation periods that are obviously faster
than their counterparts in the Pleiades (68\%).  4 stars, not included
in the preceding numerator, are M-dwarfs with rotation periods between
10 and 12.5 days.  As just discussed, the age interpretation for these
M-dwarfs is not obvious.  Regardless, the $\approx$15\% false positive
rate for CH-2 determined from the CAMD seems consistent with our
fraction of detected rotation periods; RSG-5 was missing rotation
periods for $\approx$15\% of its candidate members, even though all of
its stars appear photometrically consistent with being on a single
pre-main-sequence locus.

It is challenging to convert these stellar rotation periods to a
precise age estimate, since on the pre-main-sequence the stars are
spinning up due to thermal contraction rather than down due to
magnetized braking.  Regardless, the rotation period distributions of
both CH-2 and RSG-5 seem consistent with other 30\,Myr to 50\,Myr
clusters ({e.g.}, IC\,2602 and IC\,2391;
\citealt{douglas_stephanie_t_2021_5131306}).  They also seem
consistent with the false positive rates estimated from the
color--absolute magnitude diagrams.


\section{The Stars}
\label{sec:stars}

\begin{deluxetable}{lccc}
\tabletypesize{\scriptsize}
\tablecaption{Selected system parameters of Kepler-1643, KOI-7368, and KOI-7913. \label{tab:sysparams}}
\tablenum{1}

\tablehead{
\colhead{Parameter} & \colhead{Value} & \colhead{68\% Confidence Interval} & \colhead{Comment}
}

\startdata
\hline
\multicolumn{4}{c}{\emph{Kepler-1643}} \\
\hline
{\it Stellar parameters:} & & & \\
  Gaia $G$~[mag]                             & $13.836$           & $\pm 0.003$                & A \\
  $T_{\rm eff}$~[K]                          & $4916$             & $\pm 110$                  & B \\
  %$\log g_\star$~[cgs]                      & $4.54$             & $\pm 0.10$                 & D \\
  $\log g_\star$~[cgs]                       & $4.502$            & $\pm 0.035$                & C \\
  $R_\star$~[R$_{\odot}$]                    & $0.855$            & $\pm 0.044$                & C \\
  $M_\star$~[M$_{\odot}$]                    & $0.845$            & $\pm 0.025$                & C \\
  $\rho_\star$~[g~cm$^{-3}$]                 & $1.910$            & $\pm 0.271$                & C \\
  $P_{\rm rot}$~[days]                       & $5.106$            & $\pm 0.044$                & D \\
  Li EW~[m\AA]                               & $126$              & $+8$, $-4$                 & E \\
{\it Transit parameters:} & & & \\
%  $t_0$~[BJD$_{\rm TDB}$]                   & $X$                & $X$                        & D \\
  $P$~[days]                                 & $X$                & $X$                        & D \\
  $R_{\rm p}/R_\star$                        & $0.X$              & $+0.X$, $-0.X$             & D \\
  $b$                                        & $X$                & $X$                        & D \\
%  $a/R_\star$                               & $X$                & $+0.X$, $-0.X$             & D \\
  $R_{\rm p}$~[R$_{\oplus}$]                 & $X$                & $\pm 0.X$                  & D \\
  $t_{14}$~[hours]                           & $X$                & $X$                        & D \\
\hline
\multicolumn{4}{c}{\emph{KOI-7368}} \\
\hline
{\it Stellar parameters:} & & & \\
  Gaia $G$~[mag]                             & $12.831$           & $\pm 0.004$                & A \\
  $T_{\rm eff}$~[K]                          & $5241$             & $\pm 50$                   & F \\
  $\log g_\star$~[cgs]                       & $4.499$            & $\pm 0.030$                & C \\
  $R_\star$~[R$_{\odot}$]                    & $0.876$            & $\pm 0.035$                & C \\
  $M_\star$~[M$_{\odot}$]                    & $0.879$            & $\pm 0.018$                & C \\
  $\rho_\star$~[g~cm$^{-3}$]                 & $1.840$            & $0.225$                    & C \\
  $P_{\rm rot}$~[days]                       & $2.606$            & $0.038$                    & D \\
  Li EW~[m\AA]                               & $X$                & $X$                        & B \\
{\it Transit parameters:} & & & \\
 % $t_0$~[BJD$_{\rm TDB}$]                    & $X$                & $X$                        & D \\
  $P$~[days]                                 & $X$                & $X$                        & D \\
  $R_{\rm p}/R_\star$                        & $0.X$              & $+0.X$, $-0.X$             & D \\
  $b$                                        & $X$                & $X$                        & D \\
%  $a/R_\star$                                & $X$                & $+0.X$, $-0.X$             & D \\
  $R_{\rm p}$~[R$_{\oplus}$]                 & $X$                & $\pm 0.X$                  & D \\
  $t_{14}$~[hours]                           & $X$                & $X$                        & D \\
\hline
\multicolumn{4}{c}{\emph{KOI-7913 A}} \\
\hline
{\it Stellar parameters:} & & & \\
  Gaia $G$~[mag]                             & $14.200$           & $\pm 0.003$                & A \\
  $T_{\rm eff}$~[K]                          & $4324$             & $\pm 70$                   & B \\
  $\log g_\star$~[cgs]                       & $4.523$            & $\pm 0.043$                & C \\
  $R_\star$~[R$_{\odot}$]                    & $0.790$            & $\pm 0.049$                & C \\
  $M_\star$~[M$_{\odot}$]                    & $0.760$            & $\pm 0.025$                & C \\
  $\rho_\star$~[g~cm$^{-3}$]                 & $2.172$            & $\pm 0.379$                & C \\
  $P_{\rm rot}$~[days]                       & $3.387$            & $0.016$                    & D \\
  Li EW~[m\AA]                               & $X$                & $X$                        & B \\
  $\Delta G_{\rm AB}$~[mag]                  & $0.51$             & $0.01$                     & F \\
  Apparent sep.~[au]                   		   & $959.4$            & $1.9$                      & F \\
{\it Transit parameters:} & & & \\
  %$t_0$~[BJD$_{\rm TDB}$]                    & $X$                & $X$                        & D \\
  $P$~[days]                                 & $X$                & $X$                        & D \\
  $R_{\rm p}/R_\star$                        & $0.X$              & $+0.X$, $-0.X$             & D \\
  $b$                                        & $X$                & $X$                        & D \\
%  $a/R_\star$                                & $X$                & $+0.X$, $-0.X$             & D \\
  $R_{\rm p}$~[R$_{\oplus}$]              & $X$                & $\pm 0.X$                  & D \\
  $t_{14}$~[hours]                           & $X$                & $X$                        & D \\
\enddata
\tablecomments{
  (A) \citet{gaia_collaboration_2021_edr3}.
  (B) HIRES SpecMatch-Emp \citep{yee_SM_2017}.
  (C) Cluster isochrone \citep{choi_mesa_2016, bressan_parsec_2012}.
  (D) Kepler light curve.  The full set of transit parameters is given in CITE APPENDIX TABLE.
  (E) HIRES; this work.
  (F) TRES SPC \citep{buchhave_hatp16b_class_2010,2021tsc2.confE.124B}.
  (G) Magnitude difference and physical distance between primary and secondary; from Gaia EDR3.
  (H) HIRES SpecMatch-Synth \citep{petigura_cksi_2017}.
}
%\vspace{-1cm}
\end{deluxetable}


A few salient properties of the Kepler objects in Cep-Her can be
gleaned from Figure~\ref{fig:age}.  The stars span spectral types of
G8V (Kepler-1627) to K6V (KOI-7913 A).  The secondary in the KOI-7913
system has spectral type $\approx$K8V.  And since a star with
Solar mass and metallicity arrives at the zero-age main sequence at
$t\approx40$ Myr \citep{choi_mesa_2016}, these stars are all in the
late stages of their pre-main-sequence contraction.  

The adopted stellar parameters are listed in
Table~\ref{tab:sysparams}.  The stellar surface gravity, radius, mass,
and density are found by interpolating against the MIST isochrones
\citep{choi_mesa_2016}.  The statistical uncertainties from this
technique mostly originate from the parallax uncertainties; the
systematic uncertainties are taken to be the absolute difference
between the PARSEC \citep{bressan_parsec_2012} and MIST isochrones.
Reported uncertainties are a quadrature sum of the statistical and
systematic components. 

To verify these parameters and to analyze youth
proxies such as the Li 6708\,\AA\ doublet and H$\alpha$, we acquired
spectra.  We also acquired high resolution imaging for each system, to
constrain the existence of visual companions, including possible bound
binaries.  The system-by-system details follow, and the relevant
results are also summarized in Table~\ref{tab:sysparams}.

\subsection{Kepler\,1643}

\paragraph{Spectra}
For Kepler-1643, we acquired two iodine-free spectra from Keck/HIRES
on the nights of 2020 Aug 16 and 2021 Oct 25.  The acquisition and
analysis followed the usual techniques of the California Planet Survey
\citep{howard_cps_2010}.  We derived the stellar parameters ($T_{\rm
eff}, \log g, R_\star$) using \texttt{SpecMatch-Emp}
\citep{yee_SM_2017}, which yielded values in $<$$1$-$\sigma$ agreement
with those from the cluster-isochrone method.  This approach also
yielded $[{\rm Fe/H}]=0.13 \pm 0.09$.  Using the broadened synthetic
templates from \texttt{SpecMatch-Synth} \citep{petigura_cksi_2017}, we
found $v\sin i = 9.3 \pm 1.0\,\mkms$.  The systemic radial velocity at
the two epochs was $-9.1 \pm 1.9\,\mkms$ and $-7.8\pm 1.2\,\mkms$
respectively \citep{chubak_2012}.  To infer the equivalent width of
the \ion{Li}{1} 6708\,\AA\ doublet, we followed the procedure
described by \citet{bouma_2021_ngc2516}.  This yielded a strong
detection: ${\rm EW}_{\rm Li} = 130^{+6}_{-5}$\,m\AA, with values
consistent at $<$$1$-$\sigma$ between the two epochs.   The quoted
value does not correct for the \ion{Fe}{1} blend at 6707.44\,\AA.  Given the purported age and
effective temperature of the star, the lithium equivalent width is
somewhat low.  We discuss this in greater depth in
Appendix~\ref{app:spectra}.

\paragraph{High-Resolution Imaging}
We acquired adaptive optics imaging of Kepler-1643 on the night of
2019 June 28 using the NIRC2 imager on Keck-II.  Using the narrow
camera (FOV = 10.2\arcsec), we obtained 4 images in the $K'$ filter
($\lambda = 2.12\,\mu$m) with a total exposure time of 320\,s. 
The images did not show any additional visual companions.
We analyzed the data following \citet{kraus_impact_2016}, and
determined the detection limits by analyzing the residuals after
subtracting an empirical PSF template.  This procedure yielded
contrast limits of $\Delta K' = 4.1$ mag at $\rho = 150$ mas, $\Delta
K' = 5.8$ mag at $\rho = 300$ mas, and $\Delta K' = 8.3$ mag
at $\rho > 1000$ mas.


\subsection{KOI-7368}
\paragraph{Spectra}
For KOI-7368, we acquired a spectrum on 2015 June 1 using the echelle
spectrograph (TRES; \citealt{furesz_tres_2008}) mounted at the
Tillinghast 1.5m at the Fred Lawrence Whipple Observatory.  The
Stellar Parameter Classification pipeline for TRES has been described
by \citet{2021tsc2.confE.124B}.  It is based on the synthetic template
library constructed by \citet{Buchhave2012}.  The resulting stellar
parameters ($T_{\rm eff}, \log g, R_\star$) agreed with those from the
cluster-isochrone method within $1$-$\sigma$, though the effective
temperature was more precise by a factor of three.  Auxiliary
spectroscopic parameters included the metallicity $[{\rm Fe/H}]= -0.02
\pm 0.08$, the equatorial velocity $v\sin i = 20.21 \pm 0.50\,\mkms$,
and the systemic velocity ${\rm RV}_{\rm sys} = -10.9 \pm 0.2\,\mkms$.
The Li 6708\AA\ EW measurement procedure yielded ${\rm EW}_{\rm Li} =
236^{+16}_{-14}$\,m\AA.

\paragraph{High-Resolution Imaging}
We acquired adaptive optics imaging of KOI-7368 on the night of 2019
June 12, again using NIRC2.  The observational configuration and
reduction were identical as for Kepler-1643.  No companions were
detected, and the analysis of the image residuals yielded contrast
limits of $\Delta K' = 5.2$ mag at $\rho = 150$ mas, $\Delta K' = 6.7$
mag at $\rho = 300$ mas, and $\Delta K' = 8.7$ mag at $\rho > 1000$
mas.

\subsection{KOI-7913}

\paragraph{Binarity}
KOI-7913 is a binary.  The north-west primary is
$\approx$0.5 magnitudes brighter than the south-east secondary in
optical passbands.  The two stars are separated in Gaia EDR3 by
$3\farcs5$ on-sky, and have parallaxes consistent within $1$-$\sigma$
(with an average $\varpi=3.66 \pm 0.01$\,mas).  The apparent on-sky
separation is $959 \pm 2$ au.  The Gaia EDR3 proper motions are also
very similar.  Since two stars were resolved in the Kepler Input
Catalog and are roughly one Kepler pixel apart, an accurate crowding
metric has already been applied in the NASA Ames data products to
correct the mean flux level \citep{2017ksci.rept....6M}.  This is
important for deriving accurate transit depths.

\paragraph{Spectra}
We acquired Keck/HIRES spectra for KOI-7913 A on the night of 2021 Nov
13, and KOI-7913 B on the night of 2021 Oct 26.  The
\texttt{SpecMatch-Emp} machinery yielded $T_{\rm eff,A} = 4324 \pm
70\,{\rm K}$, and $T_{\rm eff,B} = 4038 \pm 70\,{\rm K}$.  The
remaining parameters were in agreement with those from the cluster
isochrone.  For the primary, we also found $[{\rm Fe/H}]= -0.06 \pm
0.09$, $v\sin i = 13.3 \pm 1.0\,\mkms$, and ${\rm RV}_{\rm sys} =
-17.8 \pm 1.1\,\mkms$.  For the secondary, these same parameters were
$[{\rm Fe/H}]= -0.01 \pm 0.09$, $v\sin i = 10.7 \pm 1.0\,\mkms$, and
${\rm RV}_{\rm sys} = -18.8 \pm 1.1\,\mkms$.  Neither of the KOI-7913
components shows strong lithium absorption, but this is expected given
their $\approx$K6V and $\approx$K8V spectral types.  They do however
both show H$\alpha$ in emission.  Broadly speaking, both observations
are consistent with a $\approx$40 Myr age for KOI-7913 (see
Appendix~\ref{app:spectra}).


\paragraph{High-Resolution Imaging}
We acquired adaptive optics imaging of KOI-7913 on the night of 2020
Aug 27 using the NIRC2 imager.  The observational configuration and
reduction were identical as before.  The images showed KOI-7913 A,
KOI-7913 B, and an additional faint star $\approx$$0\farcs99$ due East
of KOI-7913 B.  Applying the PSF-fitting routines from
\citet{kraus_impact_2016}, the tertiary star has a separation $\rho =
4397 \pm 3$\,mas from the primary, at a position angle $231.17^\circ
\pm 0.02^\circ$, with $\Delta K' = 6.97 \pm 0.04$.  While it is too
faint to affect the interpretation of the KOI itself, it would be
amusing if the faint neighbor were also comoving and young -- at the
age of the system, it would have a mass between 10 and
15\,M$_{\rm Jup}$.  Additional imaging epochs will tell.




\section{The Planets}
\label{sec:planets}

\begin{figure*}[tp]
	\begin{center}
		\leavevmode
		\subfloat{
			\includegraphics[width=0.85\textwidth]{f3a.pdf}
		}

		\vspace{-0.5cm}
		\subfloat{
			\includegraphics[width=0.33\textwidth]{f3b.pdf}
			\includegraphics[width=0.33\textwidth]{f3c.pdf}
		}

		\vspace{-1.27cm}	
		\subfloat{
			\includegraphics[width=0.33\textwidth]{f3d.pdf}
			\includegraphics[width=0.33\textwidth]{f3e.pdf}
		}
	\end{center}
	\vspace{-0.5cm}
	\caption{
		{\bf Raw and processed light curves for the Kepler Objects of
    Interest in Cep-Her.}   
    {\it Top}: 50 day light curve segment from the 3.9 years of Kepler
    data.  The ordinate shows the \texttt{PDCSAP} median-subtracted
    flux in units of parts-per-thousand ($\times 10^{-3}$).  The
    dominant signal is from starspots; planetary transit times
    are indicated but the individual transits are not visible at this scale.
    {\it Bottom}:
    Phase-folded transits of Kepler-1643, KOI-7913, KOI-7368, and
    Kepler-1627 with stellar variability removed.  The maximum a
    posteriori model is shown with the gray line, and the
    residual after subtracting the transit model is vertically
    displaced.  Windows over 10 hours are shown.  Gray points are
    individual flux measurements; black points are binned to 20 minute
    intervals, and have a representative 1-$\sigma$ error bar in the
    center-right of each panel. 
		\label{fig:planets}
	}
\end{figure*}

\subsection{Kepler Data}

The Kepler space telescope observed Kepler-1643, KOI-7913, and
KOI-7368 at a 30-minute cadence between May 2009 and April 2013.  For
all three systems quarters 1 through 17 were observed with minimal
data gaps.  The top panel of Figure~\ref{fig:planets} shows a 50-day
slice of the \texttt{PDCSAP} light curves for the three new Cep-Her
candidates, along with Kepler-1627.  In \texttt{PDCSAP},
non-astrophysical variability is removed through a cotrending approach
that uses a set of basis vectors derived by applying singular value
decomposition to a set of systematics-dominated light curves
\citep{smith_kepler_PDC_2017}.  In our analysis, we used the
\texttt{PDCSAP} light curves with the default optimal aperture
\citep{smith_finding_2016}.  Cadences with non-zero quality flags were
omitted.  In all cases, the stars are dominated by spot-induced
modulation with peak-to-peak variability between 2\% and 10\%.  These
signals are much larger than the transits, which have depth
$\approx$0.1\%.  To quantify the stellar rotation periods, we
calculated the Lomb-Scargle periodogram for each Kepler quarter
independently.  The resulting means and standard deviations are in
Table~\ref{tab:sysparams}.


\subsection{Transit and Stellar Variability Model}
\label{sec:fitting}

Our goals in fitting the Kepler light curves are twofold.  First, we
want to derive accurate planetary sizes and orbital properties.
Second, we want to remove the spot-induced variability signal to enable a
statistical assessment of the probability that the transit signals are
planetary.

We fitted the data as follows.  Given the transit
ephemeris from \citet{thompson_planetary_2018}, we first trimmed the
light curve to a local window around each transit that spanned three
transit durations before and after each transit midpoint.  The
out-of-transit points in each local window were then fitted with a
fourth-order polynomial, which was divided out from the light curve.
The resulting flattened transits were then fitted with a transit model
that assumed quadratic limb darkening.  The model therefore included 8
free parameters for the transit ($\{P, t_0, \log R_{\rm p}/R_\star, b,
u_1 ,u_2, R_\star, \log g\}$), 2 free parameters for the light curve
normalization and a white noise jitter ($\{\langle f \rangle, \sigma_f
\}$), and 5 fixed parameters for each transit.

We fitted the data using \texttt{exoplanet}
\citep{exoplanet:exoplanet}.  We assumed a Gaussian likelihood, and
sampled using \texttt{PyMC3}'s No-U-Turn Sampler
\citep{hoffman_no-u-turn_2014}, after having initialized to the the
maximum a posteriori (MAP) model.  We used the
\citet{gelman_inference_1992} statistic, $\hat{R}$, as our convergence
diagnostic.  The resulting fits are shown in the lower panels of
Figure~\ref{fig:planets}, and the important derived parameters are in
Table~\ref{tab:sysparams}.  The set of full parameters and their
priors are given in Appendix~\ref{app:transit}.

A potential drawback of our approach is that to remove
the starspot-induced variability, we fixed 5 
parameters per transit to their MAP values.
An alternative could be to fit the
planetary transits simultaneously with the starspot-induced
variability using a quasiperiodic Gaussian process (GP).  We explored
this approach, but found that it often required fine-tuning of the
hyperparameter priors, otherwise the GP would overfit flares,
instrumental noise, and any other variability that was not part of the
transit.  This failure mode is pernicious, in that it yields an
ill-founded sense of confidence in a visually clean fit.  Our model is
simpler, and it has the benefit that the white noise jitter never
trades off with any parameter equivalent to a damping timescale for
the coherence of the GP.  It is also computationally efficient, and it
captures the planetary parameters about which we care the most.


\subsection{Planet Validation}

In the future, it may be possible to obtain independent evidence for
the planetary nature of the Cep-Her planets, for instance by observing
spectroscopic transits.  For now,
it is of interest whether the transit signals might be astrophysical
false positives, or whether they are statistically more likely to be
planetary.  We adopt the Bayesian framework implemented in
\texttt{VESPA} to assess the relevant probabilities
\citep{morton_efficient_2012,vespa_2015}.  Briefly summarized, the
priors in \texttt{VESPA} assume the binary star occurrence rate from
\citet{raghavan_survey_2010}, direction-specific star counts from
\citet{girardi_star_2005}, and planet occurrence rates as described by
\citet[][Section~3.4]{morton_efficient_2012}.  The likelihoods are
then evaluated by forward-modeling a synthetic population of eclipsing
bodies for each astrophysical model class, in which each population
member has a known trapezoidal eclipse depth, total duration, and
ingress duration.  These summary statistics are then compared against
the actual photometric data to evaluate the probabilities of false
positive scenarios such as foreground eclipsing binaries, hierarchical
eclipsing binaries, and background eclipsing binaries.

\paragraph{Kepler-1643}
Kepler-1643~b (KOI-6186.01) was already validated as a transiting
planet by \citet{morton_false_2016}, who found a probability for any
of the aforementioned false positive scenarios of 9$\times$10$^{-6}$.
Repeating the calculation with our own stellar-variability correction
and the new NIRC2 imaging constraints, we find ${\rm FPP} =
6\times10^{-9}$.  Figure~\ref{fig:planets} shows the justification:
the transit is flat and has a high S/N ($\approx$$47$).  The shape is
therefore nearly impossible to reproduce with eclipsing binary models.

Intriguingly, Kepler-1643 failed one of the data validation
centroid shift tests (see the \texttt{q1\_q17\_dr25\_koi} data
release): the angular distance between the target star's catalog
position and the position of the transiting source was measured as
$1\farcs0$ at 4.4-$\sigma$.  The reports show however that two
outlying quarters (2 and 6) drive the offset --- the centroid locations
from the other Kepler quarters are consistent at $\lesssim 0\farcs4$.
This is an instructive exercise in how stellar
variability complicates centroid-based vetting tests.  The 
shifts measured by these tests are determined from the in- and
out-of-transit flux-weighted centroids.  For stars with significant
spot-induced variability there is no static baseline in either the in-
or out-of-transit phases.

\paragraph{KOI-7368}
KOI-7368.01 is listed on the NASA Exoplanet Archive as a ``candidate''
planet.  \citet{morton_false_2016} did not compute a false positive
probability for the system because their default trapezoidal fitting
routine failed, presumably due to the spot-induced variability.  Our
fitting approach rectifies this point, and our new NIRC2 images
revealed no new stellar companions.  Performing the relevant
calculation, we find ${\rm FPP} = 4\times10^{-3}$.  Though not
as convincing as Kepler-1643, this clears the usual threshold
for calling the planet statistically validated
\citep{morton_efficient_2012}.  The S/N of the transit is
$\approx$$32$, which indicates that it is unlikely to be caused by
systematic noise in the light curve (see Figure~\ref{fig:planets}).
The positional probability score also meets the requirements for
transiting sources thought to share positions with their target stars
\citep{2017ksci.rept...16B}. 

It bears mentioning that KOI-7368 shows a centroid shift in the
\texttt{q1\_q17\_dr25\_koi} validation reports, similar to
Kepler-1643.  For KOI-7368, the reported offset is smaller, and less
significant ($0\farcs2$; 3.0-$\sigma$).  Again, the data
validation reports show that the shift is caused by a few outlying
quarters (4, 5, 8, and 12).  Since the remaining data show
consistent scatter in their centroid locations, these
outlying quarters are likely also caused by the stellar variability.  
Our NIRC2 imaging independently shows that there are no neighboring
sources that could cause an offset of the observed amplitude.

\paragraph{KOI-7913}
KOI-7913.01 is also currently listed on the NASA Exoplanet Archive as a
``candidate'' planet.  The \citet{morton_false_2016}
analysis was of Q1-Q17 KOIs from DR24, and therefore spanned KOI-1.01
to KOI-7620.01 (omitting KOI-7913.01).  However the results of the
subsequent DR25 analysis by Morton et al.\ are listed at the NASA
Exoplanet Archive.  The relevant table gives a probability for the
system being an astrophysical false positive of $1.4\times10^{-4}$,
with the most likely false positive scenario being a blended eclipsing
binary.  Repeating the calculation with our new detrending and
NIRC2 contrast curves, we find a similar result: ${\rm FPP} =
1.3\times10^{-4}$.  Though the transit has the lowest S/N of any of
the objects discussed ($\approx$$14$), its lower FPP relative to
KOI-7368 can be understood through its flat-bottomed shape,
combined with its long transit duration relative to most
eclipsing binary models (Figure~\ref{fig:planets}).  The host star
probability usability score \citep{2017ksci.rept...16B} also meets the
usual threshold, and so the planet is statistically validated.  Its
disposition has however previously fluctuated from ``false positive''
to ``candidate'' (see Appendix~\ref{app:koi7913}).  The most likely
explanation is the presence of KOI-7913~B, which is located
$\approx0.9$ Kepler pixels away from Kepler-7913 A.  
While the $\approx$1.5 pixel FWHM of the Kepler pixel response function
implies that there is blending between the two stars,
the target-pixel level data for KOI-7913 B reveals an
entirely different stellar rotation
period (Table~\ref{tab:sysparams}), and no hint of the transit signal.
This implies that KOI-7913 B cannot host the planet.


\section{Discussion \& Conclusion}
\label{sec:disc_conc}

\begin{figure*}[!t]
	\begin{center}
		\leavevmode
		\includegraphics[width=0.9\textwidth]{f4.pdf}
	\end{center}
	\vspace{-0.6cm}
	\caption{
		{\bf Radii, orbital periods, and ages of transiting exoplanets}.
    Planets younger than a gigayear with ages more precise than a
    factor of three are emphasized. The Cep-Her planets are
    Kepler-1643~b ($\square$), KOI-7368~b ($\triangledown$),
    KOI-7913~Ab (X), and Kepler-1627~Ab (+).  Interesting trends in
    the population of planets younger than $10^8$ years old include {\it i)} their
    large sizes and {\it ii)} the lack of hot Jupiters.  The new
    objects of interest in Cep-Her have normal mini-Neptune sizes
    between 2 and 3\,$R_\oplus$, which is a novelty given their ages.
    Parameters are from the NASA Exoplanet Archive (2022 Apr 5).
		\label{fig:rp_period_age}
	}
\end{figure*}


\subsection{Normal-Sized Mini-Neptunes Exist at 40$\,$Myr}
\label{subsec:sizes}

The most significant novelty about the planets in Kepler-1643,
KOI-7368, and KOI-7913 is that their sizes are normal relative to the
known population of mini-Neptunes from Kepler.  At field star ages,
mini-Neptune sizes span 1.8\,$R_\oplus$ to 3.6\,$R_\oplus$, with the
most common size being $\approx 2.4\,R_\oplus$
\citep[][Figure~7]{Fulton_et_al_2017}.  The known planets younger than
$10^8$ years are almost all larger, with sizes between 4 and
10$\,R_\oplus$
\citep{Mann_K2_33b_2016,David_et_al_2016,benatti_possibly_2019,david_four_2019,newton_tess_2019,rizzuto_tess_2020,bouma_cluster_2020,mann_toi1227_2022}.
Figure~\ref{fig:rp_period_age} explores this in more detail by showing
the sizes, orbital periods, and ages of the known transiting planets,
emphasizing planets with precise ages.  The smallest previously known
planets comparable to the new Cep-Her mini-Neptunes are AU~Mic~c
($3.0\pm0.2\,R_\oplus$, see \citealt{martioli_aumicbc_2021} and
\citealt{gilbert_flares_2022}), Kepler-1627~Ab ($3.8\pm0.2\,R_\oplus$;
\citealt{bouma_kep1627_2022}), and AU~Mic~d ($4.2\pm0.2\,R_\oplus$;
\citealt{plavchan_planet_2020}).

The theoretical expectation is that mini-Neptunes with sizes of 2 to
3\,$R_\oplus$ should be common at ages of $10^7$ to $10^8$ years.
This expectation is tied to inferences about the initial distributions
of planetary core mass, core composition, and atmospheric mass
fraction \citep{owen_evaporation_2017}.  The Kelvin-Helmholtz cooling
timescale, which is tied to the entropy of the planetary interior
shortly after disk dispersal, also plays a significant role
\citep{owen_constraining_2020}.  As an example,
\citet{rogers_unveiling_2021} predicted that given a core-mass
distribution peaked at $\approx$$4$\,$M_\oplus$, an ice-poor rock/iron
core composition, and a typical H/He mass fraction of $\approx$4\%,
there should be a maximum in planet occurrence at 2 to 3\,$R_\oplus$
at times between 10 and 100\,Myr.  The models advanced by
\citet{gupta_signatures_2020} and \citet{lee_primordial_2021} agree;
their differences lie in the mechanism for producing the radius
valley. 

Systems such as K2-25, V1298~Tau, HIP-67522, TOI-837, and TOI-1227
have sizes that are anomalously large relative to the predicted peak
in planet occurrence at 2 to 3\,$R_\oplus$.  However, their large
sizes can be accommodated by invoking any of {\it i)} larger core
masses, {\it ii)} more volatile-rich compositions, {\it iii)}  larger
initial atmospheric mass fractions, or {\it iv)} longer thermal
cooling times.  Secure mass measurements would help constrain this
parameter space, but the $\sim$1\,\kms\ spot-induced radial velocity
semi-amplitudes makes measuring the Doppler orbits very difficult.
\citep[][]{cale_diving_2021,zicher_one_2022,klein_one_2022}.
Regardless, the new Kepler-1643, KOI-7368, and KOI-7913 systems do
demonstrate that at least some planets at 40\,Myr have sizes that are
consistent with theoretical expectations for mini-Neptunes.  While
selection effects imposed by spot-induced photometric variability are
a likely explanation for why planets this small have not previously
been identified \citep[{e.g.},][]{zhou_2021_tois}, future work should
quantify this bias more carefully, in order to enable empirical
studies of how the planetary size distribution changes at early times.


\subsection{Is CH-2 real?}
\label{subsec:ch2}

RSG-5, and Kepler-1643's membership inside it, meet typical
expectations for a star claimed to be in an open cluster.  RSG-5 is an
obvious overdensity relative to the local field, and our membership
selection easily produces a clean pre-main-sequence locus in color
versus absolute magnitude (Figure~\ref{fig:age}).  CH-2, and KOI-7913
and KOI-7368's membership inside it, do not meet these expectations in
as obvious a manner.  This is because the CH-2 association is diffuse.

To quantify the density difference between CH-2 and RSG-5, we can
compare the spatial and velocity volumes searched for each group.  For
RSG-5, we drew \nrsgfive\ candidate members from a $30\,{\rm
pc}\times30\,{\rm pc}\times40\,{\rm pc}$ rectangular prism, given a
$1.5 {\rm \mkms} \times 2.5 {\rm \mkms }$ rectangle in apparent
galactic velocity.  For CH-2, our \nchtwo\ candidate members came from
a rectangular prism of dimension $50\,{\rm pc}\times40\,{\rm
pc}\times30\,{\rm pc}$, and a rectangular box of $2 {\rm \mkms} \times
4 {\rm \mkms}$.  If we define the searched volume in units of ${\rm
pc}^3\,{\rm km^2}\,{\rm s^{-2}}$, then the volume ratio of CH-2 to
RSG-5 is 3.5 to 1.  The ratio of number densities (candidate members
per unit searched volume) in RSG-5 relative to CH-2 is 16 to 1.

Given its low density, in CH-2 really a star cluster?  For this
discussion, we adopt the definition that a star cluster is a group of
at least 12 stars that was physically associated at its time of
formation.  The ``12'' is set to distinguish star clusters from
high-order multiples \citep[see][]{krumholz_star_2019}.  We explicitly
do not require a ``star cluster'' to be gravitationally bound:
dissolved clusters as well as their tidal tails are included in our
adopted definition of ``clusters''.  We similarly do not require a
threshold number of stars per unit spatial volume.  The latter point
acknowledges that an important factor in cluster identification is
also density in velocity space.  Perhaps once stellar rotation periods
and chemical abundances reach the same level of ubiquity as stellar
proper motions, they might enable further refinement of our ability to
discover stars that formed as part of the same event.

From a data-driven perspective, demonstrating that a group of stars
was physically associated at its time of formation is challenging.
While some young groups show kinematic evidence for expansion
\citep{kuhn_kinematics_2019}, many, including Sco-Cen, do not
\citep{wright_kinematics_2018}.  This complicates the feasibility of
deriving kinematic ages through traceback, as well as through the
expansion itself \citep[see][]{crundall_chronostar_2019}.  A more
minimal approach is that suggested by \citet{tofflemire_2021}: search
for coeval, phase-space neighbors, measure their ages, and determine
if they share a common age.  This approach can demonstrate whether a
star is currently associated with a set of coeval stars, though it
falls short of determining what the association looked like in the
past.  Our analysis of CH-2 meets the latter standard for the
existence of a $\approx$40\,Myr stellar association.  It would be a
worthy exercise for future work to perform a similar search for coeval
phase-space neighbors on the entire dataset of known exoplanet hosts.
For the time being, we can offer the anecdotal point that in our
experience, most stars do not have dozens of 40\,Myr neighbors within
a local volume of a few \kms\ and tens of parsecs.


\subsection{Future work}

\paragraph{Cep-Her}
Our analysis to date has focused only on portions of Cep-Her that were
observed by Kepler: RSG-5, CH-2, and the $\delta$~Lyr cluster.  In
\citet{bouma_kep1627_2022} as well as this work, we have shown that
these groups share similar ages, and have kinematic correlations that
suggest a common origin.  With that said, the membership and
kinematics of the other Cep-Her groups shown in
Figure~\ref{fig:XYZvtang} deserve independent attention.  An important
aspect of the remaining work will be to acquire radial velocities of a
large subset of the stars, to assess whether the traceback approach
could be applicable.  Wide-field spectroscopic surveys such as LAMOST
\citep{zhao_2012_LAMOST} or SDSS-V \citep{kollmeier_2017} could enable
such analyses, while also providing sensitivity to the Li 6708\,\AA\
line.  If Gaia DR3 provides RVS spectra for the brighter F and G
dwarfs, these would also contain the calcium infrared triplet as
another age indicator.  These indicators, combined with better
kinematics, would help in definitively unraveling the formation
history of the complex.

A number of worthy photometric projects also seem possible given the
new understanding of Cep-Her.  One is asteroseismology of the
$\delta$~Sct stars, using either TESS or Kepler data
\citep{bedding_very_2020}.  For cases in which the modes are resolved,
this might yield age or metallicity estimates for the subgroups
independent of other methods.  Other projects could include a more
comprehensive analysis of the stellar rotation periods,
searches of the Kepler light curves for exocomets
\citep{zieba_transiting_2019}, and searches for missed planets around
the most rapid rotators.

\paragraph{Exoplanet demographics at early times}
Our main motivation for finding new young planets is to help benchmark
models for planetary evolution.  However demographic analyses of the
known planets between $10^7$ and $10^9$ years have so far been rather
limited.  Approximately 40 such planets are now known
(Figure~\ref{fig:age}).  About half come from K2, a quarter from TESS,
and now a quarter from Kepler.

Given the current state of the field, a few reflections regarding
experimental design of a demographic survey focused on planetary
evolution over the first gigayear might be useful.  The first is that
such a project requires a set of target stars with known ages.  A
promising way to compile relevant stars could be to combine automated
spatio-kinematic clustering from Gaia with rotation periods measured
using TESS \citep[see the appendices of][]{bouma_kep1627_2022}.  The
second consideration is that all the known young planets smaller than
$3$\,$R_\oplus$ come from either K2 or Kepler.  Demographic inferences
based on TESS are therefore limited to planetary sizes
$\gtrsim4$\,$R_\oplus$, for planets close-in to their host stars.  It
would be worthwhile to compare the occurrence rates of both types of
planets with those from the main Kepler sample.  One specific question
that seems within reach would be to clarify whether enough young stars
have been searched for the dearth of young hot Jupiters to be
significant.  Since the hot Jupiter occurrence rate is strongly
dependent on stellar mass and metallicity
\citep{petigura_metallicity_2018,petigura_cksX_2022}, particular care
would be needed to select a sample of well-studied FGK dwarfs for the
measurement, likely using stars in Sco OB2, Cep-Her, and Orion.  For
demographic studies focused on how mini-Neptune sizes evolve, the
combined K2 and Kepler dataset would be the better primary source.  



\subsection{Summary}

We have shown that Kepler-1643~b, KOI-7368~b, and KOI-7913~Ab are 40
to 50 million years old, and that each system is most likely
planetary.  The evidence for the planetary interpretation comes from
an application of \texttt{VESPA} to the Kepler data, alongside new
imaging from NIRC2.  The validity of the \texttt{VESPA} framework
rests on the premise that non-astrophysical false positives can be
rejected.  Kepler-1643 and KOI-7368 both pass all but one of the
canonical vetting tests: the check for centroid offsets between the in
and out-of-transit phases.  For both cases, the weak observed shifts
are consistent with being caused by starspot-induced variability in
specific quarters spuriously moving the stellar center-of-light.
Independently, our imaging rules out companion stars with the
brightness and positions that would be needed to otherwise explain the
reported shifts.

Each system has multiple indicators of youth that support the reported
ages.  For Kepler-1643, the strongest youth indicator is its physical
and kinematic association with RSG-5.  Based on the color--absolute
magnitude diagram, we are able to select members of this cluster with
a false positive rate of a few percent (Figure~\ref{fig:age}).
Kepler-1643 is one such member.  While the stellar rotation period
period agrees with this assessment, the star's lithium equivalent
width is marginally low, which might motivate future exploration of
lithium depletion across FGKM stars in RSG-5 (see
Appendix~\ref{app:spectra}).

The spatio-kinematic argument for the youth of KOI-7368 and KOI-7913
is weaker because they are in an association of stars, CH-2, that is
more diffuse.  For KOI-7913, stronger indicators of its age come from
its binary.  Both stellar components in KOI-7913 have isochronal ages
consistent with $40$\,Myr.  Both components also show H$\alpha$ in
emission, which for the $\approx$K6V primary is a strong indicator
that the star is $\lesssim$100$\,$Myr old.  KOI-7368 is slightly more
massive, and the Li 6708\,\AA\ measurement and stellar rotation period
provide independent verification of the star's youth.

The astrophysical implication of these considerations is that planets
$\approx$2 Earth radii in size exist at ages of 40 million years.  It
will be interesting to continue the push down to smaller planetary
sizes at comparable ages -- the planetary detections we have presented
are well above the average detection significance for Kepler planets.
There may still be room at the bottom.



%%%%%%%%%%%%%%%%%%%%%%%%%%%%%%%%%%%%%%%%%%%%%%%%%%%%%%%%%%%%%%%%%%%%%%%%%%%%%%%


%\clearpage
\acknowledgements
\raggedbottom
% Cycle 3 and 4
L.G.B{.} is supported by the Heising-Simons Foundation 51
Pegasi~b Fellowship and the TESS GI Program (NASA grants 80NSSC21K0335
and 80NSSC22K0298).
R.K{.} is supported by the Heising-Simons Foundation.
J.L.C{.} is supported by NSF AST-2009840 and the TESS GI Program (NASA
grant 80NSSC22K0299).
%
% ACKNOWLEDGE PFS / CAMPANAS.
%
%This paper includes data collected by the TESS mission, which are
%publicly available from the Mikulski Archive for Space Telescopes
%(MAST).
%%
%Funding for the TESS mission is provided by NASA's Science Mission
%directorate.
%%
%We thank the TESS architects
%%(G.~Ricker, R.~Vanderspek, D.~Latham, S.~Seager, J.~Jenkins)
%and the many TESS team members for their
%efforts to make the mission a continued success.
%
%
%This study was based in part on observations at Cerro Tololo
%Inter-American Observatory at NSF's NOIRLab (NOIRLab Prop{.} ID
%2020A-0146; 2020B-0029 PI: Bouma), which is managed by the
%Association of Universities for Research in Astronomy (AURA) under a
%cooperative agreement with the National Science Foundation.
%
%
%Finally, this research has made use of the Keck Observatory Archive (KOA),
%which is operated by the W. M. Keck Observatory and the NASA Exoplanet
%Science Institute (NExScI), under contract with the National
%Aeronautics and Space Administration.
%We also thank the Keck Observatory staff for their support of
%HIRES and remote observing.  We recognize the importance that the
%summit of Maunakea has within the indigenous Hawaiian community, and
%are deeply grateful to have the opportunity to conduct observations
%from this mountain.
%
% The Digitized Sky Survey was produced at the Space Telescope Science
% Institute under U.S. Government grant NAG W-2166.
% Figure~\ref{fig:scene} is based on photographic data obtained using
% the Oschin Schmidt Telescope on Palomar Mountain.
%

% %
% This research made use of the NASA Exoplanet Archive, which is
% operated by the California Institute of Technology, under contract
% with the National Aeronautics and Space Administration under the
% Exoplanet Exploration Program.
% %

% Resources supporting this work were provided by the NASA High-End
% Computing (HEC) Program through the NASA Advanced Supercomputing (NAS)
% Division at Ames Research Center for the production of the SPOC data
% products.
%

% A.J.\ and R.B.\ acknowledge support from project IC120009 ``Millennium
% Institute of Astrophysics (MAS)'' of the Millenium Science Initiative,
% Chilean Ministry of Economy. A.J.\ acknowledges additional support
% from FONDECYT project 1171208.  J.I.V\ acknowledges support from
% CONICYT-PFCHA/Doctorado Nacional-21191829.  R.B.\ acknowledges support
% from FONDECYT Post-doctoral Fellowship Project 3180246.
% %
% C.T.\ and C.B\ acknowledge support from Australian Research Council
% grants LE150100087, LE160100014, LE180100165, DP170103491 and
% DP190103688.
% %
% C.Z.\ is supported by a Dunlap Fellowship at the Dunlap Institute for
% Astronomy \& Astrophysics, funded through an endowment established by
% the Dunlap family and the University of Toronto.
% %
% D.D.\ acknowledges support through the TESS Guest Investigator Program
% Grant 80NSSC19K1727.
%
%
%
% %
% Based on observations obtained at the Gemini Observatory, which is
% operated by the Association of Universities for Research in Astronomy,
% Inc., under a cooperative agreement with the NSF on behalf of the
% Gemini partnership: the National Science Foundation (United States),
% National Research Council (Canada), CONICYT (Chile), Ministerio de
% Ciencia, Tecnolog\'{i}a e Innovaci\'{o}n Productiva (Argentina),
% Minist\'{e}rio da Ci\^{e}ncia, Tecnologia e Inova\c{c}\~{a}o (Brazil),
% and Korea Astronomy and Space Science Institute (Republic of Korea).
% %
% Observations in the paper made use of the High-Resolution Imaging
% instrument Zorro at Gemini-South. Zorro was funded by the NASA
% Exoplanet Exploration Program and built at the NASA Ames Research
% Center by Steve B. Howell, Nic Scott, Elliott P. Horch, and Emmett
% Quigley.
% %
% This research has made use of the VizieR catalogue access tool, CDS,
% Strasbourg, France. The original description of the VizieR service was
% published in A\&AS 143, 23.
% %
% This work has made use of data from the European Space Agency (ESA)
% mission {\it Gaia} (\url{https://www.cosmos.esa.int/gaia}), processed
% by the {\it Gaia} Data Processing and Analysis Consortium (DPAC,
% \url{https://www.cosmos.esa.int/web/gaia/dpac/consortium}). Funding
% for the DPAC has been provided by national institutions, in particular
% the institutions participating in the {\it Gaia} Multilateral
% Agreement.
%
% (Some of) The data presented herein were obtained at the W. M. Keck
% Observatory, which is operated as a scientific partnership among the
% California Institute of Technology, the University of California and
% the National Aeronautics and Space Administration. The Observatory was
% made possible by the generous financial support of the W. M. Keck
% Foundation.
% The authors wish to recognize and acknowledge the very significant
% cultural role and reverence that the summit of Maunakea has always had
% within the indigenous Hawaiian community.  We are most fortunate to
% have the opportunity to conduct observations from this mountain.
%
% \newline
%

\software{
  %\texttt{arviz} \citep{arviz_2019},
  %\texttt{altaipony} \citep{ilin_flares_2021},
  %\texttt{astrobase} \citep{bhatti_astrobase_2018},
  %\texttt{astroplan} \citep{astroplan2018},
	%\texttt{AstroImageJ} \citep{collins_astroimagej_2017},
  \texttt{astropy} \citep{astropy_2018},
  \texttt{astroquery} \citep{astroquery_2018},
  %\texttt{BATMAN} \citep{kreidberg_batman_2015},
  %\texttt{ceres} \citep{brahm_2017_ceres},
  %\texttt{cdips-pipeline} \citep{bhatti_cdips-pipeline_2019},
  %\texttt{corner} \citep{corner_2016},
  %\texttt{emcee} \citep{foreman-mackey_emcee_2013},
  \texttt{exoplanet} \citep{exoplanet:exoplanet}, and its
  dependencies \citep{exoplanet:agol20, exoplanet:kipping13, exoplanet:luger18,
   	exoplanet:theano},
	%\texttt{gala} \citep{gala,PriceWhelan_2017_gala_zenodo},
	%\texttt{IDL Astronomy User's Library} \citep{landsman_1995},
  %\texttt{IPython} \citep{perez_2007},
	%\texttt{isochrones} \citep{morton_2015_isochrones},
	%\texttt{lightkurve} \citep{lightkurve_2018},
  %\texttt{matplotlib} \citep{hunter_matplotlib_2007}, 
  %\texttt{MESA} \citep{paxton_modules_2011,paxton_modules_2013,paxton_modules_2015}
  %\texttt{numpy} \citep{walt_numpy_2011}, 
  %\texttt{pandas} \citep{mckinney-proc-scipy-2010},
  %\texttt{pyGAM} \citep{serven_pygam_2018_1476122},
  \texttt{PyMC3} \citep{salvatier_2016_PyMC3},
  %\texttt{radvel} \citep{fulton_radvel_2018},
  %\texttt{scikit-learn} \citep{scikit-learn},
  %\texttt{scipy} \citep{jones_scipy_2001},
  %\texttt{TESS-point}  \citep{burke_2020},
  \texttt{tesscut} \citep{brasseur_astrocut_2019},
  \texttt{unpopular} \citep{hattorio_2021_cpm},
\texttt{VESPA} \citep{morton_efficient_2012,vespa_2015},
  %\texttt{webplotdigitzer} \citep{rohatgi_2019},
  %\texttt{wotan} \citep{hippke_wotan_2019}.
}
\ 

\facilities{
 	{\it Astrometry}:
 	Gaia \citep{gaia_collaboration_gaia_2018,gaia_collaboration_2021_edr3}.
 	{\it Imaging}:
    Second Generation Digitized Sky Survey. %,
    %SOAR~(HRCam; \citealt{tokovinin_ten_2018}).
 	Keck:II~(NIRC2; \url{www2.keck.hawaii.edu/inst/nirc2}).
 	%Gemini:South~(Zorro; \citealt{scott_nessi_2018}.
 	%Gemini:North~(`Alopeke; \citealt{scott_nessi_2018,scott_twin_2021}.
 	{\it Spectroscopy}:
	%CTIO1.5$\,$m~(CHIRON; \citealt{tokovinin_chironfiber_2013}),
  %PFS ({\bf CITE}),
	Tillinghast:1.5m~(TRES; \citealt{furesz_tres_2008}).
  %MPG2.2$\,$m~(FEROS; \citealt{kaufer_commissioning_1999}),
	%AAT~(Veloce; \citealt{gilbert_veloce_2018}).
	%AAT~(HERMES; \citealt{lewis_2002_hermers_2df,sheinis_2015_hermes}),
 	Keck:I~(HIRES; \citealt{vogt_hires_1994}).
 	%VLT:Kueyen~(FLAMES; \citealt{pasquini_2002}).
% 	Euler1.2m~(CORALIE),
% 	ESO:3.6m~(HARPS; \citealt{mayor_setting_2003}).
 	{\it Photometry}:
%	  ASTEP:0.40$\,$m (ASTEP400),
% 	CTIO:1.0m (Y4KCam),
% 	Danish 1.54m Telescope,
%	  El Sauce:0.356$\,$m,
% 	Elizabeth 1.0m at SAAO,
% 	Euler1.2m (EulerCam),
	  Kepler \citep{borucki_kepler_2010},
% 	Magellan:Baade (MagIC),
% 	Max Planck:2.2m	(GROND; \citealt{greiner_grond7-channel_2008})
%   MuSCAT3 \citep{Narita_2020},
% 	NTT,
% 	SOAR (SOI),
 	  TESS \citep{ricker_transiting_2015},
% 	TRAPPIST \citep{jehin_trappist_2011},
% 	VLT:Antu (FORS2).
    ZTF \citep{bellm_zwicky_2019}.
}

\clearpage
\bibliographystyle{yahapj}                            
\bibliography{bibliography} 

\appendix
\section{Candidate Cep-Her Members}
\label{app:members}


\paragraph{Table~2} contains candidate Cep-Her members with weights
$D>0.02$ observed by Kepler.  The final catalog of candidate
Cep-Her members will be provided by R.~Kerr et al.\ in prep; Table~2 is
from a preliminary version of that work.  Note that more
restrictive weight cuts should be imposed should one wish to remove
the majority of field star interlopers; at least half of the stars
included in the table are unlikely to be bonafide Cep-Her members.
Table~2  was created by cross-matching candidate Cep-Her members
(selected using Gaia EDR3; Section~\ref{subsec:members}) against a
Kepler to Gaia DR2 cross-match (the \texttt{gaia-kepler.fun}
crossmatch database created by Megan Bedell).  The
\texttt{kic\_dr2\_ang\_dist} column is from the latter table.  The
EDR3 to DR2 match was performed using the
\texttt{gaiaedr3.dr2\_neighbourhood} table, and the closest proper
motion and epoch-corrected angular distance neighbor was taken as the
single best match.  The \texttt{edr3\_dr2\_mag\_diff} column gives
some indication of the reliability of this EDR3 to DR2 conversion, as
there are a few cases between Gaia DR2 and EDR3 where partially
resolved binaries became fully resolved.

\paragraph{Candidate matches between Cep-Her and the Kepler Objects of
Interest:}
The full list of candidate matches between Cep-Her and the Kepler
objects of interest is as follows -- the objects are listed in order
of descending weights, $D$.
Objects designated as confirmed planets included
Kepler-1627,
Kepler-1643,
Kepler-1331,
Kepler-1062, and
Kepler-1933.  % Newly validated by Valizadegan+2022
Objects designated as candidate planets included
KOI-5264,
KOI-8007,
KOI-7572,
KOI-7375,
KOI-7368,
KOI-7638,
KOI-5632, and
KOI-7913.
Objects designated known false positive planet candidates included
KOI-6437, 
KOI-5988, 
KOI-7871,
KOI-7655,
KOI-5024,
KOI-61,
KOI-4336,
KOI-6812,
KOI-3399, and
KOI-6277.
Finally, Kepler-1902 (KOI-3090) has one confirmed planet
(KOI-3090.02), and one false positive (KOI-3090.01).
Of these objects, only Kepler-1627, Kepler-1643, KOI-7368, and
KOI-7913 met our requirements for potentially
both {\it i)} having real planets, and {\it ii)} being $\lesssim 10^8$
years old, based on the presence of rotational modulation at the
expected period and amplitude.  One object was ambiguous: Kepler-1933.
This system has a confirmed $\approx$$1.4\,R_\oplus$ planet, a stellar
rotation period of 6.5\ days, and an effective temperature of $\approx$$5750\,{\rm K}$.  This places it near the upper envelope of the
rotation period vs.\ color distribution for the Pleiades, making
it unlikely to be $\approx$40\,Myr old.  Nonetheless, we acquired a
reconnaissance HIRES spectrum, and it yielded ${\rm EW}_{\rm Li} = 93 \pm
5$\,m\AA.  Combined with the rotation period, this suggests an age
between 100 and 200\,Myr for Kepler-1933, but the system is
unlikely to be part of Cep-Her.



\paragraph{Table~3} contains spatial, kinematic, astrometric, and
rotation period information for the \nrsgfive\ candidate RSG-5 members
and \nchtwo\ candidate CH-2 members described in
Section~\ref{subsec:members}.  These are the data used to make the
lower panels of Figure~\ref{fig:age}; as with Table~2, these are
from a preliminary version of the SPYGLASS 1\,kpc expansion (R. Kerr et
al.\ in prep). 
% Table~3 includes rotation periods for 53 stars with ZTF and 51 stars
% with TESS, for a total of 71 periods reported for unique stars and 32
% overlapping between the surveys.
We adopted the ZTF period over the
TESS period in three cases: (1) Gaia EDR3 2081755809272821248: the
Lomb-Scargle periodogram favored 6.67 days, consistent with the ZTF
period of 6.61 days; however, we flagged it as a candidate
double-dipper, which appears to have inaccurately doubled the TESS
period to 13.34 days; (2) Gaia EDR3 2081737529891330560: we found 3.06
days with TESS and 6.64 days with ZTF; we suspect that TESS captured
the 1/2-period harmonic and adopt the approximately double value from
ZTF; (3) 2134851775526125696: for this star, we measured 1.91 days
with TESS from Cycle 2, but noted that the signal appeared to be
missing in Cycle 4; ZTF found a strong signal at 12.23 days and we
adopt this as the star's period. In the remaining overlap cases, we
adopted the average between TESS and ZTF as the final period. For
these overlap stars, the median absolute deviation is 0.01 days,
showing remarkable consistency between the surveys. For three stars,
we failed to detect a period in TESS but recovered one from ZTF; in
all cases the periods appear to be 13--16 days.  These stars were: (1)
Gaia EDR3 2129930258400157440, for which TESS showed a flat light
curve while ZTF yielded a 15.3-day period; (2) Gaia EDR3
2082376861542398336, LS found a 7.6-day period which we rejected
during visual validation; we found 15.4 days with ZTF, and we suspect
that the weak/rejected signal form TESS might have been a 1/2 period
harmonic; (3) Gaia EDR3 2082397099429013120, similar to the previous
case, we rejected a 6.7-day signal from TESS and recovered a 12.8-day
period with ZTF. 


\section{Spectroscopic Youth Indicators}
\label{app:spectra}

\begin{figure*}[t]
	\begin{center}
		\leavevmode
			\includegraphics[width=0.99\textwidth]{f5.pdf}
	\end{center}
	\vspace{-0.3cm}
	\caption{
    {\bf Spectroscopic youth diagnostics for Kepler-1627, KOI-7368,
    Kepler-1643, and KOI-7913 AB. }
    The spectra are shown in the observed frame, and the stars are
    sorted left-to-right in order of decreasing effective temperature.
    \label{fig:koiyouthindicators}
	}
\end{figure*}

Figure~\ref{fig:koiyouthindicators} shows key portions of the HIRES
and TRES spectra for the Kepler objects in Cep-Her.  Lithium
absorption is obvious at 6708\AA\ in all stars except KOI-7913 B,
where it is marginal.  H$\alpha$ is in emission for both components of
KOI-7913, and in absorption for the hotter stars.  In the following,
we compare these observations against stars in benchmark open
clusters.

\subsection{Lithium}

\begin{figure*}[tp]
	\begin{center}
		\leavevmode
		\subfloat{
			\includegraphics[width=0.49\textwidth]{f6a.pdf}
			\includegraphics[width=0.49\textwidth]{f6b.pdf}
		}

		\subfloat{
			\includegraphics[width=0.49\textwidth]{f6c.pdf}
		}
	\end{center}
	\vspace{-0.3cm}
	\caption{
    {\bf Lithium 6708\AA\ and H$\alpha$ equivalent widths for the
    objects of interest compared to young open clusters and field
    stars. } Positive equivalent width means absorption; negative
    equivalent width means emission.  The statistical uncertainties on
    the newly measured equivalent widths are shown, or else are
    smaller than the markers.
    {\it Top}:
    The field stars are stars with Kepler planets from
    \citet{berger_identifying_2018}.  The ``40-50 Myr'' reference
    stars (left) are from IC\,2602 \citep{randich_gaiaeso_2018} and
    Tuc-Hor \citep{kraus_stellar_2014}.  The ``112 Myr'' stars (right)
    are from the Pleiades
    \citep{soderblom_evolution_1993,jones_evolution_1996,bouvier_pleiades_lirot_2018}.
    {\it Bottom}:
    The H$\alpha$ comparison is against Tuc-Hor
    \citep[$\approx$$40$\,Myr;][]{kraus_stellar_2014}
    \label{fig:lithium}
	}
\end{figure*}

Figure~\ref{fig:lithium} compares the measured lithium equivalent
widths of the Kepler objects against a few reference populations.  We
selected studies from the literature only where upper limits were
explicitly reported.  KOI-7368 and KOI-7913~A have secure lithium
detections, while for KOI-7913~B the detection is marginal (${\rm EW}_{\rm
Li} = 42^{+12}_{-19}$\,m\AA).  For all three stars, as well as for
Kepler-1627~A, the observed lithium equivalent width is consistent
with the stellar effective temperatures and $\approx40$\,Myr ages.

Kepler-1643, in RSG-5, is conspicuously below the 40-50$\,$Myr
sequence in the left-panel of Figure~\ref{fig:lithium}, while still
being above the field stars (${\rm EW}_{\rm Li} = 130^{+6}_{-5}$\,m\AA).
The right panel shows the comparison against the Pleiades, where
Kepler-1643 is more consistent with the observed dispersion in
lithium.

One explanation for this could be that Kepler-1643 is a field
interloper; another could be that RSG-5 is much older than 50\,Myr.  We do
not favor either, given that {\it i)} the star was selected based
on its spatio-kinematic proximity to other RSG-5 members, nearly all
of which appear consistent with being $\approx$50\,Myr old
(Figure~\ref{fig:age} upper left), and {\it ii)} the same RSG-5
members display a gyrochronal sequence consistent with being younger
than the Pleaides (Figure~\ref{fig:age} lower left).  
There is a $\approx$$1\%$ chance of being a field interloper based on
the spatio-kinematic selection, and a similar independent chance
($\approx$$1\%$) of a field K2V star having a rotation period below the
Pleiades \citep{mcquillan_rotation_2014}.
The extra observation of a strong lithium detection at these
effective temperatures yields a very small probability for Kepler-1643 
being a field interloper.
It also seems unlikely that RSG-5 could be much older than 50\,Myr,
based on its proximity to the $\delta$~Lyr cluster and IC\,2602 in the
CAMD, as well as the data shown in rotation versus color diagram.

Our preferred explanation for Kepler-1643's lithium equivalent
width is tied to the reference sample of IC\,2602 and Tuc-Hor
stars not fully exploring the possible stellar rotation
periods and lithium equivalent widths. 
Analyzing Figure~\ref{fig:lithium}, it is remarkable that in
just 50 million years, K-dwarfs between $4500$\,K and $5200$\,K go from
having a tight lithium sequence to one with a dispersion
$\approx10\times$ greater.  The existence of the Li dispersion in
Pleiades-age K-dwarfs has been known for decades; it has also been
known that the stars with the largest lithium abundances are also the
most rapidly rotating
\citep{butler_pleiades_1987,soderblom_evolution_1993}.  More recent
analyses of this correlation have been reviewed by
\citet{bouvier_lithium-rotation_2020}.  The conclusion of that work
was that the origin of the rotation-lithium correlation likely lies
within pre-main-sequence stellar physics.  If so, one would expect the
IC\,2602 and Tuc-Hor K-dwarfs to show a larger intrinsic lithium
dispersion.  A recent analysis of the $\approx$40\,Myr
NGC~2547 by \citet{binks_2022} suggests that this may
be the case, though it only had $\approx$10 stars in the relevant
effective temperature range.
This suggests that
RSG-5 and Cep-Her could be
worthy objects for a closer analysis of the lithium-rotation correlation
near the zero-age main sequence.

\subsection{H$\alpha$}

As shown in Figure~\ref{fig:koiyouthindicators}, H$\alpha$ is in
emission for both components of KOI-7913, and in absorption for the
hotter stars.  Additionally, the emission appears double-peaked for
both of the KOI-7913 components.  An important note is that KOI-7913~A and
KOI-7913~B were spatially resolved from each other during data
acquisition.  Performing a cross-correlation between each of the stars
and the nearest matches in the Keck/HIRES template library, we also
found that the CCFs for both components of KOI-7913 showed no
indications of double-lined binarity \citep{kolbl_detection_2015}.

Balmer line emission, particularly for H$\alpha$, is expected for
stars of this age.  \citet{kraus_stellar_2014} for instance, in their
survey of Tuc-Hor ($\approx$40\,Myr), observed that all
cluster members with spectral type $>$${\rm K4.5V}$ had H$\alpha$ in
emission.  The agrees with our observations that KOI-7913 shows
H$\alpha$ in emission for both components, while it is in absorption
for all of our other Kepler objects.  The double-peaked nature of the
emission, though not always present, is also common for
active stars.  Proxima Centauri, for instance, has double-peaked
H$\alpha$ emission \citep{collins_calculations_2017}.  A simple
explanation is self-absorption: photons nearest the center of the line
see a greater optical depth from higher layers of the chromosphere,
while photons on the wings are on average too far from the
rest-wavelength to excite electrons and be re-absorbed in the upper
layers.  The exact details of when a star's atmosphere reaches the
conditions for such self-absorption require non-local thermal
equilibrium models of the chromosphere
\citep{short_chromospheric_1998,2005A&A...439.1137F}.


\section{Table of Transit Fit Parameters}
\label{app:transit}

Table~4 gives the full set of fitted and derived
parameters from the model described in Section~\ref{sec:fitting}.
Priors and convergence statistics are also listed.


\section{Disposition History of KOI-7913}
\label{app:koi7913}

The disposition of KOI-7913.01 has been debated: in
\texttt{q1\_q17\_dr25\_koi} the source was flagged as a false
positive, with the comment ``cent\_kic\_pos---halo\_ghost''.  This
comment and disposition were removed in the
\texttt{q1\_q17\_dr25\_sup\_koi} data release, which renamed the
planet a ``candidate''.  In this note, we discuss the interpretation
of these flags (which do not apply to the system, according to the
latest analysis).  We also discuss how the relative on-sky positions of
KOI-7913~A and KOI-7913~B affect the interpretation of the Kepler
data.

As described by \citet{thompson_planetary_2018}, the
``cent\_kic\_pos'' flag is an indication that the measured source
centroid is offset from its expected location in the Kepler Input
Catalog.  The final Kepler data validation reports, generated 2016 Jan
30, do not show this to be the case for KOI-7913.  Moreover, the
statistical significance of any centroid offset is lower than for
KOI-7368 and Kepler-1643 (which both show centroid offsets that are
likely explained by the stellar variability).

What of the ``halo\_ghost'' flag?  This test measures the transit
strength for the pixels inside the aperture, and compares it to that
measured in the ring of pixels around said aperture (the ``halo'').
One usually expects the transit signal to be strongest in the
central aperture, rather than the halo.  Two types of false positive
scenarios can change this and trigger the flag: the first is when
optical ghosts from bright eclipsing binaries reflect off the CCD, and
contaminate the target star.  The second is when the PRF of nearby
stars directly overlaps with the PRF of the target star (see
\citealt{thompson_planetary_2018}, Section A.5.2).  The most obvious
explanation for KOI-7913 is the latter case, given that KOI-7913 B is
$\approx0.9$ Kepler pixels away from Kepler-7913 A and so it usually
part of the ``halo''.  Due to the on-sky orientation of KOI-7913 A and
KOI-7913 B, the default ``optimal aperture'' selected in quarters 3,
7, 11, and 15 in fact included both stars, while for the remaining
quarters KOI-7913 B was excluded from the optimal aperture but was
included as part of the halo (see pages 35 through 71 of the data
validation reports.)  

Given the orientation of the stars and the $\approx$1.5 pixel FWHM of
the Kepler pixel response function, some blending between the two
stars is present.  The pointing geometries from quarters 3, 7, 11, and
15 however did not affect the observed transit depths,
which is an indication that the crowding metric applied in the data
products accurately correct the mean flux level
\citep{2017ksci.rept....6M}.  Analysis of the target-pixel data that
was separately acquired for KOI-7913~B also reveals a
different stellar rotation period, and no hint of the transit signal.

\clearpage
%% \begin{deluxetable}{} command tell LaTeX how many columns
%% there are and how to align them.
%\startlongtable
\begin{deluxetable*}{lll}
    
%% Keep a portrait orientation

%% Over-ride the default font size
%% Use Default (12pt)
%\tabletypesize{\scriptsize}
\tabletypesize{\small}

%% Use \tablewidth{?pt} to over-ride the default table width.
%% If you are unhappy with the default look at the end of the
%% *.log file to see what the default was set at before adjusting
%% this value.

%% This is the title of the table.
\tablecaption{Candidate Cep-Her members observed by Kepler}
\label{tab:cepherkepler}
%% This command over-rides LaTeX's natural table count
%% and replaces it with this number.  LaTeX will increment 
%% all other tables after this table based on this number
%\tablenum{3}

%% The \tablehead gives provides the column headers.  It
%% is currently set up so that the column labels are on the
%% top line and the units surrounded by ()s are in the 
%% bottom line.  You may add more header information by writing
%% another line between these lines. For each column that requries
%% extra information be sure to include a \colhead{text} command
%% and remember to end any extra lines with \\ and include the 
%% correct number of &s.
\tablehead{
  \colhead{Parameter} &
  \colhead{Example Value} &
  \colhead{Description}
}

%% All data must appear between the \startdata and \enddata commands
%
% paste from
% /Users/luke/Dropbox/proj/earhart/results/tables/NGC_2516_Prot_cleaned_header.tex
% via drivers/write_NGC2516_main_table.py
\startdata
\texttt{dr2\_source\_id}      & 2073765172933035008 & Gaia DR2 source identifier. \\
\texttt{dr3\_source\_id}      & 2073765172933035008 & Gaia (E)DR3 source identifier. \\
\texttt{kepid} & 5641711      & KIC identifier. \\
\texttt{ra} &    297.4099     & Gaia EDR3 right ascension [deg]. \\
\texttt{dec} &   40.8972      & Gaia EDR3 declination [deg]. \\
  \texttt{weight} & 0.0409    & Strength of the connectivity to other candidate cluster members. \\
  \texttt{v\_l} & -0.5061     & Longitudinal galactic velocity [km\,s$^{-1}$]. \\
  \texttt{v\_b} & -8.2328     & Latitudinal galactic velocity [km\,s$^{-1}$]. \\
  \texttt{x\_pc} & -8035.42&  & Galactocentric $X$ position coordinate [pc]. \\
  \texttt{y\_pc} & 331.44     & Galactocentric $Y$ position coordinate [pc]. \\
  \texttt{z\_pc} & 65.32      & Galactocentric $Z$ position coordinate [pc]. \\
  \texttt{kic\_dr2\_ang\_dist} & 0.298 & Separation between KIC and Gaia DR2 positions [arcsec]. \\
  \texttt{edr3\_dr2\_mag\_diff} & 0.002 & $G$-band difference between EDR3 and DR2 source match [mag]. \\
\enddata

%% Include any \tablenotetext{key}{text}, \tablerefs{ref list},
%% or \tablecomments{text} between the \enddata and 
%% \end{deluxetable} commands

%% General table comment marker
\tablecomments{
Table~\ref{tab:cepherkepler} is published in its entirety in a machine-readable
format.  One entry is shown for guidance regarding form and content.
This involved a conversion based 

is a concatenation of all candidate
NGC\,2516 members reported by \citetalias{cantatgaudin_gaia_2018},
\citetalias{kounkel_untangling_2019}, and \citetalias{meingast_2021}
based on the Gaia DR2 data.  Different levels of purity and
completeness can be achieved using different cuts on photometric
periods, periodogram powers, and lithium eqiuvalent widths.  Sets
$\mathcal{A}$ and $\mathcal{B}$ provide two possible levels of
cleaning (see Section~\ref{subsubsec:cluster}).  When the
target star is the only star present in the TESS aperture,
\texttt{nequal}$=0$, \texttt{nclose}$=1$, and \texttt{nfaint}$=1$.
The light curves are available at
\url{https://archive.stsci.edu/hlsp/cdips}.
Supplementary plots enabling analyses of individual stars
are available at \url{https://lgbouma.com/notes}.
}
\vspace{-0.5cm}
\end{deluxetable*}

\vspace{-6cm}
%% \begin{deluxetable}{} command tell LaTeX how many columns
%% there are and how to align them.
%\startlongtable
\begin{deluxetable}{lll}
    
%% Keep a portrait orientation

%% Over-ride the default font size
%% Use Default (12pt)
%\tabletypesize{\scriptsize}
\tabletypesize{\footnotesize}

%% Use \tablewidth{?pt} to over-ride the default table width.
%% If you are unhappy with the default look at the end of the
%% *.log file to see what the default was set at before adjusting
%% this value.

%% This is the title of the table.
\tablecaption{Rotation periods and kinematics for candidate RSG-5 and
  CH-2 members.}
\label{tab:rot}
%% This command over-rides LaTeX's natural table count
%% and replaces it with this number.  LaTeX will increment 
%% all other tables after this table based on this number
\tablenum{3}

%% The \tablehead gives provides the column headers.  It
%% is currently set up so that the column labels are on the
%% top line and the units surrounded by ()s are in the 
%% bottom line.  You may add more header information by writing
%% another line between these lines. For each column that requries
%% extra information be sure to include a \colhead{text} command
%% and remember to end any extra lines with \\ and include the 
%% correct number of &s.
\tablehead{
  \colhead{Parameter} &
  \colhead{Example Value} &
  \colhead{Description}
}
%% All data must appear between the \startdata and \enddata commands
%
\startdata
\texttt{dr3\_source\_id}      & 2127562009133684480 & Gaia (E)DR3 source identifier. \\
\texttt{ra} &    291.02306     & Gaia EDR3 right ascension [deg]. \\
\texttt{dec} &  46.43843      & Gaia EDR3 declination [deg]. \\
\texttt{parallax} &  3.7099      & Gaia EDR3 parallax [milliarcsec]. \\
\texttt{ruwe} &  0.981      & Gaia EDR3 renormalized unit weight error. \\
\texttt{weight} & 0.087    & Strength of connectivity to other candidate cluster members. \\
\texttt{v\_l} & 2.78     & Longitudinal galactic velocity, $v_{l^*}$ [km\,s$^{-1}$]. \\
\texttt{v\_b} & -2.87     & Latitudinal galactic velocity [km\,s$^{-1}$]. \\
\texttt{x\_pc} & -8068.5  & Galactocentric $X$ position coordinate [pc]. \\
\texttt{y\_pc} & 256.0     & Galactocentric $Y$ position coordinate [pc]. \\
\texttt{z\_pc} & 86.3      & Galactocentric $Z$ position coordinate [pc]. \\
\texttt{(BP-RP)0} & -0.115 &  Gaia $G_\mathrm{BP}$-$G_\mathrm{RP}$   color, minus $E$($G_\mathrm{BP}$-$G_\mathrm{RP}$). \\
\texttt{(M\_G)0} & 0.442 & Absolute $G$-band magnitude, corrected for extinction. \\
\texttt{cluster} & CH-2 & RSG-5 or CH-2. \\
\texttt{Prot\_Adopted} & NaN & Adopted rotation period if available, else NaN [days]. \\
\texttt{Prot\_TESS} & NaN & TESS rotation period if available, else NaN [days]. \\
\texttt{Prot\_ZTF}  & NaN & ZTF rotation period if available, else NaN [days]. \\
\texttt{Prot\_Confused} & NaN & Boolean flag; true when stars are photometrically blended. \\
\enddata
%% General table comment marker
\tablecomments{Table~3 is published in its entirety in a machine-readable
format.  One entry is shown for guidance regarding form and content.
}
\vspace{-0.5cm}
\end{deluxetable}

\begin{deluxetable*}{lllrrrrrrr}
	%
	\tablecaption{ Priors and posteriors for the transit models with local
  polynomials removed.}
	\label{tab:koifull}
	%
	\tabletypesize{\scriptsize}
	%\tabletypesize{\small}
	%
	\tablenum{4}
	%
	\tablehead{
		\colhead{Param.} & 
		\colhead{Unit} &
		\colhead{Prior} & 
		\colhead{Median} & 
		\colhead{Mean} & 
		\colhead{Std{.} Dev.} &
		\colhead{3\% HDI} &
		\colhead{97\% HDI} &
		\colhead{ESS} &
		\colhead{$\hat{R}-1$}
	}
  %
	\startdata
\hline
\multicolumn{10}{c}{\emph{Kepler-1643}} \\
\hline
$P$ & d & $\mathcal{N}(5.34264; 0.01000)$ & 5.3426257 & 5.3426258 & 0.0000101 & 5.3426071 & 5.3426454 & 7884 & 1.1e-03 \\
$t_0^{(1)}$ & d & $\mathcal{N}(134.38; 0.02)$ & 134.3820 & 134.3820 & 0.0011 & 134.3799 & 134.3841 & 7390 & 3.7e-04 \\
$\log R_{\rm p}/R_\star$ & -- & $\mathcal{U}(-6.215; 0.000)$ & -3.688 & -3.689 & 0.021 & -3.728 & -3.653 & 4449 & -7.8e-05 \\
$b$ & -- & $\mathcal{U}(0; 1+R_{\mathrm{p}}/R_\star)$ & 0.583 & 0.578 & 0.051 & 0.485 & 0.673 & 4705 & 1.9e-04 \\
$u_1$ & -- & \citet{exoplanet:kipping13} & 0.26 & 0.29 & 0.21 & 0.00 & 0.68 & 5324 & 7.9e-04 \\
$u_2$ & -- & \citet{exoplanet:kipping13} & 0.32 & 0.31 & 0.32 & -0.26 & 0.88 & 4908 & 8.4e-04 \\
$R_\star$ & $R_\odot$ & $\mathcal{N}(0.855; 0.044)$ & 0.851 & 0.851 & 0.045 & 0.766 & 0.933 & 7473 & 7.2e-04 \\
$\log g$ & cgs & $\mathcal{N}(4.502; 0.035)$ & 4.507 & 4.507 & 0.035 & 4.442 & 4.576 & 6530 & -1.4e-04 \\
$\log \sigma_f$ & -- & $\mathcal{N}(\log\langle \sigma_f \rangle; 2.000)$ & -8.520 & -8.520 & 0.019 & -8.556 & -8.486 & 7966 & 2.1e-04 \\
$\langle f \rangle$ & -- & $\mathcal{N}(1.000; 0.100)$ & 1.000 & 1.000 & 0.000 & 1.000 & 1.000 & 7488 & 3.2e-04 \\
$R_{\rm p}/R_\star$ & -- & -- & 0.025 & 0.025 & 0.001 & 0.024 & 0.026 & 4449 & -7.8e-05 \\
$\rho_\star$ & g$\ $cm$^{-3}$ & -- & 1.94 & 1.95 & 0.19 & 1.60 & 2.31 & 6081 & 9.4e-05 \\
$R_{\rm p}$ & $R_{\mathrm{Jup}}$ & -- & 0.207 & 0.207 & 0.012 & 0.184 & 0.231 & 6326 & 2.5e-04 \\
$R_{\rm p}$ & $R_{\mathrm{Earth}}$ & -- & 2.32 & 2.32 & 0.13 & 2.06 & 2.59 & 6326 & 2.5e-04 \\
$a/R_\star$ & -- & -- & 14.31 & 14.32 & 0.47 & 13.49 & 15.23 & 6081 & 8.2e-05 \\
$\cos i$ & -- & -- & 0.041 & 0.040 & 0.005 & 0.032 & 0.049 & 4929 & 2.4e-04 \\
$T_{14}$ & hr & -- & 2.41 & 2.41 & 0.06 & 2.30 & 2.53 & 4774 & 5.3e-04 \\
$T_{13}$ & hr & -- & 2.23 & 2.23 & 0.07 & 2.11 & 2.36 & 4561 & 6.2e-04 \\
\hline
\multicolumn{10}{c}{\emph{KOI-7368}} \\
\hline
$P$ & d & $\mathcal{N}(6.84294; 0.01000)$ & 6.8430344 & 6.8430341 & 0.0000125 & 6.8430107 & 6.8430574 & 10045 & 6.5e-05 \\
$t_0^{(1)}$ & d & $\mathcal{N}(137.06; 0.02)$ & 137.0463 & 137.0463 & 0.0014 & 137.0437 & 137.0489 & 10303 & 9.2e-05 \\
$\log R_{\rm p}/R_\star$ & -- & $\mathcal{U}(-4.605; 0.000)$ & -3.760 & -3.763 & 0.031 & -3.819 & -3.708 & 4043 & 6.3e-04 \\
$b$ & -- & $\mathcal{U}(0; 1+R_{\mathrm{p}}/R_\star)$ & 0.508 & 0.500 & 0.064 & 0.380 & 0.612 & 4434 & 3.5e-04 \\
$u_1$ & -- & \citet{exoplanet:kipping13} & 0.98 & 0.95 & 0.27 & 0.43 & 1.42 & 5809 & -5.6e-05 \\
$u_2$ & -- & \citet{exoplanet:kipping13} & -0.19 & -0.16 & 0.31 & -0.66 & 0.42 & 4387 & 2.6e-04 \\
$R_\star$ & $R_\odot$ & $\mathcal{N}(0.876; 0.035)$ & 0.874 & 0.874 & 0.036 & 0.804 & 0.938 & 9902 & 7.3e-04 \\
$\log g$ & cgs & $\mathcal{N}(4.499; 0.030)$ & 4.503 & 4.502 & 0.030 & 4.445 & 4.557 & 7527 & 2.7e-05 \\
$\log \sigma_f$ & -- & $\mathcal{N}(\log\langle \sigma_f \rangle; 2.000)$ & -8.314 & -8.314 & 0.012 & -8.337 & -8.292 & 10636 & 1.3e-03 \\
$\langle f \rangle$ & -- & $\mathcal{N}(1.000; 0.100)$ & 1.000 & 1.000 & 0.000 & 1.000 & 1.000 & 9742 & -2.9e-04 \\
$R_{\rm p}/R_\star$ & -- & -- & 0.023 & 0.023 & 0.001 & 0.022 & 0.025 & 4043 & 6.3e-04 \\
$\rho_\star$ & g$\ $cm$^{-3}$ & -- & 1.87 & 1.88 & 0.15 & 1.59 & 2.16 & 6829 & 3.4e-04 \\
$R_{\rm p}$ & $R_{\mathrm{Jup}}$ & -- & 0.198 & 0.198 & 0.011 & 0.177 & 0.218 & 5676 & 2.8e-04 \\
$R_{\rm p}$ & $R_{\mathrm{Earth}}$ & -- & 2.22 & 2.22 & 0.12 & 1.98 & 2.44 & 5676 & 2.8e-04 \\
$a/R_\star$ & -- & -- & 16.67 & 16.68 & 0.45 & 15.86 & 17.54 & 6829 & 3.3e-04 \\
$\cos i$ & -- & -- & 0.030 & 0.030 & 0.004 & 0.022 & 0.038 & 4518 & 5.4e-04 \\
$T_{14}$ & hr & -- & 2.79 & 2.79 & 0.07 & 2.65 & 2.93 & 4845 & 5.0e-04 \\
$T_{13}$ & hr & -- & 2.62 & 2.62 & 0.08 & 2.47 & 2.78 & 4575 & 3.1e-04 \\
\hline
\multicolumn{10}{c}{\emph{KOI-7913}} \\
\hline
$P$ & d & $\mathcal{N}(24.27838; 0.01000)$ & 24.278553 & 24.278571 & 0.000263 & 24.278112 & 24.279085 & 4413 & 1.5e-03 \\
$t_0^{(1)}$ & d & $\mathcal{N}(154.51; 0.05)$ & 154.5121 & 154.5124 & 0.0063 & 154.4998 & 154.5237 & 5612 & 6.0e-04 \\
$\log R_{\rm p}/R_\star$ & -- & $\mathcal{U}(-5.298; 0.000)$ & -3.599 & -3.602 & 0.046 & -3.689 & -3.519 & 4290 & 5.6e-04 \\
$b$ & -- & $\mathcal{U}(0; 1+R_{\mathrm{p}}/R_\star)$ & 0.312 & 0.298 & 0.153 & 0.005 & 0.523 & 2373 & 1.8e-03 \\
$u_1$ & -- & \citet{exoplanet:kipping13} & 0.27 & 0.34 & 0.28 & 0.00 & 0.86 & 4491 & -6.1e-05 \\
$u_2$ & -- & \citet{exoplanet:kipping13} & 0.21 & 0.23 & 0.32 & -0.31 & 0.86 & 5935 & 7.0e-04 \\
$R_\star$ & $R_\odot$ & $\mathcal{N}(0.790; 0.049)$ & 0.788 & 0.788 & 0.049 & 0.699 & 0.881 & 6847 & 2.8e-04 \\
$\log g$ & cgs & $\mathcal{N}(4.523; 0.043)$ & 4.526 & 4.527 & 0.042 & 4.450 & 4.606 & 5714 & 6.6e-04 \\
$\log \sigma_f$ & -- & $\mathcal{N}(\log\langle \sigma_f \rangle; 2.000)$ & -7.197 & -7.197 & 0.019 & -7.230 & -7.161 & 6976 & 1.4e-04 \\
$\langle f \rangle$ & -- & $\mathcal{N}(1.000; 0.100)$ & 1.000 & 1.000 & 0.000 & 1.000 & 1.000 & 6998 & 2.8e-04 \\
$R_{\rm p}/R_\star$ & -- & -- & 0.027 & 0.027 & 0.001 & 0.025 & 0.030 & 4290 & 5.6e-04 \\
$\rho_\star$ & g$\ $cm$^{-3}$ & -- & 2.20 & 2.21 & 0.25 & 1.78 & 2.70 & 5357 & 5.6e-04 \\
$R_{\rm p}$ & $R_{\mathrm{Jup}}$ & -- & 0.209 & 0.209 & 0.016 & 0.179 & 0.238 & 4882 & 1.3e-03 \\
$R_{\rm p}$ & $R_{\mathrm{Earth}}$ & -- & 2.34 & 2.34 & 0.18 & 2.01 & 2.67 & 4882 & 1.3e-03 \\
$a/R_\star$ & -- & -- & 40.92 & 40.95 & 1.54 & 38.14 & 43.84 & 5357 & 6.6e-04 \\
$\cos i$ & -- & -- & 0.008 & 0.007 & 0.004 & 0.000 & 0.013 & 2344 & 1.9e-03 \\
$T_{14}$ & hr & -- & 4.39 & 4.40 & 0.21 & 3.98 & 4.76 & 3952 & 5.6e-04 \\
$T_{13}$ & hr & -- & 4.13 & 4.13 & 0.22 & 3.72 & 4.55 & 3632 & 7.6e-04 \\
	\enddata
	%
	\tablecomments{
		ESS refers to the number of effective samples.
		$\hat{R}$ is the Gelman-Rubin convergence diagnostic.
		Logarithms in this table are base-$e$.
		$\mathcal{U}$ denotes a uniform distribution,
		and $\mathcal{N}$ a normal distribution.
		\added{ Posterior values quoted in the text are means and standard
		deviations for symmetric distributions,
		and are otherwise medians bracketed by the
		upper and lower 84.1 and 15.9 percentile deviations.}
    %Many of the $T_{13}$ statistics may be \texttt{nan} in the event of a
    %grazing transit.
     (1) The ephemeris is in units of BJKD (BJDTDB-2454833).
		%  (2) Although $\mathcal{U}(0,1+R_{\rm p}/R_\star)$ is formally
		%  correct, for this model we assumed a non-grazing transit to enable
		%  sampling in $\log \delta$.
		%  (3) The eccentricity vectors are sampled in the $(e\cos\omega,
		%  e\sin\omega)$ plane.
		%  (4) The true planet size is a factor of $((F_1+F_2)/F_1)^{1/2}$
		%  larger than that from the fit because of dilution from Kepler
		%  1627B, where $F_1$ is the flux from the primary, and $F_2$ is that
		%  from the secondary; the mean and standard deviation of $R_{\rm
		%  p}=3.817\pm0.158\,R_{\oplus}$ quoted in the text includes this correction,
		%  assuming $(F_1+F_2)/F_1\approx 1.015$.
	}
	\vspace{-0.3cm}
\end{deluxetable*}


%\listofchanges
%\allauthors
\end{document}


