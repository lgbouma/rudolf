%% \begin{deluxetable}{} command tell LaTeX how many columns
%% there are and how to align them.
%\startlongtable
\begin{deluxetable}{lll}
    
%% Keep a portrait orientation

%% Over-ride the default font size
%% Use Default (12pt)
%\tabletypesize{\scriptsize}
\tabletypesize{\footnotesize}

%% Use \tablewidth{?pt} to over-ride the default table width.
%% If you are unhappy with the default look at the end of the
%% *.log file to see what the default was set at before adjusting
%% this value.

%% This is the title of the table.
\tablecaption{Rotation periods and kinematics for candidate RSG-5 and
  CH-2 members.}
\label{tab:rot}
%% This command over-rides LaTeX's natural table count
%% and replaces it with this number.  LaTeX will increment 
%% all other tables after this table based on this number
\tablenum{3}

%% The \tablehead gives provides the column headers.  It
%% is currently set up so that the column labels are on the
%% top line and the units surrounded by ()s are in the 
%% bottom line.  You may add more header information by writing
%% another line between these lines. For each column that requries
%% extra information be sure to include a \colhead{text} command
%% and remember to end any extra lines with \\ and include the 
%% correct number of &s.
\tablehead{
  \colhead{Parameter} &
  \colhead{Example Value} &
  \colhead{Description}
}
%% All data must appear between the \startdata and \enddata commands
%
\startdata
\texttt{dr3\_source\_id}      & 2127562009133684480 & Gaia (E)DR3 source identifier. \\
\texttt{ra} &    291.02306     & Gaia EDR3 right ascension [deg]. \\
\texttt{dec} &  46.43843      & Gaia EDR3 declination [deg]. \\
\texttt{parallax} &  3.7099      & Gaia EDR3 parallax [milliarcsec]. \\
\texttt{ruwe} &  0.981      & Gaia EDR3 renormalized unit weight error. \\
\texttt{weight} & 0.087    & Strength of connectivity to other candidate cluster members. \\
\texttt{v\_l} & 2.78     & Longitudinal galactic velocity, $v_{l^*}$ [km\,s$^{-1}$]. \\
\texttt{v\_b} & -2.87     & Latitudinal galactic velocity [km\,s$^{-1}$]. \\
\texttt{x\_pc} & -8068.5  & Galactocentric $X$ position coordinate [pc]. \\
\texttt{y\_pc} & 256.0     & Galactocentric $Y$ position coordinate [pc]. \\
\texttt{z\_pc} & 86.3      & Galactocentric $Z$ position coordinate [pc]. \\
\texttt{(BP-RP)0} & -0.115 &  Gaia $G_\mathrm{BP}$-$G_\mathrm{RP}$   color, minus $E$($G_\mathrm{BP}$-$G_\mathrm{RP}$). \\
\texttt{(M\_G)0} & 0.442 & Absolute $G$-band magnitude, corrected for extinction. \\
\texttt{cluster} & CH-2 & RSG-5 or CH-2. \\
\texttt{Prot\_Adopted} & NaN & Adopted rotation period if available, else NaN [days]. \\
\texttt{Prot\_TESS} & NaN & TESS rotation period if available, else NaN [days]. \\
\texttt{Prot\_ZTF}  & NaN & ZTF rotation period if available, else NaN [days]. \\
\texttt{Prot\_Confused} & NaN & Boolean flag; true when stars are photometrically blended. \\
\enddata
%% General table comment marker
\tablecomments{Table~3 is published in its entirety in a machine-readable
format.  One entry is shown for guidance regarding form and content.
}
\vspace{-0.5cm}
\end{deluxetable}
