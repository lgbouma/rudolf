\begin{table*}
\scriptsize
\setlength{\tabcolsep}{2pt}
\centering
\caption{Literature and Measured Properties for Kepler$\,$1627 A}
\label{tab:starparams}
%\tablenum{2}
\begin{tabular}{llcc}
  \hline
  \hline
Other identifiers\dotfill & \\
\multicolumn{3}{c}{TIC 120105470} \\
\multicolumn{3}{c}{GAIADR2 2103737241426734336} \\
\multicolumn{3}{c}{GAIAEDR3 2103737241426734336} \\
\hline
\hline
Parameter & Description & Value & Source\\
\hline 
$\alpha_{J2015.5}$\dotfill	&Right Ascension (hh:mm:ss)\dotfill & 18:56:13.6 & 1	\\
$\delta_{J2015.5}$\dotfill	&Declination (dd:mm:ss)\dotfill & +41:34:36.22 & 1	\\
%$l_{J2015.5}$\dotfill	&Galactic Longitude (deg)\dotfill & 288.2644 & 1	\\
%$b_{J2015.5}$\dotfill	&Galactic Latitude (deg)\dotfill & -5.7950 & 1	\\
%\\
%$NUV$\dotfill           & GALEX $NUV$ mag.\dotfill & 13.804 $\pm$ 0.004 & 2 \\
%$FUV$\dotfill           & GALEX $FUV$ mag.\dotfill & 18.466 $\pm$ 0.056 & 2 \\
\\
%B\dotfill			&Johnson B mag.\dotfill & 11.119 $\pm$ 0.107		& 2	\\
V\dotfill			&Johnson V mag.\dotfill & 13.11 $\pm$ 0.08		& 2	\\
%$B$\tablenote{The uncertainties of the photometry have a systematic error floor applied. Even still, the global fit requires a significant scaling of the uncertainties quoted here to be consistent with our model, suggesting they are still significantly underestimated for one or more of the broad band magnitudes}\dotfill		& APASS Johnson $B$ mag.\dotfill	& 13.001 $\pm$	0.02& 2	\\
%$V$\dotfill		& APASS Johnson $V$ mag.\dotfill	& 11.808 $\pm$	0.02& 2	\\
%\\
${\rm G}$\dotfill     & Gaia $G$ mag.\dotfill     & 13.049$\pm$0.02 & 1\\
%${\rm Bp}$\dotfill     & Gaia $Bp$ mag.\dotfill     & 10.695 $\pm$0.020 & 1\\
%${\rm Rp}$\dotfill     & Gaia $Rp$ mag.\dotfill     & 9.887$\pm$0.020 & 1\\
${\rm T}$\dotfill     & TESS $T$ mag.\dotfill     & 12.53$\pm$0.02 & 2\\
%$u'$\dotfill        & Sloan $u'$ mag.\dotfill & 14.706 $\pm$ 0.006& 3\\
%$g'$\dotfill		& APASS Sloan $g'$ mag.\dotfill	& 12.407 $\pm$ 0.02	&  2	\\
%$r'$\dotfill		& APASS Sloan $r'$ mag.\dotfill	& 11.311 $\pm$ 0.02	&  2	\\
%$i'$\dotfill		& APASS Sloan $i'$ mag.\dotfill	& 10.927 $\pm$ 0.04 &  2	\\
%\\
J\dotfill			& 2MASS J mag.\dotfill & 11.69  $\pm$ 0.02	& 3	\\
H\dotfill			& 2MASS H mag.\dotfill & 11.30 $\pm$ 0.02	    &  3	\\
K$_{\rm S}$\dotfill			& 2MASS ${\rm K_S}$ mag.\dotfill & 11.19 $\pm$ 0.02 &  3	\\
%\\
%W1\dotfill		& WISE1 mag.\dotfill & 8.901 $\pm$ 0.023 & 4	\\
%W2\dotfill		& WISE2 mag.\dotfill & 8.875 $\pm$ 0.021 &  4 \\
%W3\dotfill		& WISE3 mag.\dotfill &  8.875 $\pm$ 0.020& 4	\\
%W4\dotfill		& WISE4 mag.\dotfill & 8.936 $\pm$ N/A &  4	\\
\\
$\pi$\dotfill & Gaia EDR3 parallax (mas) \dotfill & 3.009 $\pm$ 0.032 &  1 \\
$d$\dotfill & Distance (pc)\dotfill & $329.5 \pm 3.5$ & 1, 4 \\
$\mu_{\alpha}$\dotfill		& Gaia EDR3 proper motion\dotfill & 1.716 $\pm$ 0.034 & 1 \\
                    & \hspace{3pt} in RA (mas yr$^{-1}$)	&  \\
$\mu_{\delta}$\dotfill		& Gaia EDR3 proper motion\dotfill 	&  -1.315 $\pm$ 0.034 &  1 \\
                    & \hspace{3pt} in DEC (mas yr$^{-1}$) &  \\
RUWE\dotfill		& Gaia EDR3 renormalized\dotfill 	&  2.899 &  1 \\
                    & \hspace{3pt} unit weight error &  \\
%
\\
RV\dotfill & Systemic radial \hspace{9pt}\dotfill  & $-14.3 \pm 1.0$ & 5 \\
                    & \hspace{3pt} velocity (\kms)  & \\
$v\sin{i_\star}$\dotfill &  Rotational velocity (\kms) \hspace{9pt}\dotfill &  18.9 $\pm$ 1.0 & 5 \\
%$v_{\rm mac}$\dotfill &  Macroturbulence velocity (\kms) \hspace{9pt}\dotfill &  8.4 $\pm$ 2.9 & 5 \\
${\rm [Fe/H]}$\dotfill &   Metallicity \hspace{9pt}\dotfill & -0.02 $\pm$ 0.09 & 5 \\
$T_{\rm eff}$\dotfill &  Effective Temperature (K) \hspace{9pt}\dotfill & 5777 $\pm$ 110 &  5  \\
$\log{g_{\star}}$\dotfill &  Surface Gravity (cgs)\hspace{9pt}\dotfill &  4.6 $\pm$ 0.1  &  5 \\
%
% $E(B-V)$\dotfill & Reddening (mag)\dotfill & $0.06 \pm 0.02$ & 9 \\
%
Spec. Type\dotfill & Spectral Type\dotfill & 	G8V & 5 \\
%
%FIXME all these values need to be determined
$R_\star$\dotfill & Stellar radius ($R_\odot$)\dotfill & 1.022$\pm$0.083 & 5 \\
$M_\star$\dotfill & Stellar mass ($R_\odot$)\dotfill & 1.118$\pm$0.059 & 5 \\
%$F_{\rm bol}$\dotfill & Stellar bolometric flux (cgs)\dotfill & (1.967$\pm$0.046)$\times10^{-9}$ & 9 \\
%
Li EW\dotfill & 6708\AA\ Equiv{.} Width (m\AA) \dotfill & $233^{+5}_{-7}$  & 5 \\
%
%FIXME this value need to be determined
$A_{\rm V}$\dotfill & Interstellar reddening (mag)\dotfill & 0.20$\pm$0.03 & 6 \\
%
$P_{\rm rot}$\dotfill & Rotation period (d)\dotfill & $2.642\pm 0.042$  & 7 \\
Age & Adopted stellar age (Myr)\dotfill & $35^{+6}_{-5}$  &  8 \\
% $U^{*}$\dotfill & Space Velocity (\kms)\dotfill & $26.24 \pm 0.46$  & \S\ref{sec:uvw} \\
% $V$\dotfill       & Space Velocity (\kms)\dotfill & $-71.52 \pm 1.68$ & \S\ref{sec:uvw} \\
% $W$\dotfill       & Space Velocity (\kms)\dotfill & $ -1.31 \pm 0.27$ & \S\ref{sec:uvw} \\
\hline
\end{tabular}
\begin{flushleft}
 \footnotesize{ \textsc{NOTE}---
%$\dagger$ Systemic RV uncertainty is the standard deviation of single-transit radial velocities, as quoted in Gaia DR2. %$*$ $U$ is in the direction of the Galactic center. \\
Provenances are:
$^1$\citet{gaia_collaboration_gaia_2018},
$^2$\citet{stassun_TIC8_2019},
$^3$\citet{skrutskie_tmass_2006},
$^4$\citet{Lindegren_2021_offset},
$^5$HIRES spectra and \citet{yee_SM_2017},
%$^6$Method~2 (cluster isochrone, Section~\ref{subsec:starparams}),
%$^7$FEROS spectra,
$^6$\citet{capitanio_threedimensional_2017} and \citet{lallement_threedimensional_2018},
$^7$Kepler light curve,
$^8$Pre-main-sequence CMD, with LDB age for IC~2602 being most
important (Section~\ref{sec:cmd}).
%$^{10}$Method~1 (photometric SED fit, Section~\ref{subsec:starparams}).}
}
\end{flushleft}
\vspace{-0.5cm}
\end{table*}
