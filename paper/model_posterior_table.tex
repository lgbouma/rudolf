% Table of best fit parameters
%\startlongtable
\begin{deluxetable*}{lllrrrrrrr}
%
  \tablecaption{ Priors and posteriors for the transit and stellar
  variability model fitted to the long-cadence Kepler 1627b
  photometric timeseries.}
\label{tab:posterior}
%
\tabletypesize{\scriptsize}
%\tabletypesize{\small}
%
%\tablenum{2}
%
\tablehead{
  \colhead{Param.} & 
  \colhead{Unit} &
  \colhead{Prior} & 
  \colhead{Median} & 
  \colhead{Mean} & 
  \colhead{Std{.} Dev.} &
  \colhead{3\%} &
  \colhead{97\%} &
  \colhead{ESS} &
  \colhead{$\hat{R}-1$}
}

%/Users/luke/Dropbox/proj/rudolf/results/run_gptransit/Kepler_1627_gptransit_posteriortable.tex
\startdata
{\it Sampled} & & & & & & & & & \\
\hline
$P$ & d & $\mathcal{N}(7.20281; 0.01000)$ & 7.2028035 & 7.2028033 & 0.0000073 & 7.2027893 & 7.2028171 & 1910.5903732 & 0.0035905 \\
$t_0^{(1)}$ & d & $\mathcal{N}(120.79053; 0.02000)$ & 120.790504 & 120.790505 & 0.0009438 & 120.7886867 & 120.7922431 & 1564.1105056 & 0.0003213 \\
$\log R_{\rm p}/R_\star$ & -- & $\mathcal{U}(-4.605; 0.000)$ & -3.33523 & -3.33569 & 0.06618 & -3.45772 & -3.21617 & 1173.56574 & 0.00310 \\
$b$ & -- & $\mathcal{U}(0; 1+R_{\mathrm{p}}/R_\star)$ & 0.3971 & 0.3886 & 0.2070 & 0.0204 & 0.7289 & 378.9528 & 0.0177 \\
$u_1$ & -- & \citet{exoplanet:kipping13} & 0.28 & 0.30 & 0.179 & 0.002 & 0.603 & 1161.95 & 0.005 \\
$u_2$ & -- & \citet{exoplanet:kipping13} & 0.425 & 0.381 & 0.314 & -0.197 & 0.912 & 900.884 & 0.002 \\
$R_\star$ & $R_\odot$ & $\mathcal{T}(0.910; 0.052)$ & 0.911 & 0.910 & 0.051 & 0.814 & 1.004 & 2265.830 & -0.001 \\
$\log g$ & cgs & $\mathcal{N}(4.600; 0.100)$ & 4.604 & 4.601 & 0.094 & 4.417 & 4.769 & 923.943 & 0. \\
$\langle f \rangle$ & -- & $\mathcal{N}(0.500; 0.100)$ & 0.4999 & 0.4999 & 0.0003 & 0.4993 & 0.5005 & 2964.6676 & 0.0013 \\
$e^{(2)}$ & -- & \citet{vaneylen19} & 0.127 & 0.168 & 0.147 & 0. & 0.446 & 518.492 & 0.004 \\
$\omega$ & rad & $\mathcal{U}(0.000; 6.283)$ & -0.235 & -0.170 & 1.867 & -2.879 & 3.132 & 1212.260 & 0.006 \\
$\log \sigma_f$ & -- & $\mathcal{N}(\log\langle \sigma_f \rangle; 2.000)$ & -8.016 & -8.016 & 0.008 & -8.031 & -8.001 & 2224.427 & -0. \\
$\rho$ & d & $\mathcal{U}(1.000; 10.000)$ & 2.953 & 2.955 & 0.096 & 2.777 & 3.131 & 1936.492 & 0. \\
$\sigma$ & d$^{-1}$ & $\mathrm{InvGamma}(1.000; 5.000)$ & 0.013 & 0.013 & 0.001 & 0.012 & 0.014 & 2155.687 & -0. \\
$\sigma_{\mathrm{rot}}$ & d$^{-1}$ & $\mathrm{InvGamma}(1.000; 5.000)$ & 0.897 & 0.933 & 0.222 & 0.552 & 1.316 & 2147.324 & 0.003 \\
$\log P_{\mathrm{rot}}$ & $\log (\mathrm{d})$ & $\mathcal{N}(0.958; 0.020)$ & 0.964 & 0.964 & 0.001 & 0.963 & 0.966 & 2486.029 & 0. \\
$\log Q_0$ & -- & $\mathcal{N}(0.000; 2.000)$ & 12.935 & 12.960 & 0.454 & 12.157 & 13.857 & 2126.581 & 0.004 \\
$\log \mathrm{d}Q$ & -- & $\mathcal{N}(0.000; 2.000)$ & 0.029 & 0.021 & 2.032 & -3.785 & 3.780 & 1755.941 & 0. \\
$f$ & -- & $\mathcal{U}(0.100; 1.000)$ & 0.111 & 0.113 & 0.010 & 0.1 & 0.130 & 1253.358 & -0.0 \\
\hline
{\it Derived} & & & & & & & & & \\
\hline
$R_{\rm p}/R_\star$ & -- & -- & 0.036 & 0.036 & 0.002 & 0.031 & 0.040 & 1173.566 & 0.003 \\
$\rho_\star$ & g$\ $cm$^{-3}$ & -- & 2.27 & 2.312 & 0.516 & 1.408 & 3.272 & 910.746 & 0.001 \\
$R_{\rm p}$ & $R_{\mathrm{Jup}}$ & -- & 0.316 & 0.317 & 0.037 & 0.251 & 0.386 & 1692.736 & 0.001 \\
$a/R_\star$ & -- & -- & 18.396 & 18.407 & 1.375 & 15.690 & 20.780 & 910.717 & 0.001 \\
$\cos i$ & -- & -- & 0.022 & 0.021 & 0.011 & 0.002 & 0.038 & 439.303 & 0.011 \\
$T_{14}$ & hr & -- & 2.822 & 2.823 & 0.057 & 2.710 & 2.917 & 1086.008 & 0.002 \\
$T_{13}$ & hr & -- & 2.578 & 2.566 & 0.083 & 2.430 & 2.714 & 590.630 & 0.011 \\
\enddata
%
\tablecomments{
  ESS refers to the number of effective samples.
  $\hat{R}$ is the Gelman-Rubin convergence diagnostic.
  Logarithms through this table are in base-$e$.
  $\mathcal{U}$ denotes a uniform distribution,
  $\mathcal{N}$ a normal distribution, and
  $\mathcal{T}$ a truncated normal bounded between zero and an upper limit much larger than the mean.
  (1) The ephemeris is in units of BJDTDB - 2454833.
  (2) The eccentricity vectors are sampled in the $(e\cos\omega,
  e\sin\omega)$ basis.
%
% (2) Uninformative quadratic limb-darkening prior from \citet{exoplanet:kipping13}, implemented by \citet{exoplanet:exoplanet}.
% The precision achieved in the ground-based data did not appear to
% necessitate using bandpass-dependent limb-darkening coefficients.
% For comparison, the \citet{claret_limb_2017} parameters for
% the appropriate $T_{\rm eff}$ and $\log g$ in TESS-band would have been 
% $(u_1, u_2) = (0.3249, 0.235)$.
%
% (2) Assuming an informative quadratic limb-darkening prior with
% values about those given for the appropriate $T_{\rm eff}$ and
% $\log g$ in TESS-band from \citet{claret_limb_2017}. The precision
% achieved in the ground-based data did not appear to necessitate using
% bandpass-dependent limb-darkening coefficients.
% (3) The second and third contact points do not exist for a grazing transit.
% {\it Notation}:
% $a_{ij;\mathrm{Instr}}$ denotes the $i^{\rm th}$ transit of a
% particular instrument, and the $j^{\rm th}$ polynomial detrending
% order.
}
\vspace{-0.3cm}
\end{deluxetable*}
