% Table of best fit parameters
%\startlongtable
\begin{deluxetable*}{lllrrrrrrr}
%
  \tablecaption{ Priors and Posteriors for Model Fitted to the Long
  Cadence Kepler 1627Ab Light Curve.}
\label{tab:posterior}
%
\tabletypesize{\scriptsize}
%\tabletypesize{\small}
%
%\tablenum{2}
%
\tablehead{
  \colhead{Param.} & 
  \colhead{Unit} &
  \colhead{Prior} & 
  \colhead{Median} & 
  \colhead{Mean} & 
  \colhead{Std{.} Dev.} &
  \colhead{3\%} &
  \colhead{97\%} &
  \colhead{ESS} &
  \colhead{$\hat{R}-1$}
}

%/Users/luke/Dropbox/proj/rudolf/results/run_RotGPtransit/Kepler_1627_RotGPtransit_posteriortable.tex
\startdata
{\it Sampled} & & & & & & & & & \\
\hline
$P$ & d & $\mathcal{N}(7.20281; 0.01000)$ & 7.2028038 & 7.2028038 & 0.0000073 & 7.2027895 & 7.2028168 & 7464 & 3.9e-04 \\
$t_0^{(1)}$ & d & $\mathcal{N}(120.79053; 0.02000)$ & 120.7904317 & 120.7904254 & 0.0009570 & 120.7886377 & 120.7921911 & 3880 & 2.0e-03 \\
$\log \delta$ & -- & $\mathcal{N}(-6.3200; 2.0000)$ & -6.3430 & -6.3434 & 0.0354 & -6.4094 & -6.2767 & 6457 & 3.0e-04 \\
$b^{(2)}$ & -- & $\mathcal{U}(0.000; 1.000)$ & 0.4669 & 0.4442 & 0.2025 & 0.0662 & 0.8133 & 1154 & 1.6e-03 \\
$u_1$ & -- & \citet{exoplanet:kipping13} & 0.271 & 0.294 & 0.190 & 0.000 & 0.628 & 3604 & 1.5e-03 \\
$u_2$ & -- & \citet{exoplanet:kipping13} & 0.414 & 0.377 & 0.326 & -0.240 & 0.902 & 3209 & 1.4e-03 \\
$R_\star$ & $R_\odot$ & $\mathcal{N}(0.881; 0.018)$ & 0.881 & 0.881 & 0.018 & 0.847 & 0.915 & 8977 & 3.1e-04 \\
$\log g$ & cgs & $\mathcal{N}(4.530; 0.050)$ & 4.532 & 4.533 & 0.051 & 4.435 & 4.627 & 6844 & 1.6e-03 \\
$\langle f \rangle$ & -- & $\mathcal{N}(0.000; 0.100)$ & -0.0003 & -0.0003 & 0.0001 & -0.0005 & -0.0000 & 8328 & 1.1e-03 \\
$e^{(3)}$ & -- & \citet{vaneylen19} & 0.154 & 0.186 & 0.152 & 0.000 & 0.459 & 1867 & 2.0e-03 \\
$\omega$ & rad & $\mathcal{U}(0.000; 6.283)$ & 0.055 & 0.029 & 1.845 & -3.139 & 2.850 & 3557 & 8.6e-05 \\
$\log \sigma_f$ & -- & $\mathcal{N}(\log\langle \sigma_f \rangle; 2.000)$ & -8.035 & -8.035 & 0.008 & -8.049 & -8.021 & 9590 & 3.9e-04 \\
$\sigma_{\mathrm{rot}}$ & d$^{-1}$ & $\mathrm{InvGamma}(1.000; 5.000)$ & 0.070 & 0.070 & 0.001 & 0.068 & 0.072 & 9419 & 1.4e-03 \\
$\log P_{\mathrm{rot}}$ & $\log (\mathrm{d})$ & $\mathcal{N}(0.958; 0.020)$ & 0.978 & 0.978 & 0.001 & 0.975 & 0.980 & 8320 & 2.2e-04 \\
$\log Q_0$ & -- & $\mathcal{N}(0.000; 2.000)$ & -0.327 & -0.326 & 0.043 & -0.407 & -0.246 & 9659 & 2.7e-04 \\
$\log \mathrm{d}Q$ & -- & $\mathcal{N}(0.000; 2.000)$ & 7.697 & 7.698 & 0.103 & 7.511 & 7.899 & 5824 & 3.7e-04 \\
$f$ & -- & $\mathcal{U}(0.010; 1.000)$ & 0.01006 & 0.01009 & 0.00009 & 0.01000 & 0.01025 & 4645 & 4.0e-04 \\
{\it Derived} & & & & & & & & & \\
\hline
$\delta$ & -- & -- & 0.001759 & 0.001759 & 0.000062 & 0.001641 & 0.001875 & 6457 & 3.0e-04 \\
$R_{\rm p}/R_\star$ & -- & -- & 0.039 & 0.039 & 0.001 & 0.037 & 0.042 & 1811 & 1.1e-03 \\
$\rho_\star$ & g$\ $cm$^{-3}$ & -- & 1.990 & 2.004 & 0.240 & 1.570 & 2.461 & 6905 & 2.1e-03 \\
$R_{\rm p}^{(4)}$ & $R_{\mathrm{Jup}}$ & -- & 0.337 & 0.338 & 0.014 & 0.314 & 0.367 & 2311 & 1.0e-03 \\
$R_{\rm p}^{(4)}$ & $R_{\mathrm{Earth}}$ & -- & 3.777 & 3.789 & 0.157 & 3.52 & 4.114 & 2311 & 1.0e-03 \\
$a/R_\star$ & -- & -- & 17.606 & 17.619 & 0.702 & 16.277 & 18.906 & 6905 & 2.1e-03 \\
$\cos i$ & -- & -- & 0.027 & 0.025 & 0.010 & 0.004 & 0.040 & 1312 & 1.2e-03 \\
$T_{14}$ & hr & -- & 2.841 & 2.843 & 0.060 & 2.734 & 2.958 & 3199 & 3.6e-04 \\
$T_{13}$ & hr & -- & 2.555 & 2.539 & 0.094 & 2.360 & 2.692 & 1960 & 1.4e-03 \\
\enddata
%
\tablecomments{
  ESS refers to the number of effective samples.
  $\hat{R}$ is the Gelman-Rubin convergence diagnostic.
  Logarithms in this table are base-$e$.
  $\mathcal{U}$ denotes a uniform distribution,
  and $\mathcal{N}$ a normal distribution.
  (1) The ephemeris is in units of BJDTDB - 2454833.
  (2) Although $\mathcal{U}(0,1+R_{\rm p}/R_\star)$ is formally
  correct, for this model we assumed a non-grazing transit to enable
  sampling in $\log \delta$.
  (3) The eccentricity vectors are sampled in the $(e\cos\omega,
  e\sin\omega)$ plane.
  (4) The true planet size is a factor of $((F_1+F_2)/F_1)^{1/2}$
  larger than that from the fit because of dilution from Kepler
  1627B, where $F_1$ is the flux from the primary, and $F_2$ is that
  from the secondary; the mean and standard deviation of $R_{\rm
  p}=3.817\pm0.158\,R_{\oplus}$ quoted in the text includes this correction,
  assuming $(F_1+F_2)/F_1\approx 1.015$.
}
\vspace{-0.3cm}
\end{deluxetable*}
