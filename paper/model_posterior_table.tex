% Table of best fit parameters
%\startlongtable
\begin{deluxetable*}{lllrrrrrrr}
%
  \tablecaption{ Priors and Posteriors for Model Fitted to the Long
  Cadence Kepler 1627Ab Light Curve.}
\label{tab:posterior}
%
\tabletypesize{\scriptsize}
%\tabletypesize{\small}
%
%\tablenum{2}
%
\tablehead{
  \colhead{Param.} & 
  \colhead{Unit} &
  \colhead{Prior} & 
  \colhead{Median} & 
  \colhead{Mean} & 
  \colhead{Std{.} Dev.} &
  \colhead{3\%} &
  \colhead{97\%} &
  \colhead{ESS} &
  \colhead{$\hat{R}-1$}
}

%/Users/luke/Dropbox/proj/rudolf/results/run_RotGPtransit/Kepler_1627_RotGPtransit_posteriortable.tex
\startdata
{\it Sampled} & & & & & & & & & \\
\hline
$P$ & d & $\mathcal{N}(7.20281; 0.01000)$ & 7.2028039 & 7.2028039 & 7.4e-06 & 7.2027901 & 7.202818 & 8673 & 0.0008753 \\
$t_0^{(1)}$ & d & $\mathcal{N}(120.79053; 0.02000)$ & 120.7904073 & 120.7904027 & 0.0009503 & 120.7886845 & 120.7922021 & 6349 & 0.0005352 \\
$\log R_{\rm p}/R_\star$ & -- & $\mathcal{U}(-4.605; 0.000)$ & -3.36501 & -3.36176 & 0.0406 & -3.43579 & -3.28478 & 2217 & 0.00138 \\
$b$ & -- & $\mathcal{U}(0; 1+R_{\mathrm{p}}/R_\star)$ & 0.4638 & 0.4386 & 0.1994 & 0.0011 & 0.7415 & 1273 & 0.0015 \\
$u_1$ & -- & \citet{exoplanet:kipping13} & 0.27 & 0.293 & 0.184 & 0.0 & 0.613 & 3534 & 0.001 \\
$u_2$ & -- & \citet{exoplanet:kipping13} & 0.422 & 0.385 & 0.317 & -0.213 & 0.901 & 3486 & 0.001 \\
$R_\star$ & $R_\odot$ & $\mathcal{T}(0.881; 0.018)$ & 0.881 & 0.881 & 0.018 & 0.844 & 0.913 & 6920 & -0.0 \\
$\log g$ & cgs & $\mathcal{N}(4.530; 0.050)$ & 4.531 & 4.532 & 0.051 & 4.438 & 4.628 & 7262 & 0.0 \\
$\langle f \rangle$ & -- & $\mathcal{N}(0.500; 0.100)$ & 0.4999 & 0.4999 & 0.0001 & 0.4997 & 0.5 & 8553 & 0.0002 \\
$e^{(2)}$ & -- & \citet{vaneylen19} & 0.147 & 0.181 & 0.152 & 0.0 & 0.456 & 1954 & 0.001 \\
$\omega$ & rad & $\mathcal{U}(0.000; 6.283)$ & 0.097 & 0.058 & 1.84 & -2.867 & 3.141 & 3805 & 0.001 \\
$\log \sigma_f$ & -- & $\mathcal{N}(\log\langle \sigma_f \rangle; 2.000)$ & -8.034 & -8.034 & 0.008 & -8.048 & -8.02 & 7639 & 0.0 \\
$\sigma_{\mathrm{rot}}$ & d$^{-1}$ & $\mathrm{InvGamma}(1.000; 5.000)$ & 0.07 & 0.07 & 0.001 & 0.068 & 0.072 & 8198 & 0.001 \\
$\log P_{\mathrm{rot}}$ & $\log (\mathrm{d})$ & $\mathcal{N}(0.958; 0.020)$ & 0.978 & 0.978 & 0.001 & 0.975 & 0.98 & 7991 & 0.0 \\
$\log Q_0$ & -- & $\mathcal{N}(0.000; 2.000)$ & -0.325 & -0.325 & 0.043 & -0.407 & -0.246 & 8304 & 0.002 \\
$\log \mathrm{d}Q$ & -- & $\mathcal{N}(0.000; 2.000)$ & 7.699 & 7.697 & 0.103 & 7.505 & 7.888 & 8140 & 0.001 \\
$f$ & -- & $\mathcal{U}(0.010; 1.000)$ & 0.01 & 0.01 & 0.0 & 0.01 & 0.01 & 5097 & 0.002 \\
$R_{\rm p}/R_\star$ & -- & -- & 0.035 & 0.035 & 0.001 & 0.032 & 0.037 & 2217 & 0.001 \\
$\rho_\star$ & g$\ $cm$^{-3}$ & -- & 1.984 & 2.001 & 0.236 & 1.573 & 2.45 & 7248 & 0.0 \\
$R_{\rm p}$ & $R_{\mathrm{Jup}}$ & -- & 0.297 & 0.298 & 0.016 & 0.269 & 0.33 & 3047 & 0.001 \\
$R_{\rm p}^{(3)}$ & $R_{\mathrm{Earth}}$ & -- & 3.329 & 3.34 & 0.179 & 3.015 & 3.699 & 3047 & 0.001 \\
$a/R_\star$ & -- & -- & 17.589 & 17.611 & 0.691 & 16.28 & 18.87 & 7247 & 0.0 \\
$\cos i$ & -- & -- & 0.027 & 0.025 & 0.01 & 0.003 & 0.039 & 1435 & 0.002 \\
$T_{14}$ & hr & -- & 2.825 & 2.826 & 0.057 & 2.717 & 2.927 & 4099 & 0.001 \\
$T_{13}$ & hr & -- & 2.575 & 2.562 & 0.086 & 2.415 & 2.71 & 2190 & -0.0 \\
\enddata
%
\tablecomments{
  ESS refers to the number of effective samples.
  $\hat{R}$ is the Gelman-Rubin convergence diagnostic.
  Logarithms in this table are in base-$e$.
  $\mathcal{U}$ denotes a uniform distribution,
  $\mathcal{N}$ a normal distribution, and
  $\mathcal{T}$ a truncated normal bounded between zero and an upper limit much larger than the mean.
  (1) The ephemeris is in units of BJDTDB - 2454833.
  (2) The eccentricity vectors are sampled in the $(e\cos\omega,
  e\sin\omega)$ basis.
  (3) The true planet size is a factor of $((F_1+F_2)/F_1)^{1/2}$
  larger than that reported here because of dilution from Kepler
  1627B, for $F_1$ the flux from the primary, and $F_2$ that from the
  secondary; the value of $R_{\rm p}=3.4\pm0.2$ quoted in the text
  accounts for this, assuming $(F_1+F_2)/F_1\approx 1.01$.
%
% (2) Uninformative quadratic limb-darkening prior from \citet{exoplanet:kipping13}, implemented by \citet{exoplanet:exoplanet}.
% The precision achieved in the ground-based data did not appear to
% necessitate using bandpass-dependent limb-darkening coefficients.
% For comparison, the \citet{claret_limb_2017} parameters for
% the appropriate $T_{\rm eff}$ and $\log g$ in TESS-band would have been 
% $(u_1, u_2) = (0.3249, 0.235)$.
%
% (2) Assuming an informative quadratic limb-darkening prior with
% values about those given for the appropriate $T_{\rm eff}$ and
% $\log g$ in TESS-band from \citet{claret_limb_2017}. The precision
% achieved in the ground-based data did not appear to necessitate using
% bandpass-dependent limb-darkening coefficients.
% (3) The second and third contact points do not exist for a grazing transit.
% {\it Notation}:
% $a_{ij;\mathrm{Instr}}$ denotes the $i^{\rm th}$ transit of a
% particular instrument, and the $j^{\rm th}$ polynomial detrending
% order.
}
\vspace{-0.3cm}
\end{deluxetable*}
