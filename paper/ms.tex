%%%%%%%%%%%%%%%%%%%%%%%%%%%%%%%%%%%%%%%%%%%%%%%%%%%%%%%%%%%%%%%%%%%%%%%%%%%%%%%

\documentclass[12pt,twocolumn,tighten]{aastex63}
%\documentclass[12pt,twocolumn,tighten,linenumbers]{aastex63}
%\documentclass[12pt,twocolumn,tighten,trackchanges]{aastex63}
\usepackage{amsmath,amstext,amssymb}
\usepackage[T1]{fontenc}
\usepackage{apjfonts}
\usepackage[figure,figure*]{hypcap}
\usepackage{graphics,graphicx}
\usepackage{hyperref}
\usepackage{natbib}
\usepackage[caption=false]{subfig} % for subfloat
\usepackage{enumitem} % for specific spacing of enumerate
\usepackage{epigraph}

\renewcommand*{\sectionautorefname}{Section} %for \autoref
\renewcommand*{\subsectionautorefname}{Section} %for \autoref

\newcommand{\cn}{$\delta$~Lyr\ cluster} % cluster name
\newcommand{\sn}{Kepler\,1627} % star system name (binary)
\newcommand{\pn}{Kepler\,1627Ab} % plane name

% gaia target sample numbers
\newcommand{\nkinematic}{3{,}298} % "fullfaint" kinematic sample, core+halo (CG18+KC19+M21). from any call of earhat.helpers._get_fullfaint_dataframes
\newcommand{\nnbhd}{13{,}843} % make_compstar_NGC_2516_sourcelist.py
\newcommand{\ncore}{1{,}106}  % "fullfaint" kinematic sample, CG18. from any call of earhat.helpers._get_fullfaint_dataframes
\newcommand{\nhalo}{2{,}192} % "fullfaint" kinematic sample, KC19+M21. from any call of earhat.helpers._get_fullfaint_dataframes


%
% Symbols
%
\newcommand{\kms}{\,km\,s$^{-1}$}
\newcommand{\ms}{\,m\,s$^{-1}$}
\newcommand{\bpmrpo}{(G_{\rm BP}-G_{\rm RP})_0}
\newcommand{\bpmrp}{G_{\rm BP}-G_{\rm RP}}

%% Reintroduced the \received and \accepted commands from AASTeX v5.2.
%% Add "Submitted to " argument.
\received{---}
\revised{---}
\accepted{---}
\submitjournal{Nature}
\shorttitle{Kepler\,1627}

\begin{document}

\defcitealias{cantatgaudin_gaia_2018}{CG18}
\defcitealias{kounkel_untangling_2019}{KC19}
\defcitealias{meingast_2021}{M21}

% Cluster Difference Imaging Photometric Survey. III.
% Rotation and Lithium in an Open Cluster Spanning 500\,Parsecs
\title{
  A 30 Million Year Old Mini-Neptune in the Kepler Field
}

%\suppressAffiliations
%\NewPageAfterKeywords
\correspondingauthor{L.\,G.\,Bouma}
\email{luke@astro.caltech.edu}

\author[0000-0002-0514-5538]{L. G. Bouma}
\altaffiliation{51 Pegasi b Fellow}
\affiliation{Cahill Center for Astrophysics, California Institute of Technology, Pasadena, CA 91125, USA}

% Key authors:
% ... Kinematics
\author[0000-0002-6549-9792]{R.~Kerr} % Y
\affiliation{Department of Astronomy, The University of Texas at Austin, Austin, TX 78712, USA}% % ... Kepler correlations
%
% ... stellar rotation & the initial crossmatch
\author[0000-0002-2792-134X]{J. L. Curtis} % Y
\affiliation{Department of Astronomy, Columbia University, 550 West 120th Street, New York, NY 10027, USA}
%
% HIRES
\author[0000-0002-0531-1073]{H. Isaacson} % Y
\affiliation{Astronomy Department, University of California, Berkeley, CA 94720, USA}
%
% planet-fitting, kinematics, cluster.
\author{L. A. Hillenbrand} % R
\affiliation{Cahill Center for Astrophysics, California Institute of Technology, Pasadena, CA 91125, USA}
%
% HIRES Collaborators
\author[0000-0001-8638-0320]{A. W. Howard} % Y
\affiliation{Cahill Center for Astrophysics, California Institute of Technology, Pasadena, CA 91125, USA}
%
% AO IMAGING
\author[0000-0001-9811-568X]{A.~L.~Kraus} % Y
\affiliation{Department of Astronomy, The University of Texas at Austin, Austin, TX 78712, USA}
%
% TRES
\author[0000-0001-6637-5401]{A. Bieryla} % Y
\affiliation{Center for Astrophysics \textbar \ Harvard \& Smithsonian, 60 Garden St, Cambridge, MA 02138, USA}
%
% TRES
\author[0000-0001-9911-7388]{D. W.~Latham} % Y
\affiliation{Center for Astrophysics \textbar \ Harvard \& Smithsonian, 60 Garden St, Cambridge, MA 02138, USA}
%
% AO / NIRC2
\author[0000-0003-0967-2893]{E. A.~Petigura} % Y
\affiliation{Department of Physics \& Astronomy, University of California Los Angeles, Los Angeles, CA 90095, USA}
%
% AO / NIRC2
\author[0000-0001-8832-4488]{D. Huber} % R
\affiliation{Institute for Astronomy, University of Hawaii, 2680 Woodlawn Drive, Honolulu, HI 96822, USA}


\begin{abstract}
  The Gaia satellite is revitalizing our understanding of nearby
  open clusters and moving groups.  Here, we focus on the
  underappreciated \cn.  Based on rotation periods from TESS and
  lithium measurements from ground-based spectrographs, we find the
  age of the cluster to be $30\pm XX$\,Myr.  \sn\ is a binary system
  in the cluster, serendipitously observed by the Kepler satellite
  because the primary is nearby and Sun-like.  Kepler\,1627A was
  previously found to host a $3.56\pm 0.04\,R_\oplus$ mini-Neptune on a
  7.2\,day orbit.  We re-validate the existence of \pn, cementing it
  as the youngest planet with a well-measured age observed by the main
  Kepler mission.  Newly derived ages from Gaia offer the opportunity
  to significantly expand the census of age-dated planets -- we offer
  a literature compilation of young stars to enable this expansion.
  The properties of \pn\ are may also help clarify how the orbits and
  atmospheres of the mini-Neptune planets evolve.
\end{abstract}

\keywords{
  planetary evolution (XXXX),
  stellar associations (1582),
  open star clusters (1160),
	stellar ages (1581),
}

%%%%%%%%%%%%%%%%%%%%%%%%%%%%%%%%%%%%%%%%%%%%%%%%%%%%%%%%%%%%%%%%%%%%%%%%%%%%%%%


\section{Introduction}

At the time of the main Kepler
mission (2009--2013), only four open clusters were known in the Kepler
field: NGC\,6866, NGC\,6811, NGC\,6819,
and NGC\,6791, with ages spanning 0.7\,Gyr to 9\,Gyr
\citep{meibom_kepler_2011}.  


Section~\ref{sec:first}.  Section~\ref{sec:second}.  In
Section~\ref{sec:discussion}, we discuss.
Section~\ref{sec:conclusion} gives our conclusions.


\section{first}
\label{sec:first}

\begin{figure*}[t]
	\begin{center}
		\leavevmode
		\includegraphics[width=0.99\textwidth]{f2.pdf}
	\end{center}
	\vspace{-0.7cm}
  \caption{
  {\bf Kepler and TESS views of the $\delta$\,Lyr cluster.} Colored points are
  kinematically selected members of the $\delta$\,Lyr cluster (black points in
  Figure~\ref{fig:XYZvtang}).  Both Kepler (gray panels) and TESS (colored
  points) observed portions of the cluster.  Gray points are naked-eye stars
  ($m_{\rm V}<6.5$), three of which are annotated.  Kepler\,1627 (green star) was
  observed during the entirety of the Kepler mission, and has (so far) been observed
  for two lunar months by TESS.
  \label{fig:skychart}
  }
\end{figure*}



\section{second}
\label{sec:second}

\begin{figure*}[t]
	\begin{center}
		\leavevmode
		\includegraphics[width=1\textwidth]{f5.pdf}
	\end{center}
	\vspace{-0.7cm}
	\caption{
		{\bf Phase-folded transit of Kepler 1627b, with stellar
    variability removed.}  
    The 1-$\sigma$ model uncertainties (faint purple band) and the
    median model are shown.
		\label{fig:phasefold}
	}
\end{figure*}




\section{Discussion}
\label{sec:discussion}


\section{Conclusion}
\label{sec:conclusion}



%%%%%%%%%%%%%%%%%%%%%%%%%%%%%%%%%%%%%%%%%%%%%%%%%%%%%%%%%%%%%%%%%%%%%%%%%%%%%%%


%\clearpage
\acknowledgements
\raggedbottom

L.G.B{.} and J.L.C{.} are grateful to T{.}~David for help with the
transit fitting.
% is grateful to G.~Zhou, B{.}~Tofflemire, A{.}~McWilliam,
% E{.}~Newton, M{.}~Kounkel, A{.}~Kraus, L{.}~Hillenbrand, and
% K{.}~Hawkins for the discussions on young stars, rotation, and lithium
% that encouraged this analysis.
%
L.G.B{.} acknowledges support by the TESS GI Program, programs
G011103 and G022117, through NASA grants 80NSSC19K0386 and
80NSSC19K1728.
%
L.G.B{.} was also supported by a Charlotte Elizabeth Procter
Fellowship from Princeton University.
%
This study was based in part on observations at Cerro Tololo
Inter-American Observatory at NSF's NOIRLab (NOIRLab Prop{.} ID
2020A-0146; 2020B-0029 PI: Bouma), which is managed by the
Association of Universities for Research in Astronomy (AURA) under a
cooperative agreement with the National Science Foundation.
%
% ACKNOWLEDGE PFS / CAMPANAS.
%
This paper also includes data collected by the TESS mission, which are
publicly available from the Mikulski Archive for Space Telescopes
(MAST).
%
Funding for the TESS mission is provided by NASA's Science Mission
directorate.
%
We thank the TESS Architects (G.~Ricker, R.~Vanderspek, D.~Latham,
S.~Seager, J.~Jenkins) and the many TESS team members for their
efforts to make the mission a continued success.
%

%
% The Digitized Sky Survey was produced at the Space Telescope Science
% Institute under U.S. Government grant NAG W-2166.
% Figure~\ref{fig:scene} is based on photographic data obtained using
% the Oschin Schmidt Telescope on Palomar Mountain.
%

% %
% This research made use of the NASA Exoplanet Archive, which is
% operated by the California Institute of Technology, under contract
% with the National Aeronautics and Space Administration under the
% Exoplanet Exploration Program.
% %

% Resources supporting this work were provided by the NASA High-End
% Computing (HEC) Program through the NASA Advanced Supercomputing (NAS)
% Division at Ames Research Center for the production of the SPOC data
% products.
%

% A.J.\ and R.B.\ acknowledge support from project IC120009 ``Millennium
% Institute of Astrophysics (MAS)'' of the Millenium Science Initiative,
% Chilean Ministry of Economy. A.J.\ acknowledges additional support
% from FONDECYT project 1171208.  J.I.V\ acknowledges support from
% CONICYT-PFCHA/Doctorado Nacional-21191829.  R.B.\ acknowledges support
% from FONDECYT Post-doctoral Fellowship Project 3180246.
% %
% C.T.\ and C.B\ acknowledge support from Australian Research Council
% grants LE150100087, LE160100014, LE180100165, DP170103491 and
% DP190103688.
% %
% C.Z.\ is supported by a Dunlap Fellowship at the Dunlap Institute for
% Astronomy \& Astrophysics, funded through an endowment established by
% the Dunlap family and the University of Toronto.
% %
% D.D.\ acknowledges support through the TESS Guest Investigator Program
% Grant 80NSSC19K1727.
%
%
%
% %
% Based on observations obtained at the Gemini Observatory, which is
% operated by the Association of Universities for Research in Astronomy,
% Inc., under a cooperative agreement with the NSF on behalf of the
% Gemini partnership: the National Science Foundation (United States),
% National Research Council (Canada), CONICYT (Chile), Ministerio de
% Ciencia, Tecnolog\'{i}a e Innovaci\'{o}n Productiva (Argentina),
% Minist\'{e}rio da Ci\^{e}ncia, Tecnologia e Inova\c{c}\~{a}o (Brazil),
% and Korea Astronomy and Space Science Institute (Republic of Korea).
% %
% Observations in the paper made use of the High-Resolution Imaging
% instrument Zorro at Gemini-South. Zorro was funded by the NASA
% Exoplanet Exploration Program and built at the NASA Ames Research
% Center by Steve B. Howell, Nic Scott, Elliott P. Horch, and Emmett
% Quigley.
% %
% This research has made use of the VizieR catalogue access tool, CDS,
% Strasbourg, France. The original description of the VizieR service was
% published in A\&AS 143, 23.
% %
% This work has made use of data from the European Space Agency (ESA)
% mission {\it Gaia} (\url{https://www.cosmos.esa.int/gaia}), processed
% by the {\it Gaia} Data Processing and Analysis Consortium (DPAC,
% \url{https://www.cosmos.esa.int/web/gaia/dpac/consortium}). Funding
% for the DPAC has been provided by national institutions, in particular
% the institutions participating in the {\it Gaia} Multilateral
% Agreement.
%
% (Some of) The data presented herein were obtained at the W. M. Keck
% Observatory, which is operated as a scientific partnership among the
% California Institute of Technology, the University of California and
% the National Aeronautics and Space Administration. The Observatory was
% made possible by the generous financial support of the W. M. Keck
% Foundation.
% The authors wish to recognize and acknowledge the very significant
% cultural role and reverence that the summit of Maunakea has always had
% within the indigenous Hawaiian community.  We are most fortunate to
% have the opportunity to conduct observations from this mountain.
%
% \newline
%

\software{
  %\texttt{arviz} \citep{arviz_2019},
  \texttt{astrobase} \citep{bhatti_astrobase_2018},
  %\texttt{astroplan} \citep{astroplan2018},
	%\texttt{AstroImageJ} \citep{collins_astroimagej_2017},
  \texttt{astropy} \citep{astropy_2018},
  \texttt{astroquery} \citep{astroquery_2018},
  %\texttt{BATMAN} \citep{kreidberg_batman_2015},
  %\texttt{ceres} \citep{brahm_2017_ceres},
  %\texttt{cdips-pipeline} \citep{bhatti_cdips-pipeline_2019},
  \texttt{corner} \citep{corner_2016},
  %\texttt{emcee} \citep{foreman-mackey_emcee_2013},
  \texttt{exoplanet} \citep{exoplanet:exoplanet}, and its
  dependencies \citep{exoplanet:agol20, exoplanet:kipping13, exoplanet:luger18,
   	exoplanet:theano},
	%\texttt{gala} \citep{gala,PriceWhelan_2017_gala_zenodo},
	%\texttt{IDL Astronomy User's Library} \citep{landsman_1995},
  \texttt{IPython} \citep{perez_2007},
	%\texttt{isochrones} \citep{morton_2015_isochrones},
	%\texttt{lightkurve} \citep{lightkurve_2018},
  \texttt{matplotlib} \citep{hunter_matplotlib_2007}, 
  %\texttt{MESA} \citep{paxton_modules_2011,paxton_modules_2013,paxton_modules_2015}
  \texttt{numpy} \citep{walt_numpy_2011}, 
  \texttt{pandas} \citep{mckinney-proc-scipy-2010},
  %\texttt{pyGAM} \citep{serven_pygam_2018_1476122},
  \texttt{PyMC3} \citep{salvatier_2016_PyMC3},
  %\texttt{radvel} \citep{fulton_radvel_2018},
  %\texttt{scikit-learn} \citep{scikit-learn},
  \texttt{scipy} \citep{jones_scipy_2001},
  \texttt{TESS-point}  \citep{burke_2020},
  %\texttt{tesscut} \citep{brasseur_astrocut_2019},
	%\texttt{VESPA} \citep{morton_efficient_2012,vespa_2015},
  %\texttt{webplotdigitzer} \citep{rohatgi_2019},
  \texttt{wotan} \citep{hippke_wotan_2019}.
}
\ 

\facilities{
 	{\it Astrometry}:
 	Gaia \citep{gaia_collaboration_gaia_2018,gaia_collaboration_2020_edr3}.
 	{\it Imaging}:
    Second Generation Digitized Sky Survey. %,
    %SOAR~(HRCam; \citealt{tokovinin_ten_2018}).
 	%Keck:II~(NIRC2; \url{www2.keck.hawaii.edu/inst/nirc2}).
 	%Gemini:South~(Zorro; \citealt{scott_nessi_2018}.
 	{\it Spectroscopy}:
	CTIO1.5$\,$m~(CHIRON; \citealt{tokovinin_chironfiber_2013}),
  %PFS ({\bf CITE}),
  %  MPG2.2$\,$m~(FEROS; \citealt{kaufer_commissioning_1999}),
	%AAT~(Veloce; \citealt{gilbert_veloce_2018}).
	AAT~(HERMES; \citealt{lewis_2002_hermers_2df,sheinis_2015_hermes}),
 	%Keck:I~(HIRES; \citealt{vogt_hires_1994}).
 	VLT:Kueyen~(FLAMES; \citealt{pasquini_2002}).
% 	Euler1.2m~(CORALIE),
% 	ESO:3.6m~(HARPS; \citealt{mayor_setting_2003}).
 	{\it Photometry}:
%	  ASTEP:0.40$\,$m (ASTEP400),
% 	CTIO:1.0m (Y4KCam),
% 	Danish 1.54m Telescope,
%	  El Sauce:0.356$\,$m,
% 	Elizabeth 1.0m at SAAO,
% 	Euler1.2m (EulerCam),
% 	Magellan:Baade (MagIC),
% 	Max Planck:2.2m	(GROND; \citealt{greiner_grond7-channel_2008})
% 	NTT,
% 	SOAR (SOI),
 	TESS \citep{ricker_transiting_2015}.
% 	TRAPPIST \citep{jehin_trappist_2011},
% 	VLT:Antu (FORS2).
}

% \input{TOI837_phot_table.tex}
% \input{TOI837_rv_table.tex}
% \input{ic2602_ages.tex}
% \begin{table*}
\scriptsize
\setlength{\tabcolsep}{2pt}
\centering
\caption{Literature and Measured Properties for Kepler$\,$1627 A}
\label{tab:starparams}
%\tablenum{2}
\begin{tabular}{llcc}
  \hline
  \hline
Other identifiers\dotfill & \\
\multicolumn{3}{c}{TIC 120105470} \\
\multicolumn{3}{c}{GAIADR2 2103737241426734336} \\
\multicolumn{3}{c}{GAIAEDR3 2103737241426734336} \\
\hline
\hline
Parameter & Description & Value & Source\\
\hline 
$\alpha_{J2015.5}$\dotfill	&Right Ascension (hh:mm:ss)\dotfill & 18:56:13.6 & 1	\\
$\delta_{J2015.5}$\dotfill	&Declination (dd:mm:ss)\dotfill & +41:34:36.22 & 1	\\
%$l_{J2015.5}$\dotfill	&Galactic Longitude (deg)\dotfill & 288.2644 & 1	\\
%$b_{J2015.5}$\dotfill	&Galactic Latitude (deg)\dotfill & -5.7950 & 1	\\
%\\
%$NUV$\dotfill           & GALEX $NUV$ mag.\dotfill & 13.804 $\pm$ 0.004 & 2 \\
%$FUV$\dotfill           & GALEX $FUV$ mag.\dotfill & 18.466 $\pm$ 0.056 & 2 \\
\\
%B\dotfill			&Johnson B mag.\dotfill & 11.119 $\pm$ 0.107		& 2	\\
V\dotfill			&Johnson V mag.\dotfill & 13.11 $\pm$ 0.08		& 2	\\
%$B$\tablenote{The uncertainties of the photometry have a systematic error floor applied. Even still, the global fit requires a significant scaling of the uncertainties quoted here to be consistent with our model, suggesting they are still significantly underestimated for one or more of the broad band magnitudes}\dotfill		& APASS Johnson $B$ mag.\dotfill	& 13.001 $\pm$	0.02& 2	\\
%$V$\dotfill		& APASS Johnson $V$ mag.\dotfill	& 11.808 $\pm$	0.02& 2	\\
%\\
${\rm G}$\dotfill     & Gaia $G$ mag.\dotfill     & 13.049$\pm$0.02 & 1\\
%${\rm Bp}$\dotfill     & Gaia $Bp$ mag.\dotfill     & 10.695 $\pm$0.020 & 1\\
%${\rm Rp}$\dotfill     & Gaia $Rp$ mag.\dotfill     & 9.887$\pm$0.020 & 1\\
${\rm T}$\dotfill     & TESS $T$ mag.\dotfill     & 12.53$\pm$0.02 & 2\\
%$u'$\dotfill        & Sloan $u'$ mag.\dotfill & 14.706 $\pm$ 0.006& 3\\
%$g'$\dotfill		& APASS Sloan $g'$ mag.\dotfill	& 12.407 $\pm$ 0.02	&  2	\\
%$r'$\dotfill		& APASS Sloan $r'$ mag.\dotfill	& 11.311 $\pm$ 0.02	&  2	\\
%$i'$\dotfill		& APASS Sloan $i'$ mag.\dotfill	& 10.927 $\pm$ 0.04 &  2	\\
%\\
J\dotfill			& 2MASS J mag.\dotfill & 11.69  $\pm$ 0.02	& 3	\\
H\dotfill			& 2MASS H mag.\dotfill & 11.30 $\pm$ 0.02	    &  3	\\
K$_{\rm S}$\dotfill			& 2MASS ${\rm K_S}$ mag.\dotfill & 11.19 $\pm$ 0.02 &  3	\\
%\\
%W1\dotfill		& WISE1 mag.\dotfill & 8.901 $\pm$ 0.023 & 4	\\
%W2\dotfill		& WISE2 mag.\dotfill & 8.875 $\pm$ 0.021 &  4 \\
%W3\dotfill		& WISE3 mag.\dotfill &  8.875 $\pm$ 0.020& 4	\\
%W4\dotfill		& WISE4 mag.\dotfill & 8.936 $\pm$ N/A &  4	\\
\\
$\pi$\dotfill & Gaia EDR3 parallax (mas) \dotfill & 3.009 $\pm$ 0.032 &  1 \\
$d$\dotfill & Distance (pc)\dotfill & $329.5 \pm 3.5$ & 1, 4 \\
$\mu_{\alpha}$\dotfill		& Gaia EDR3 proper motion\dotfill & 1.716 $\pm$ 0.034 & 1 \\
                    & \hspace{3pt} in RA (mas yr$^{-1}$)	&  \\
$\mu_{\delta}$\dotfill		& Gaia EDR3 proper motion\dotfill 	&  -1.315 $\pm$ 0.034 &  1 \\
                    & \hspace{3pt} in DEC (mas yr$^{-1}$) &  \\
RUWE\dotfill		& Gaia EDR3 renormalized\dotfill 	&  2.899 &  1 \\
                    & \hspace{3pt} unit weight error &  \\
%
\\
RV\dotfill & Systemic radial \hspace{9pt}\dotfill  & $-14.3 \pm 1.0$ & 5 \\
                    & \hspace{3pt} velocity (\kms)  & \\
Spec. Type\dotfill & Spectral Type\dotfill & 	G8V & 5 \\
$v\sin{i_\star}$\dotfill &  Rotational velocity (\kms) \hspace{9pt}\dotfill &  18.9 $\pm$ 1.0 & 5 \\
Li EW\dotfill & 6708\AA\ Equiv{.} Width (m\AA) \dotfill & $233^{+5}_{-7}$  & 5 \\
%$v_{\rm mac}$\dotfill &  Macroturbulence velocity (\kms) \hspace{9pt}\dotfill &  8.4 $\pm$ 2.9 & 5 \\
%${\rm [Fe/H]}$\dotfill &   Metallicity$^\dagger$ \hspace{9pt}\dotfill & -0.02 $\pm$ 0.09 & 5 \\
%$T_{\rm eff}$\dotfill &  Effective Temperature (K) \hspace{9pt}\dotfill & 5777 $\pm$ 110 &  5  \\
%$\log{g_{\star}}$\dotfill &  Surface Gravity (cgs)\hspace{9pt}\dotfill &  4.6 $\pm$ 0.1  &  5 \\
$T_{\rm eff}$\dotfill &  Effective Temperature (K) \hspace{9pt}\dotfill & 5505 $\pm$ 39 &  6  \\
$\log{g_{\star}}$\dotfill &  Surface Gravity (cgs)\hspace{9pt}\dotfill &  4.53 $\pm$ 0.02  &  6 \\
%
% $E(B-V)$\dotfill & Reddening (mag)\dotfill & $0.06 \pm 0.02$ & 9 \\
%
%
$R_\star$\dotfill & Stellar radius ($R_\odot$)\dotfill & 0.881$\pm$0.018 & 6 \\
$M_\star$\dotfill & Stellar mass ($R_\odot$)\dotfill & 0.953$\pm$0.019 & 6 \\
%$F_{\rm bol}$\dotfill & Stellar bolometric flux (cgs)\dotfill & (1.967$\pm$0.046)$\times10^{-9}$ & 9 \\
%
%
$A_{\rm V}$\dotfill & Interstellar reddening (mag)\dotfill & 0.2$\pm$0.1 & 6 \\
${\rm [Fe/H]}$\dotfill &   Metallicity \hspace{9pt}\dotfill & 0.1 $\pm$ 0.1 & 6 \\
%
$P_{\rm rot}$\dotfill & Rotation period (d)\dotfill & $2.642\pm 0.042$  & 7 \\
Age & Adopted stellar age (Myr)\dotfill & $38^{+6}_{-5}$  &  8 \\
% $U^{*}$\dotfill & Space Velocity (\kms)\dotfill & $26.24 \pm 0.46$  & \S\ref{sec:uvw} \\
% $V$\dotfill       & Space Velocity (\kms)\dotfill & $-71.52 \pm 1.68$ & \S\ref{sec:uvw} \\
% $W$\dotfill       & Space Velocity (\kms)\dotfill & $ -1.31 \pm 0.27$ & \S\ref{sec:uvw} \\
\hline
\end{tabular}
\begin{flushleft}
 \footnotesize{ \textsc{NOTE}---
Provenances are:
$^1$\citet{gaia_collaboration_gaia_2018},
$^2$\citet{stassun_TIC8_2019},
$^3$\citet{skrutskie_tmass_2006},
$^4$\citet{Lindegren_2021_offset},
$^5$HIRES spectra and \citet{yee_SM_2017},
$^6$Cluster isochrone (MIST+PARSEC),
%$^7$FEROS spectra,
$^7$\citet{capitanio_threedimensional_2017} and \citet{lallement_threedimensional_2018},
$^8$Kepler light curve,
$^9$Pre-main-sequence CMD, with LDB age for IC~2602 being most
important (Section~\ref{sec:cmd}).
%$*$ $U$ is in the direction of the Galactic center. \\
%$^{10}$Method~1 (photometric SED fit, Section~\ref{subsec:starparams}).}
}
\end{flushleft}
\vspace{-0.5cm}
\end{table*}

% \input{compparams.tex}



% Table of best fit parameters
%\startlongtable
\begin{deluxetable*}{lllrrrrrrr}
%
  \tablecaption{ Priors and Posteriors for Model Fitted to the Long
  Cadence Kepler 1627Ab Light Curve.}
\label{tab:posterior}
%
\tabletypesize{\scriptsize}
%\tabletypesize{\small}
%
%\tablenum{2}
%
\tablehead{
  \colhead{Param.} & 
  \colhead{Unit} &
  \colhead{Prior} & 
  \colhead{Median} & 
  \colhead{Mean} & 
  \colhead{Std{.} Dev.} &
  \colhead{3\%} &
  \colhead{97\%} &
  \colhead{ESS} &
  \colhead{$\hat{R}-1$}
}

%/Users/luke/Dropbox/proj/rudolf/results/run_RotGPtransit/Kepler_1627_RotGPtransit_posteriortable.tex
\startdata
{\it Sampled} & & & & & & & & & \\
\hline
$P$ & d & $\mathcal{N}(7.20281; 0.01000)$ & 7.2028038 & 7.2028038 & 0.0000073 & 7.2027895 & 7.2028168 & 7464 & 3.9e-04 \\
$t_0^{(1)}$ & d & $\mathcal{N}(120.79053; 0.02000)$ & 120.7904317 & 120.7904254 & 0.0009570 & 120.7886377 & 120.7921911 & 3880 & 2.0e-03 \\
$\log \delta$ & -- & $\mathcal{N}(-6.3200; 2.0000)$ & -6.3430 & -6.3434 & 0.0354 & -6.4094 & -6.2767 & 6457 & 3.0e-04 \\
$b^{(2)}$ & -- & $\mathcal{U}(0.000; 1.000)$ & 0.4669 & 0.4442 & 0.2025 & 0.0662 & 0.8133 & 1154 & 1.6e-03 \\
$u_1$ & -- & \citet{exoplanet:kipping13} & 0.271 & 0.294 & 0.190 & 0.000 & 0.628 & 3604 & 1.5e-03 \\
$u_2$ & -- & \citet{exoplanet:kipping13} & 0.414 & 0.377 & 0.326 & -0.240 & 0.902 & 3209 & 1.4e-03 \\
$R_\star$ & $R_\odot$ & $\mathcal{N}(0.881; 0.018)$ & 0.881 & 0.881 & 0.018 & 0.847 & 0.915 & 8977 & 3.1e-04 \\
$\log g$ & cgs & $\mathcal{N}(4.530; 0.050)$ & 4.532 & 4.533 & 0.051 & 4.435 & 4.627 & 6844 & 1.6e-03 \\
$\langle f \rangle$ & -- & $\mathcal{N}(0.000; 0.100)$ & -0.0003 & -0.0003 & 0.0001 & -0.0005 & -0.0000 & 8328 & 1.1e-03 \\
$e^{(3)}$ & -- & \citet{vaneylen19} & 0.154 & 0.186 & 0.152 & 0.000 & 0.459 & 1867 & 2.0e-03 \\
$\omega$ & rad & $\mathcal{U}(0.000; 6.283)$ & 0.055 & 0.029 & 1.845 & -3.139 & 2.850 & 3557 & 8.6e-05 \\
$\log \sigma_f$ & -- & $\mathcal{N}(\log\langle \sigma_f \rangle; 2.000)$ & -8.035 & -8.035 & 0.008 & -8.049 & -8.021 & 9590 & 3.9e-04 \\
$\sigma_{\mathrm{rot}}$ & d$^{-1}$ & $\mathrm{InvGamma}(1.000; 5.000)$ & 0.070 & 0.070 & 0.001 & 0.068 & 0.072 & 9419 & 1.4e-03 \\
$\log P_{\mathrm{rot}}$ & $\log (\mathrm{d})$ & $\mathcal{N}(0.958; 0.020)$ & 0.978 & 0.978 & 0.001 & 0.975 & 0.980 & 8320 & 2.2e-04 \\
$\log Q_0$ & -- & $\mathcal{N}(0.000; 2.000)$ & -0.327 & -0.326 & 0.043 & -0.407 & -0.246 & 9659 & 2.7e-04 \\
$\log \mathrm{d}Q$ & -- & $\mathcal{N}(0.000; 2.000)$ & 7.697 & 7.698 & 0.103 & 7.511 & 7.899 & 5824 & 3.7e-04 \\
$f$ & -- & $\mathcal{U}(0.010; 1.000)$ & 0.01006 & 0.01009 & 0.00009 & 0.01000 & 0.01025 & 4645 & 4.0e-04 \\
{\it Derived} & & & & & & & & & \\
\hline
$\delta$ & -- & -- & 0.001759 & 0.001759 & 0.000062 & 0.001641 & 0.001875 & 6457 & 3.0e-04 \\
$R_{\rm p}/R_\star$ & -- & -- & 0.039 & 0.039 & 0.001 & 0.037 & 0.042 & 1811 & 1.1e-03 \\
$\rho_\star$ & g$\ $cm$^{-3}$ & -- & 1.990 & 2.004 & 0.240 & 1.570 & 2.461 & 6905 & 2.1e-03 \\
$R_{\rm p}^{(4)}$ & $R_{\mathrm{Jup}}$ & -- & 0.337 & 0.338 & 0.014 & 0.314 & 0.367 & 2311 & 1.0e-03 \\
$R_{\rm p}^{(4)}$ & $R_{\mathrm{Earth}}$ & -- & 3.777 & 3.789 & 0.157 & 3.52 & 4.114 & 2311 & 1.0e-03 \\
$a/R_\star$ & -- & -- & 17.606 & 17.619 & 0.702 & 16.277 & 18.906 & 6905 & 2.1e-03 \\
$\cos i$ & -- & -- & 0.027 & 0.025 & 0.010 & 0.004 & 0.040 & 1312 & 1.2e-03 \\
$T_{14}$ & hr & -- & 2.841 & 2.843 & 0.060 & 2.734 & 2.958 & 3199 & 3.6e-04 \\
$T_{13}$ & hr & -- & 2.555 & 2.539 & 0.094 & 2.360 & 2.692 & 1960 & 1.4e-03 \\
\enddata
%
\tablecomments{
  ESS refers to the number of effective samples.
  $\hat{R}$ is the Gelman-Rubin convergence diagnostic.
  Logarithms in this table are base-$e$.
  $\mathcal{U}$ denotes a uniform distribution,
  and $\mathcal{N}$ a normal distribution.
  (1) The ephemeris is in units of BJDTDB - 2454833.
  (2) Although $\mathcal{U}(0,1+R_{\rm p}/R_\star)$ is formally
  correct, for this model we assumed a non-grazing transit to enable
  sampling in $\log \delta$.
  (3) The eccentricity vectors are sampled in the $(e\cos\omega,
  e\sin\omega)$ plane.
  (4) The true planet size is a factor of $((F_1+F_2)/F_1)^{1/2}$
  larger than that from the fit because of dilution from Kepler
  1627B, where $F_1$ is the flux from the primary, and $F_2$ is that
  from the secondary; the mean and standard deviation of $R_{\rm
  p}=3.817\pm0.158\,R_{\oplus}$ quoted in the text includes this correction,
  assuming $(F_1+F_2)/F_1\approx 1.015$.
}
\vspace{-0.3cm}
\end{deluxetable*}


\clearpage
\bibliographystyle{yahapj}                            
\bibliography{bibliography} 

\appendix
\section{Young, Age-Dated, and Age-Dateable Star Compilation}
\label{app:targetlist}

%% \begin{deluxetable}{} command tell LaTeX how many columns
%% there are and how to align them.
%\startlongtable
\begin{deluxetable*}{lll}
    
%% Keep a portrait orientation

%% Over-ride the default font size
%% Use Default (12pt)
\tabletypesize{\scriptsize}
%\tabletypesize{\small}

%% Use \tablewidth{?pt} to over-ride the default table width.
%% If you are unhappy with the default look at the end of the
%% *.log file to see what the default was set at before adjusting
%% this value.

%% This is the title of the table.
\tablecaption{Young, Age-dated, and Age-dateable Stars Within the
  Nearest Few Kiloparsecs.}
\label{tab:v06}

%% This command over-rides LaTeX's natural table count
%% and replaces it with this number.  LaTeX will increment 
%% all other tables after this table based on this number
%\tablenum{3}

%% The \tablehead gives provides the column headers.  It
%% is currently set up so that the column labels are on the
%% top line and the units surrounded by ()s are in the 
%% bottom line.  You may add more header information by writing
%% another line between these lines. For each column that requries
%% extra information be sure to include a \colhead{text} command
%% and remember to end any extra lines with \\ and include the 
%% correct number of &s.
\tablehead{
  \colhead{Parameter} &
  \colhead{Example Value} &
  \colhead{Description}
}

%% All data must appear between the \startdata and \enddata commands
%
% paste from
% /Users/luke/Dropbox/proj/rudolf/results/tables/v06_main_tableheader.tex
% via drivers/write_v06_main_tableheader.py
\startdata
          \texttt{source\_id} &                                          1709456705329541504 &                                              Gaia DR2 source identifier. \\
                  \texttt{ra} &                                                      247.826 &                                          Gaia DR2 right ascension [deg]. \\
                 \texttt{dec} &                                                       79.789 &                                              Gaia DR2 declination [deg]. \\
            \texttt{parallax} &                                                       35.345 &                                                 Gaia DR2 parallax [mas]. \\
     \texttt{parallax\_error} &                                                        0.028 &                                     Gaia DR2 parallax uncertainty [mas]. \\
                \texttt{pmra} &                                                       94.884 &      Gaia DR2 proper motion $\mu_\alpha \cos \delta$ [mas$\,$yr$^{-1}$]. \\
               \texttt{pmdec} &                                                      -86.971 &                  Gaia DR2 proper motion $\mu_\delta$ [mas$\,$yr$^{-1}$]. \\
  \texttt{phot\_g\_mean\_mag} &                                                         6.85 &                                                  Gaia DR2 $G$ magnitude. \\
 \texttt{phot\_bp\_mean\_mag} &                                                        6.409 &                                      Gaia DR2 $G_\mathrm{BP}$ magnitude. \\
 \texttt{phot\_rp\_mean\_mag} &                                                        7.189 &                                      Gaia DR2 $G_\mathrm{RP}$ magnitude. \\
             \texttt{cluster} &                  NASAExoArchive\_ps\_20210506,Uma,IR\_excess &                                   Comma-separated cluster or group name. \\
                 \texttt{age} &                                                 9.48,nan,nan &  Comma-separated logarithm (base-10) of reported$^{\rm a}$ age in years. \\
           \texttt{mean\_age} &                                                         9.48 &                           Mean (ignoring NaNs) of $\texttt{age}$ column. \\
       \texttt{reference\_id} &       NASAExoArchive\_ps\_20210506,Ujjwal2020,CottenSong2016 &                           Comma-separted provenance of group membership. \\
  \texttt{reference\_bibcode} &  2013PASP..125..989A,2020AJ....159..166U,2016ApJS..225...15C &                   ADS bibcode corresponding to $\texttt{reference\_id}$. \\
\enddata

%% Include any \tablenotetext{key}{text}, \tablerefs{ref list},
%% or \tablecomments{text} between the \enddata and 
%% \end{deluxetable} commands

%% General table comment marker
\tablecomments{
Table~\ref{tab:v06} is published in its entirety in a machine-readable
format.   This table is a concatenation of the studies listed in
Table~\ref{tab:metadata}.  One entry is shown for guidance regarding
form and content.  In this particular example, the star has a cold
Jupiter on a 16 year orbit, HD 150706b \citep{2012AA...545A..55B}.  An
infrared excess has been reported \citep{CottenSong2016}, and the star
was identified by \citet{Ujjwal2020} as a candidate UMa moving group
member ($\approx 400\,{\rm Myr}$; \citealt{mann_tess_2020}).  The
star's RV activity and TESS rotation period corroborate its youth.
}
\vspace{-0.5cm}
\end{deluxetable*}

%% \begin{deluxetable}{} command tell LaTeX how many columns
%% there are and how to align them.
%\startlongtable
\begin{deluxetable*}{lccc}
    
%% Keep a portrait orientation

%% Over-ride the default font size
%% Use Default (12pt)
\tabletypesize{\scriptsize}
%\tabletypesize{\small}
%\tabletypesize{\normal}

%% Use \tablewidth{?pt} to over-ride the default table width.
%% If you are unhappy with the default look at the end of the
%% *.log file to see what the default was set at before adjusting
%% this value.

%% This is the title of the table.
\tablecaption{Provenances of Young and Age-dateable Stars.}
\label{tab:metadata}

%% This command over-rides LaTeX's natural table count
%% and replaces it with this number.  LaTeX will increment 
%% all other tables after this table based on this number
%\tablenum{3}

%% The \tablehead gives provides the column headers.  It
%% is currently set up so that the column labels are on the
%% top line and the units surrounded by ()s are in the 
%% bottom line.  You may add more header information by writing
%% another line between these lines. For each column that requries
%% extra information be sure to include a \colhead{text} command
%% and remember to end any extra lines with \\ and include the 
%% correct number of &s.
\tablehead{
  \colhead{Reference} &
  \colhead{$N_{\rm Gaia}$} &
  \colhead{$N_{\rm Age}$} &
  \colhead{$N_{G_{\rm RP}<16}$}
}

%% All data must appear between the \startdata and \enddata commands
%
% paste from
% /Users/luke/Dropbox/proj/rudolf/results/tables/metadata_table_data.tex
% via drivers/write_metadata_table.py
\startdata
                           \citet{Kounkel2020}  &             987376 &            987376 &                775363 \\
                     \citet{CantatGaudin2020a}  &             433669 &            412671 &                269566 \\
                     \citet{CantatGaudin2018a}  &             399654 &            381837 &                246067 \\
                      \citet{KounkelCovey2019}  &             288370 &            288370 &                229506 \\
                     \citet{CantatGaudin2020b}  &             233369 &            227370 &                183974 \\
                           \citet{Zari2018} UMS &              86102 &                 0 &                 86102 \\
                  \citet{SIMBAD} $\texttt{Y*?}$ &              61432 &                 0 &                 45076 \\
                           \citet{Zari2018} PMS &              43719 &                 0 &                 38435 \\
\citet{GaiaCollaboration2018} $d>250\,{\rm pc}$ &              35506 &             31182 &                 18830 \\
                      \citet{CastroGinard2020}  &              33635 &             24834 &                 31662 \\
                              \citet{Kerr2021}  &              30518 &             25324 &                 27307 \\
                  \citet{SIMBAD} $\texttt{Y*O}$ &              28406 &                 0 &                 16205 \\
                        \citet{VillaVelez2018}  &              14459 &             14459 &                 13866 \\
                     \citet{CantatGaudin2019a}  &              11843 &             11843 &                  9246 \\
                        \citet{Damiani2019} PMS &              10839 &             10839 &                  9901 \\
                                \citet{Oh2017}  &              10379 &                 0 &                 10370 \\
                          \citet{Meingast2021}  &               7925 &              7925 &                  5878 \\
                 \citet{SIMBAD} $\texttt{pMS*}$ &               5901 &                 0 &                  3006 \\
\citet{GaiaCollaboration2018} $d<250\,{\rm pc}$ &               5378 &               817 &                  3968 \\
                           \citet{Kounkel2018}  &               5207 &              3740 &                  5207 \\
                        \citet{Ratzenbock2020}  &               4269 &              4269 &                  2662 \\
                  \citet{SIMBAD} $\texttt{TT*}$ &               4022 &                 0 &                  3344 \\
                        \citet{Damiani2019} UMS &               3598 &              3598 &                  3598 \\
                           \citet{Rizzuto2017}  &               3294 &              3294 &                  2757 \\
            \citet{NASAExoArchive_ps_20210506}  &               3107 &               868 &                  3098 \\
                              \citet{Tian2020}  &               1989 &              1989 &                  1394 \\
                           \citet{Goldman2018}  &               1844 &              1844 &                  1783 \\
                        \citet{CottenSong2016}  &               1695 &                 0 &                  1693 \\
                            \citet{Gagne2018a}  &               1429 &                 0 &                  1389 \\
             \citet{RoserSchilbach2020} Psc-Eri &               1387 &              1387 &                  1107 \\
            \citet{RoserSchilbach2020} Pleiades &               1245 &              1245 &                  1019 \\
                  \citet{SIMBAD} $\texttt{TT?}$ &               1198 &                 0 &                   853 \\
                            \citet{Gagne2018c}  &                914 &                 0 &                   913 \\
                          \citet{Pavlidou2021}  &                913 &               913 &                   504 \\
                            \citet{Gagne2018b}  &                692 &                 0 &                   692 \\
                            \citet{Ujjwal2020}  &                563 &                 0 &                   563 \\
                             \citet{Gagne2020}  &                566 &               566 &                   351 \\
                      \citet{EsplinLuhman2019}  &                377 &               443 &                   296 \\
                     \citet{Roccatagliata2020}  &                283 &               283 &                   232 \\
                          \citet{Meingast2019}  &                238 &               238 &                   238 \\
                 \citet{Furnkranz2019} Coma-Ber &                214 &               214 &                   213 \\
           \citet{Furnkranz2019} Neighbor Group &                177 &               177 &                   167 \\
                             \citet{Kraus2014}  &                145 &               145 &                   145 \\
\enddata

%% Include any \tablenotetext{key}{text}, \tablerefs{ref list},
%% or \tablecomments{text} between the \enddata and 
%% \end{deluxetable} commands

%% General table comment marker
\tablecomments{
Table~\ref{tab:metadata} describes the provenances for the young and
age-dateable stars in Table~\ref{tab:v06}.  $N_{\rm Gaia}$: number of
Gaia stars we parsed from the literature source.  $N_{\rm Age}$:
number of stars in the literature source with ages reported.
$N_{G_{\rm RP}<16}$: number of Gaia stars we parsed from the
literature source with either $G_{\rm RP}<16$, or a parallax S/N
exceeding 5 and a distance closer than 100\,pc.  The latter criterion
included a few hundred white dwarfs that would have otherwise been
neglected.  Some studies are listed multiple times if they contain
multiple tables.  \citet{SIMBAD} refers to the \texttt{SIMBAD}
database.
}
\vspace{-0.5cm}
\end{deluxetable*}


The \texttt{v0.5} CDIPS target catalog (Table~\ref{tab:v05}) includes
some important updates from previous versions.  As in
\citet{bouma_cdipsI_2019}, we collected membership information for
young, age-dated, or age-dateable stars from across the literature.
Table~\ref{tab:metadata} gives a list of the sources included, and
some brief summary statistics.

The first major important change is that the extent of analyses
performed on the Gaia data at the time of our compilation was wide and
deep enough that we opted to neglect pre-Gaia analyses, except in
cases for which spectroscopically confirmed samples of stars had been
collected.  The membership lists for instance of
\citet{Kharchenko_et_al_2013} and \citet{dias_proper_2014} (MWSC and
DAML) were no longer required, especially given their relatively high
field-star contamination rates compared to Gaia-derived membership
catalogs.

For any of the catalogs for which Gaia DR2 identifiers were not
immediately available, we either followed the spatial (plus
proper-motion) crossmatching procedures described in
\citet{bouma_cdipsI_2019}, or else we pulled the Gaia DR2 source
identifiers associated with the catalog from SIMBAD.  We consequently
opted to drop the $\texttt{ext\_catalog\_name}$ and $\texttt{dist}$
columns maintained in \citet{bouma_cdipsI_2019}, as these were only
populated for a small number of stars.

The most crucial parameters of a given star for our purposes are the
Gaia DR2 source identifier ($\texttt{source\_id}$), the cluster name
($\texttt{cluster}$), and the ($\texttt{age}$).  Given the
hierarchical nature of many stellar associations, we do not attempt to
resolve the cluster names to a single unique string.  The Orion
complex for instance, can be divided into almost one hundred kinematic
subgroups \citep{kounkel_apogee2_2018}.  Similar complexity applies to
the problem of determining homogeneous ages, which we do not attempt
to resolve.  Instead, we simply merged the cluster names and ages
reported by various authors together.

This means that our ``age'' column can be null, for cases in which the
original authors did not report an age, and a reference literature age
was not readily available.  Nonetheless, since we do generally prefer
stars with known ages, we made a few additional efforts to populate
this column.  When available, the age provenance is from the original
analysis of the cluster.  However, in a few cases we adopted other
ages when the string-based crossmatches on the ``cluster'' name was
straightforward.  In particular, we used the ages determined by
\citet{CantatGaudin2020b} to assign ages to the catalogs from
\citet{GaiaCollaboration2018}, \citet{CantatGaudin2018a},
\citet{CastroGinard2020}, and \citet{CantatGaudin2020a}.

The catalogs we included for which ages were not immediately available
were those of \citet{CottenSong2016}, \citet{Oh2017},
\citet{Zari2018}, \citet{Gagne2018a},
\citet{Gagne2018b},\citet{Gagne2018c}, and \citet{Ujjwal2020}.  While
in principle the moving group members discussed by
\citet{Gagne2018a,Gagne2018b,Gagne2018c} and \citet{Ujjwal2020} have
easily associated ages, our SIMBAD cross-matching lost the moving
group association from those studies, which should therefore be
recovered tools such as BANYAN
$\Sigma$.\footnote{\url{http://www.exoplanetes.umontreal.ca/banyan/banyansigma.php}}.
We also included the SIMBAD object identifiers $\texttt{TT*}$,
$\texttt{Y*O}, $\texttt{Y*?}, $\texttt{TT?}$, and $\texttt{pMS*}$.
Finally, we also included every star in the NASA Exoplanet Archive
$\texttt{ps}$ table that had a Gaia identifier available
\citep{NASAExoArchive_ps_20210506}.  If the age had finite
uncertainties, we also included it, since stellar ages determined
through the combination of isochrone-fitting and transit-derived
stellar densities typically have higher precision than from isochrones
alone.

The technical manipulations for the merging, cleaning, and joining
were performed using $\texttt{pandas}$
\citep{mckinney-proc-scipy-2010}.  The eventual crossmatch (using the
Gaia DR2 $\texttt{source\_id}$) against the Gaia DR2 archive was performed
asychronously on the Gaia archive
website\footnote{\url{https://gea.esac.esa.int/archive/}}.


\section{Kinematic Selection of $\delta$\,Lyr Cluster Members}

\begin{figure*}[t]
	\begin{center}
		\leavevmode
		\includegraphics[width=1\textwidth]{f1.pdf}
	\end{center}
	\vspace{-0.7cm}
	\caption{
		{\bf Galactic position and tangential velocities of the
			$\delta$\,Lyr cluster (also known as Theia 73 and Stephenson\,1).}  Points are
		candidate cluster members with $\varpi/\sigma_\varpi > 20$, reported
		to be in the group by \citet{kounkel_untangling_2019}.  We focus
		on stars in a small region (black points) in the kinematic
		vicinity of Kepler\,1627 (yellow star).  The other candidate cluster
		members (gray points) may or may not share the ages of the selected
		kinematic group.  The location of the Sun is ($\odot$) is shown.
		\label{fig:XYZvtang}
	}
\end{figure*}

Figure~\ref{fig:XYZvtang} shows members of the \cn\
reported
by \citet{kounkel_untangling_2019} to be in the group.
Galactic positions are determined and plotted only for stars with parallax
signal-to-noise exceeding 20.
The location of the Sun is shown on the plots.
The non-uniform ``clumps'' might be
an artifact of the data processing steps performed by \citet{kounkel_untangling_2019}.
We therefore only consider stars in the immediate kinematic group
around \sn. 
The tangential velocities relative to \sn\ are shown in the bottom right panel.
These are computed by assuming that every star has the same three-dimensional
spatial velocity as \sn, where we assume a systemic radial velocity of
$-16.7 \pm 0.2$\,\kms\ based on the reconnaissance spectra obtained by
A.~Howard on HIRES and D.~Latham on TRES.
The relevant projection effects are then taken into account,
as discussed by {\it e.g.}, \citet{Meingast2021} and {\bf L.~Bouma et al (2021, submitted)}.


\listofchanges

%\allauthors
\end{document}
